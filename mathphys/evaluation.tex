% !TEX ROOT = ../ersti.tex
\section{Lehrevaluation}
\label{eval}



\noindent Neben dem \hyperref[kummerkasten]{Kummerkasten} hast du gegen Mitte des Semesters mit der \emph{Evaluation} noch die Möglichkeit, deinen Profs und Tutorinnen Feedback zu geben. Das läuft so ab, dass irgendwann im Semester eine Mail von \textit{heiquality} mit Zugangslinks zu einem Onlineportal kommt. In diesem Onlineportal kann man dann alle seine Vorlesungen und Übungen bewerten.
Das Ganze dient zweierlei Zwecken: Die Studienkommissionen bekommen Einsicht in die Evaluationsergebnisse, sodass besonders schlechte (oder gute) Veranstaltungen mit den Dozentinnen nachbesprochen erden, um das Lehrniveau zu steigern bzw.\ zu halten. In der Physik wird auch regelmäßig ein Lehrpreis vergeben\footnote{\url{https://mathphys.info/w/evaluation-und-lehrpreis/}}. Zum Anderen werden in der Physik die Ergebnisse auch veröffentlicht (falls die Dozentin ihre Einverständnis gegeben hat), sodass du nach passenden Vorlesungen und Tutorien suchen kannst. Du findest die Ergebnisse der Physik der letzten beiden Semester z.B. im Fachschaftsraum, sowie im \gls{KIP}-Foyer, im Arbeitsraum und am Philosophenweg am Infobrett im Treppenhaus.


\begin{figure}[h]
    \begin{center}
        \includegraphics{bilder/eval_1.png}\\
        \includegraphics{bilder/eval_2.png}\\
        \includegraphics{bilder/eval_3.png}\\
        \includegraphics{bilder/eval_4.png}\\
        \includegraphics{bilder/eval_5.png}\\
    \end{center}
\end{figure}

\vfill
