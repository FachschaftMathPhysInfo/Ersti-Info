\section{Die Fachschaft MathPhys $\mu\varphi$}
Wir, die „Fachschaft“, sind ein Haufen von Studis quer durch alle Semester und auch immerhin zwei Fakultäten (Physik \& Astro sowie Mathe \& Info), die sich als unabhängige Vertreter aller Physiker, Mathematiker und Informatiker sieht. Wenngleich eine solche Vertretung von offizieller Seite aus nicht vorgesehen ist, so würde ohne unsere Arbeit wohl einiges anders sein als heute. Wie und warum Studis de facto mundtot sind, steht im entsprechenden Artikel über Hochschulpolitik.

\subsection*{Man kann uns nicht immer sehen, aber da sind wir trotzdem\dots}
Wie bereits erwähnt, sehen wir unsere Aufgabe in der Studierendenvertretung. Dabei kommt es uns insbesondere darauf an, dass wir die Qualität des Studiums an den Fachbereichen verbessern.
Wenn du dieses Heft in den Händen hälst, hast du sicher schon von unserer Erstsemestereinführung gehört oder daran teilgenommen. Das ist in der Regel der erste Berührungspunkt mit uns für einen Ersti an der Uni Heidelberg.
Danach passiert ein Großteil unserer Arbeit für den normalen Studierenden im Verborgenen. So zum Beispiel die Arbeit in den Gremien oder Kommissionen der jeweiligen Fachbereiche, wo wir uns als studentische Vertreter für das Wohl der Studis einsetzen. In den Studiengebührenkomissionen wird zum Beispiel über die Verwendung der Gelder für die Fakultäten beraten. Denn, wenn ihr schon zahlen müsst, dann soll es zumindest so verwendet werden, dass es euch was nützt.

In den Berufungskomissionen wird, wie der Name schon vermuten lässt, über die Berufung neuer Professoren an die Uni Heidelberg entschieden. Daneben gibt es auch noch die Studienkommissionen und die Fakultätsräte, in denen jeweils über den entsprechenden Fachbereich diskutiert und abgestimmt wird (z.B. Prüfungsordnungen)

Das, was dann wieder alle Studis mitbekommen und auch rege nutzen, ist unser Verleih von Altklausuren oder Prüfungsberichten für die Zwischenprüfungen. Das Problem ist logischerweise, dass wir nach der Umstellung recht wenig Bachelorklausuren haben. Von daher der Appell an euch: Bringt eure Klausuren ins Fachschaftsbüro und schreibt fleißig Prüfungsberichte, denn ihr habt ja schließlich auch davon profitiert.
Seit dem Sommersemester 2008 gibt es aus Studiengebühren finanzierte Skripte zu den laufenden Veranstaltungen. Um ein Skript zu bekommen, müsst ihr einfach während der Sprechstunden mit eurem Studiausweis in den Fachschaftsraum kommen.
Ein weiterer wichtiger Punkt ist die Vorlesungsevaluation. Seit einigen Jahren ist es gesetzlich festgelegt, dass Vorlesungen evaluiert, also bewertet werden müssen. Dazu verteilen wir in den Vorlesungen anonyme Fragebögen, werten diese aus und hängen die Ergebnisse in den Instituten aus. Das ist für uns zwar mit enorm viel Aufwand verbunden, hat aber  Vorteile gegenüber einer externen Evaluation. Ziel des Ganzen ist es, die Qualität der Lehre zu verbessern. Damit das auch so klappt, ist es ungemein wichtig, dass ihr die Eval ernst nehmt.

\subsection*{Mens sana in corpore sano oder auch: Der Student lebt nicht für die Vorlesung allein \dots}
Um Himmels willen: NEIN!!! Um mit den hartnäckigen Vorurteilen gegenüber der Zunft der Physiker und Mathematiker aufzuräumen (und natürlich um selbst Spaß zu haben) veranstalten wir einmal im Semester ein Volleyballturnier  und das legendäre MathPhysTheo-Fest. Das „Vollmond Turnier“ findet nachts statt und ihr tretet gemeinsam mit Freunden im Kampf um Ehre und den Siegerpokal (=\,Bierkasten) an. Aber Vorsicht: als Konkurrenten treten auch eure Profs an.
Die Teilnahme ist frei und auch Zuschauer werden immer gerne gesehen.


Zum MathPhysRom-Fest gibt es nicht allzu viel zu sagen außer: Kommt, tanzt, feiert und \dots Helft mit! Wir suchen immer Helfer in allen möglichen Bereichen, als Dankeschön bekommt ihr ein tolles T-Shirt, Getränke- und Garderobengutscheine und natürlich freien Eintritt! Keine Sorge, ihr habt trotz Helferschicht noch genug Zeit um euren Uni-Zettel-Kommilitonen-Frust rauszutanzen.


\dots apropos Frust: Ein paar Wochen nach Semesterbeginn richten wir für euch ein Frustcafé aus. Dort könnt ihr bei Kaffee, Keksen und Kuchen eure Sorgen über die Zettel und das Leben im Allgemeinen mit uns oder euren Leidensgenossen besprechen.
Wenn das dann nicht reichen sollte, gibt es immer noch unsere Caféte als Anlaufstelle.

Dies ist ein altes Gartenhäuschen im Philosophenweg 12, das wir im Sommer 2009 liebevoll renoviert haben. Neben Sofas und Tischchen stehen auch Kaffeemaschine, Wasserkocher und Tassen drin. Kaffee und Tee gibt es zum Selbstkostenpreis und ab und zu stehen auch Kekse oder Gummibärchen da.


Falls euch das noch nicht genug Informationen sind, ihr etwas nicht verstanden habt, oder ihr Interesse an Mitarbeit habt, dann schaut einfach vorbei und fragt. Noch besser ist es allerdings, wenn ihr in der Fachschaftssitzung vorbeischaut. Dort laufen sozusagen alle Fäden zusammen und es wird diskutiert, beschlossen und abgestimmt. Einmal im Semester fahren wir auf das \gls{FSWE} irgendwo hin, meist tief in den Odenwald. Das Wochenende dient dazu alle fachschaftsrelevanten Themen ausführlich in einzelnen AKs (Arbeitskreise) besprechen zu können und natürlich nicht zuletzt um einander besser kennen zu lernen und Spaß zu haben. Für Erstis ist die Teilnahme kostenlos und von der Verpflegung könnte sich die Mensa mehr als nur eine Scheibe abschneiden.
Haben wir euer Interesse geweckt? Gut, dann sehen wir uns am Mittwoch um 18\,\gls{c.t.}!

\begin{center}
\large
\textbf{Fachschaftssitzung}

\textbf{jeden Mittwoch um 18 Uhr im Fachschaftsraum}
\end{center}
