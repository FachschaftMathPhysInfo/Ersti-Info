% !TEX ROOT = ../ersti.tex
Wir, die \emph{Fachschaft}, sind ein Haufen von Studis quer durch alle Semester und aus immerhin zwei Fakultäten -- Physik \& Astro sowie Mathe \& Info -- die sich als basisdemokratisch organisierte, unabhängige Vertreterinnen der Physik-, Mathe- und Informatik-Studis sehen. Folglich sind alle Studis der beiden Fakultäten in der Fachschaft willkommen, redeberechtigt und stimmberechtigt. Genaueres zur Arbeitsweise der Fachschaft findet ihr auf den nächsten Seiten.

Im weiteren Sinne besteht die \emph{Fachschaft} aus allen Studierenden der Mathe, Physik oder Informatik. Ihr seid mit eurer Immatrikulation Teil der \emph{Fachschaft}, wenn auch nicht unbedingt aktiv beteiligt. Ihr habt das Recht, Entscheidungen zu fällen, nutzt es!

Um unsere Fachschaftsarbeit effizienter zu gestalten, haben wir uns dazu entschlossen, uns für inhaltliche Fragen in die Bereiche Physik, sowie Mathematik und Informatik aufzuteilen, Organisatorisches und Veranstaltungen allerdings gemeinsam durchzuführen. Daher treffen wir uns erst zu einer gemeinsamen Sitzung und gehen danach in die Einzelsitzungen. Diese Sitzungen sind der Dreh- und Angelpunkt unserer Arbeit, alles wichtige wird hier besprochen.

\begin{center}
    \large \textbf{Fachschaftssitzungen}

    \textbf{jeden Mittwoch um 18 Uhr \gls{c.t.}\footnote{c.t. heißt cum tempore und bedeutet, dass die Sitzung offiziell erst 15 Minuten nach der angegebenen Zeit beginnt im Gegensatz zu s.t. (sine tempore).}}

\end{center}
\begin{figure}[b]
  \ifthenelse{\boolean{druckversion}}{%
        \includegraphics[width=\linewidth]{fs-logo_bw.pdf}
    }{%
        \includegraphics[width=\linewidth]{fs-logo_4c.pdf}
    }

\end{figure}

\section[Alles verändert sich, wenn du es veränderst \dots]{Alles verändert sich, \\wenn du es veränderst \dots}
Wie bereits erwähnt, sehen wir unsere Aufgabe in der Studierendenvertretung. Dabei kommt es uns insbesondere darauf an, dass wir die von uns vertretenen Studiengänge verbessern. Darüber hinaus mischen wir uns manchmal auch in politische Fragestellungen an der Uni ein. Das alles geschieht auf vielerlei Weise.

Wenn ihr dieses Heft in den Händen haltet, habt ihr sicher schon von unserer \emph{Erstieinführung} gehört oder daran teilgenommen. Sie soll den Schock der ersten Uniwoche abmildern und ist in der Regel der erste Berührungspunkt mit uns für Erstis an der Uni Heidelberg.

Zu Beginn des Semesters werdet ihr in einigen Veranstaltungen \emph{Skripte} ausgeteilt bekommen. Diese drucken wir, wenn die Profs ein Skript bereitstellen, für die laufenden Veranstaltungen und organisieren die Verteilung. Wenn ihr die Skriptausgabe verpasst, kommt einfach in den Fachschaftsraum und fragt nach, ob es noch welche gibt.

Während des Semesters arbeiten wir aktiv in den \emph{Gremien und Kommissionen} der Fakultäten, wo wir die studentischen Interessen so gut vertreten, wie es die Umstände erlauben. In den \emph{Studienkommissionen} wird über Prüfungsordnungen, Zulassungsordnungen und generell alles, was die Lehre an der Fakultät als Ganzes betrifft, diskutiert. Auch über die Verwendung der \emph{Qualitätssicherungsmittel} wird in der Studienkommission beraten. Hier wird ausgearbeitet, wie das Studium ablaufen soll -- entsprechend wichtig ist es, dass wir uns dazu in den Fachschaftssitzungen eine Meinung bilden und diese in den Kommissionen vertreten. In den \emph{Berufungskommissionen}, in denen über die Anstellung neuer Professorinnen an die Uni Heidelberg entschieden wird, setzen wir uns dafür ein, dass gute Didaktikerinnen mit spannendem Lehrangebot berufen werden. Nach der Vorarbeit durch die Kommissionen diskutieren und entscheiden die \emph{Fakultätsräte} als fakultätsweite Gremien. Die 6 bzw.~8 studentischen Vertreterinnen in den Fakultätsräten der Mathe \& Info bzw.~Physik \& Astro werden jährlich von allen Studis gewählt und kommen aus der Fachschaft. Darüber hinaus erarbeiten und diskutieren wir universitätsweite Finanzierungs"= und Positionierungsanträge aus dem \gls{StuRa}.

Spätestens wenn sich das Semester zum Ende neigt, werdet ihr einen Service unsererseits zu schätzen wissen: Den Verleih von \emph{Altklausuren} oder Prüfungsberichten für die Zwischenprüfungen. Auch wenn die Umstellung auf Bachelor und Master bereits einige Semester zurück liegt und wir für viele Fächer Altklausuren und Berichte haben, freuen wir uns immer über neuen Input. Insbesondere bei nicht obligatorischen Veranstaltungen und im Master ist der Vorrat noch mager. Von daher der Appell an euch: Bringt eure Klausuren in den Fachschaftsraum und schreibt fleißig Prüfungsberichte, denn ihr habt ja schließlich auch davon profitiert.


% \subsection{Ich will lieber tanzen gehn \dots}
% Um mit den hartnäckigen Vorurteilen gegenüber der Zunft der Physikerinnen, Informatikerinnen und Mathematikerinnen aufzuräumen -- und natürlich um selbst Spaß zu haben -- veranstalten wir einmal im Semester das legendäre \emph{MathPhysTheo-Fest} und mehrere Spieleabende. Lange Zeit veranstalteten wir zusammen mit der Fachschaft Romanistik das legendäre Fest \glqq{}MathPhysRom\grqq{}. Seit 2011 erbeten wir göttlichen Beistand und feiern nun zusammen mit der Fachschaft Theologie. Auch wenn das \glqq{}MathPhysTheo\grqq{} nun einen neuen Namen trägt, ist es immer noch legendär, also: Kommt, tanzt, feiert und \dots Helft mit! Wir suchen immer Helferinnen in allen möglichen Bereichen. Als Dankeschön bekommt ihr ein tolles T-Shirt, Getränke- und Garderobengutscheine und natürlich freien Eintritt! Keine Sorge, ihr habt trotz Helferschicht noch genug Zeit, um euren Uni-Zettel-Frust rauszutanzen.

%Die Cafete kann momentan leider nicht genutzt werden, deshalb wird dieser Absatz auskommentiert.
%Wenn das dann nicht reichen sollte, gibt es immer noch unseren \emph{Fachschaftsraum INF 205, Raum 01.301}, als Anlaufstelle. Am Philweg gibts außerdem die Caféte. Dies ist ein altes Gartenhäuschen am Philosophenweg 12, das wir im Sommer 2009 liebevoll renoviert und 2013 wieder nutzbar gemacht zu haben. Neben Sofas und Tischchen stehen auch Kaffeemaschine, Wasserkocher und Tassen drin. Kaffee und Tee gibt es zum Selbstkostenpreis und ab und zu stehen auch Kekse oder Gummibärchen da. Zur Zeit ist sie leider eher selten geöffnet, aber zumindest Dienstags zu den \gls{StuRa}-Sitzungen und zur Caféten-Fete jeden ersten Donnerstag im Monat ist meistens jemand da.

Falls euch das noch nicht genug Informationen sind, ihr etwas nicht verstanden habt, oder ihr Interesse an Mitarbeit habt, dann schaut einfach vorbei und fragt. Noch besser ist es allerdings, wenn ihr in der Fachschaftssitzung vorbeischaut. Dort laufen sozusagen alle Fäden zusammen, es wird diskutiert und beschlossen.

Einmal im Semester fahren wir auf das \gls{FSWE} irgendwo hin, meist tief in den Odenwald. Das Wochenende dient dazu, alle fachschaftsrelevanten Themen ausführlich in einzelnen AKs (Arbeitskreisen) besprechen zu können und natürlich nicht zuletzt, um einander besser kennen zu lernen und Spaß zu haben. Für Erstis ist die Teilnahme kostenlos und von der Verpflegung könnte sich die Mensa mehr als nur eine Scheibe abschneiden.
