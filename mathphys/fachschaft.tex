\section{Die Fachschaft MathPhys}
Wir, die „Fachschaft“, sind ein Haufen von Studis quer durch alle Semester und auch immerhin zwei Fakultäten (Physik \& Astro sowie Mathe \& Info), die sich als basisdemokratisch organisierte, unabhängige Vertreterin der Physik"=, Mathe"= und Informatik"=Studis sieht. Folglich sind alle Studis der beiden Fakultäten in der Fachschaft willkommen, redeberechtigt und stimmberechtigt. Genaueres zur Arbeitsweise der Fachschaft findest du auf den nächsten Seiten.

Im weiteren Sinne sind die „Fachschaft“ alle Studierenden der Mathe, Physik oder Informatik. Du bist mit Deiner Immatrikulation Teil der „Fachschaft“, wenn auch nicht unbedingt aktiv beteiligt. Du hast das Recht, Entscheidungen zu fällen, nutze es.

Seit Einführung der Verfassten Studierendenschaft zum aktuellen Wintersemester befindet sich die Fachschaft allerdings in einem Umstrukturierungsprozess, da die Verfasste Studierendenschaft die bisherige Struktur nicht unterstützt. Bisher besteht die Umstrukturierung hauptsächlich aus der Aufspaltung der vormals gemeinsamen Sitzung in drei verschiedene Sitzungen, die der Dreh\= und Angelpunkt unsere Arbeit sind.

\begin{center}
\large
\textbf{Fachschaftssitzungen}

\textbf{jeden Mittwoch um 18 Uhr \gls{c.t.}}
\end{center}

\sidebar{
  \centering
  \chaptersidebarpushdown
%  \includegraphics[width=3cm]{bilder/fs-logo_small.pdf}
  \includegraphics[width=3cm]{bilder/fs-logo_bw.pdf}
}

\subsection*{Alles verändert sich, wenn du es veränderst\dots}
Wie bereits erwähnt, sehen wir unsere Aufgabe in der Studierendenvertretung. Dabei kommt es uns insbesondere darauf an, dass wir die von uns vertretenen Studiengänge verbessern. Darüber hinaus mischen wir uns manchmal auch in politische Fragestellungen an der Uni ein. Das alles geschieht auf vielerlei Weise.

Wenn du dieses Heft in den Händen hältst, hast du sicher schon von unserer Erstieinführung gehört oder daran teilgenommen. Sie soll den Schock der ersten Uniwoche abmildern und ist in der Regel der erste Berührungspunkt mit uns für eineN Ersti an der Uni Heidelberg.

Zu Beginn des Semesters wirst du in einigen Veranstaltungen Skripte ausgeteilt bekommen. Diese drucken wir, wenn die Profs ein Skript bereitstellen, für die laufenden Veranstaltungen seit dem Sommersemester 2008 aus Studiengebühren"=/Qualitätssicherungsmitteln. Wenn du die Skriptausgabe verpasst, komm einfach in den Fachschaftsraum und frag nach ob es noch welche gibt.

Während des Semesters arbeiten wir aktiv in den Gremien und Kommissionen der Fakultäten, wo wir die studentischen Interessen so gut vertreten, wie es die Umstände erlauben.
In den Studienkommissionen wird über Prüfungsordnungen, Zulassungsordnungen und generell alles was die Lehre an der Fakultät als ganzes betrifft, diskutiert. Auch über die Verwendung der Qualitätssicherungsmittel, welche die Studiengebührenmittel ersetzen, wird in der Studienkommission beraten. Hier wird ausgearbeitet, wie das Studium ablaufen soll -- entsprechend wichtig ist es, dass wir uns dazu in den Fachschaftssitzungen eine Meinung bilden und diese in den Kommissionen vertreten.
In den Berufungskommissionen, wo über die Berufung neuer ProfessorInnen an die Uni Heidelberg entschieden wird, setzen wir uns dafür ein, dass gute DidaktikerInnen mit spannendem Lehrangebot berufen werden.
Nach der Vorarbeit durch die Kommissionen diskutieren und entscheiden die Fakultätsräte als fakultätsweite Gremien. Die 5 bzw. 6 studentischen VertreterInnen in den Fakultätsräten der Mathe \& Info bzw. Physik \& Astro werden jährlich von allen Studis gewählt und kommen aus der Fachschaft.
Darüberhinaus erarbeiten und diskutieren wir universitätsweite Finanzierungs"= und Positionierungsanträge aus der Fachschaftskonferenz (FSK).

Spätestens wenn sich das Semester zum Ende neigt, werdet ihr einen Service unsererseits zu schätzen wissen: Den Verleih von Altklausuren oder Prüfungsberichten für die Zwischenprüfungen. Auch wenn die Umstellung auf Bachelor und Master bereits einige Semester zurück liegt und wir für viele Fächer Altklausuren und Berichte haben, freuen wir uns immer über neuen Input. Insbesondere bei nicht obligatorischen Veranstaltungen und im Master ist der Vorrat noch mager. Von daher der Appell an euch: Bringt eure Klausuren ins Fachschaftsbüro und schreibt fleißig Prüfungsberichte, denn ihr habt ja schließlich auch davon profitiert.


\subsection*{Ich will lieber tanzen gehn\dots}
Um mit den hartnäckigen Vorurteilen gegenüber der Zunft der PhysikerInnen und MathematikerInnen aufzuräumen (und natürlich um selbst Spaß zu haben) veranstalten wir einmal im Semester das legendäre MathPhysTheo-Fest, mehrere Spieleabende und ein Volleyball-Turnier.
Das „Vollmond Turnier“ findet nachts statt und ihr tretet gemeinsam mit FreundInnen im Kampf mit dem Ball um Ehre und Siegerpokal (=\,Bierkasten) an. Aber Vorsicht: als KonkurrentInnen treten auch eure Profs an. Die Teilnahme ist frei und auch Zuschauer werden immer gerne gesehen.
Lange Zeit verstalteten wir zusammen mit der Fachschaft Romanistik das legendäre Fest „MathPhysRom“. Seit 2011 erbeten wir göttlichen Beistand und feiern nun zusammen mit der Fachschaft Theologie. Auch wenn das „MathPhysTheo“ nun einen neuen Namen trägt, ist es immernoch legendär, also: Kommt, tanzt, feiert und \dots Helft mit! Wir suchen immer HelferInnen in allen möglichen Bereichen. Als Dankeschön bekommt ihr ein tolles T-Shirt, Getränke- und Garderobengutscheine und natürlich freien Eintritt! Keine Sorge, ihr habt trotz Helferschicht noch genug Zeit um euren Uni-Zettel-Frust rauszutanzen.

Wenn das dann nicht reichen sollte, gibt es immer noch unseren Fachschaftsraum INF 305, Raum 045, als Anlaufstelle.
Am Philweg gibts außerdem die Caféte. Dies ist ein altes Gartenhäuschen im Philosophenweg 12, das wir im Sommer 2009 liebevoll renoviert und 2013 wieder nutzbargemacht zu haben. Neben Sofas und Tischchen stehen auch Kaffeemaschine, Wasserkocher und Tassen drin. Kaffee und Tee gibt es zum Selbstkostenpreis und ab und zu stehen auch Kekse oder Gummibärchen da.

Falls euch das noch nicht genug Informationen sind, ihr etwas nicht verstanden habt, oder ihr Interesse an Mitarbeit habt, dann schaut einfach vorbei und fragt. Noch besser ist es allerdings, wenn ihr in der Fachschaftssitzung vorbeischaut. Dort laufen sozusagen alle Fäden zusammen, es wird diskutiert und beschlossen.

Einmal im Semester fahren wir auf das \gls{FSWE} irgendwo hin, meist tief in den Odenwald. Das Wochenende dient dazu alle fachschaftsrelevanten Themen ausführlich in einzelnen AKs (Arbeitskreise) besprechen zu können und natürlich nicht zuletzt um einander besser kennen zu lernen und Spaß zu haben. Für Erstis ist die Teilnahme kostenlos und von der Verpflegung könnte sich die Mensa mehr als nur eine Scheibe abschneiden.
