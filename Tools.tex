\documentclass[12pt]{article}
\usepackage{ngerman}
\usepackage[utf8]{inputenc}
\usepackage[arrow, matrix, curve]{xy}
\usepackage{ragged2e}
\usepackage{array}
\usepackage{amssymb}
\usepackage{enumitem}
\usepackage{amsmath}
\usepackage{amsfonts}
\usepackage{amssymb}
\usepackage{makeidx}
\usepackage{mathtools}
\usepackage{listings}

\usepackage[a4paper,left=2cm,right=2cm,top=2cm,bottom=2.5cm]{geometry}
\setlength{\parindent}{0pt}

\usepackage{tabularx}
\newcolumntype{L}[1]{>{\raggedright\arraybackslash}p{#1}} % linksbündig mit Breitenangabe
\newcolumntype{C}[1]{>{\centering\arraybackslash}p{#1}} % zentriert mit Breitenangabe
\newcolumntype{R}[1]{>{\raggedleft\arraybackslash}p{#1}} % rechtsbündig mit Breitenangabe

\begin{document}
	Damit ihr auf einen Blick sehen könnt, welches Tool wofür verwendet wird, findet ihr hier eine kurze Zusammenfassung. Tools, die als Funktion Onlinevorlesungen aufweisen, wurden im vergangenen Semester verwendet, es ist also möglich, dass in diesem Semester auch andere Tools verwendet werden. Welche Tools eure Professorinnen und Tutorinnen letztendlich verwenden werden, werden sie euch dann rechtzeitig mitteilen. \\
	
	\begin{tabular}{| L{2cm} | L{4cm} | L{9cm}|}
		\hline
		\textbf{Tool} & \textbf{Was ist es?} &  \textbf{Wofür wird es verwendet?} \\ \hline
		Cisco Webex & Videokonferenztool & Onlinevorlesungen, Seminare, Sprechstunden, Tutorien \\ \hline
		
		heiconf & Uniinternes Videokonferenztool & Onlinevorlesungen, Seminare, Sprechstunden, Tutorien \\ \hline
		
		LSF &  Campus-Management-System der Uni & Alles, was mit Verwaltung zu tun hat: Vorlesungsverzeichnis, Informationen zu Personen, Gebäuden, etc., Rückmeldung durchführen, Bescheinigungen ausdrucken, Anmeldungen für Veranstaltungen und Klausuren (in anderen Studiengängen), ...\\ \hline
		
		Mampf & Mathematische Medienplattform & Vorlesungsvideos, Beispielvideos, Quizzes, weitere Lernangebote \\ \hline
		
		Microsoft Teams & Videokonferenztool & Onlinevorlesungen, Seminare, Sprechstunden, Tutorien \\ \hline
		
		Moodle & Lernplattform & Lernmaterialien, Zettelabgabe, Punkteeinsicht, Klausuranmeldung, organisatorische Informationen \\ \hline
		
		Müsli & Tool zum Organisieren von Übungsgruppen & Eintragen in Übungsgruppen($\rightarrow$ teilweise Anmeldung zu Veranstaltungen), Punkte für Übungszettel, Klausuranmeldung, Einsicht der Klausurnote \\ \hline
		
		Twitch & Live-streaming-Videoportal & Onlinevorlesungen (in Kombination mit Chat-Tool) \\ \hline
		
		Zoom & Videokonferenztool & Onlinevorlesungen, Seminare, Sprechstunden, Tutorien \\ \hline
	\end{tabular}
\end{document}