% !TEX ROOT = ../ersti.tex
\section{EDV}

\subsection{Übersicht der digitalen Tools}
\label{digitale-tools-uebersicht}

Im Uni-Alltag benötigt ihr zahlreiche digitale Tools, die teilweise von der Universität selbst, teilweise von externen Anbietern gehostet werden. Dementsprechend sind sie manchmal frei zugänglich oder ihr benötigt einen Account; meist euren Uni-Account.

\begin{description}
    \item[\href{https://heico.uni-heidelberg.de/}{heiCO (Heidelberg Campus Online)}]
    Neues Campus-Management-System, das zukünftig für alle studentischen Angelegenheiten wie Bewerbung, Zulassung, Prüfungsanmeldung, Bescheinigungen etc. verwendet werden soll. Das System befindet sich aktuell im Aufbau und wird nach und nach eingeführt.

	\item[\href{https://lsf.uni-heidelberg.de/}{LSF (Lehre, Studium und Forschung)}]
    Informationssystem der Uni. Bietet aktuell alles, was mit Verwaltung zu tun hat: Vorlesungsverzeichnis, Informationen zu Personen, Gebäuden, Rückmeldung durchführen, Bescheinigungen ausdrucken. In Zukunft soll es (teilweise) durch heiCO ersetzt werden.

	\item[\href{https://moodle.uni-heidelberg.de}{Moodle}]
    E-Learning-Plattform der Uni, auf der in manchen Vorlesungen Lernmaterialien, Zettelabgaben, Punkteeinsicht, Klausuranmeldung und organisatorische Informationen angeboten werden.

	\item[\href{https://uebungen.physik.uni-heidelberg.de}{PhÜ (Physik-Übungruppen)}]
    Der Name ist Programm. Hier könnt ihr euch in der Physik in Übungsgruppen eintragen, zu Veranstaltungen anmelden sowie Punkte für Übungszettel und die Klausurnoten einsehen. Integriert Rocket-Chatrooms, die teilweise für die Kommunikation mit den Tutorinnen genutzt werden.

	\item[\href{https://muesli.mathi.uni-heidelberg.de}{MÜSLI}]
    In der Mathe und Info werden mit diesem System Übungsgruppen organisiert. Man kann sich in Übungsgruppen eintragen, zu Veranstaltungen und Klausuren anmelden und Punkte für Übungszettel und die Klausurnote einsehen.

	\item[\href{https://mampf.mathi.uni-heidelberg.de}{MaMpf (Mathematische Medienplattform)}]
    MaMpf bündelt und vernetzt verschiedene E-Learning-Angebote: Vorlesungsvideos und Skripte mit inhaltlicher Gliederung, Übungsblätter, eine umfangreiche Sammlung von Multiple-Choice-Fragen und Quizaufgaben inklusive angeleiteter Beweise und vieles mehr. Mehr Infos zu dieser Plattform auf \autopageref{mampf}.

	\item[\href{https://heiconf.uni-heidelberg.de}{heiCONF}]
    Uni-internes Videokonferenztool basierend auf BigBlueButton.

	\item[\href{https://meet.jit.si/}{Jitsi Meet}]
    Open-source Videokonferenztool.
\end{description}


\subsection{Universitätsrechenzentrum (URZ)}
\label{urz}
Computer sind die Triebfeder unseres Zeitalters und auch im Studium kommt ihr nicht um sie herum. Das beginnt zum Beispiel schon damit, dass die meisten Übungszettel nur online erscheinen und selbst ausgedruckt werden müssen. Später möchtet ihr eure Bachelor- oder Masterarbeit bzw.\ eure Examensarbeit sicher nicht handschriftlich anfertigen, sondern lieber in \LaTeX \footnote{„Latech“ gesprochen. LibreOffice -- oder noch schlimmer -- Word helfen nicht weiter, weil man Formeln nur sehr umständlich eingeben kann. Außerdem schafft es \LaTeX{} Blocktexte ohne größere Freiräume und mit weniger Bindestrichen zu setzen, wie man z.\,B.\ an diesem Ersti-Info sehen kann. Wenn ihr \LaTeX{} lernen möchtet, haltet in den nächsten Semestern nach einem entsprechenden Kurs Ausschau oder nutzt eins der vielen Onlinetutorials.} setzen. Das \gls{URZ} bietet eine ganze Menge von Services an, die hier kurz erläutert werden sollen.

\subsection{Campus Card}
\label{campuscard}

\begin{figure}[b]
    \centering
    \includegraphics[width=.75\linewidth]{bilder/planning.png}
\end{figure}

Wie an den meisten Unis ist der Studierendenausweis auch in Heidelberg voll modern und multifunktional. Neben der Ausweisfunktion wird er zum Ausleihen von Büchern in der Universitätsbibliothek und als \emph{Geldkarte} zum Bezahlen sowohl in der Mensa als auch an Kopierern und an einigen Waschstationen des Studierendenwerks benutzt. Um diese Funktion nutzen zu können, muss man Geld auf den Ausweis laden, dafür stehen in den Mensen Automaten zur Verfügung. Zusätzlich gilt der Ausweis im \gls{VRN} werktags ab 19 Uhr und an Wochenenden ohne Zeitbegrenzung auch als Fahrausweis (Details auf \autopageref{verkehrsmittel}).

Auf dem RFID-Chip der Karte ist nur die ID für die Bezahlfunktion des Studierendenwerks gespeichert und daher auch die einzige Information, die sich ohne Sichtverbindung auslesen lässt. Sofern euer Geldbeutel dünn genug ist und die Campus Card nicht durch Kleingeld, Bankkarten o.ä. abgeschirmt wird, lässt sich sogar durch Auflegen des Geldbeutels bezahlen -- allerdings sind die Lesegeräte nicht besonders stark und auch nicht besonders schnell, sodass ihr das am besten dann ausprobiert, wenn nicht so viel Betrieb ist. Alle anderen Informationen wie euer Name, die Matrikelnummer usw. sind lediglich aufgedruckt. Wichtig ist vor allem, dass euer Studierendenausweis auch validiert ist, sonst ist er nämlich nicht gültig. Diese Validierung müsst ihr jedes Semester nach eurer Rückmeldung durchführen, vergesst das nicht!

\subsection{Account}
Damit ihr überhaupt Zugang zum URZ bekommt, benötigt ihr Nutzername und Passwort. Euer Nutzername (Uni-ID) ist auf eurem multifunktionalen Studierendenausweis aufgedruckt. Um dann den Account nutzen zu können, muss die Uni-ID freigeschaltet\footnote{\url{https://online-services.urz.uni-heidelberg.de/id/uni_hd_account_activate.php}} werden. Hier könnt ihr auch euer Passwort und eure Uni-E-Mail-Adresse auswählen. Die Freischaltung kann man auch direkt beim Infoservice im URZ erledigen.

\vspace*{-2mm}
\subsection{E-Mail}
\vspace*{-1mm}
Mit der Freischaltung erhaltet ihr automatisch eine E-Mail-Adresse die auf \emph{@stud.uni-heidelberg.de} endet und momentan mit 1000 MB Speicherplatz nicht so schlecht ausgestattet ist. Größter Vorteil ist ihre Werbefreiheit und sehr gute Verfügbarkeit. \textbf{Achtung:} Solltet ihr die Adresse trotz allem nicht verwenden wollen, ändert \emph{unbedingt} eure Stammdaten im LSF\footnote{\url{https://lsf.uni-heidelberg.de}} und lasst euch die E-Mail-Adresse weiterleiten. Viele wichtige E-Mails wie die Rückmeldeerinnerung, Nachrichten von euren Tutorinnen oder Rundmails der Vorlesungen gehen dort ein.

%\begin{figure*}[b]
%    \centering
%    \begin{subfigure}{.2\textwidth}
%	    \includegraphics[height=3cm]{bilder/new_car_1.png}
%    \end{subfigure}
%    \begin{subfigure}{.2\textwidth}
%        \includegraphics[height=3cm]{bilder/new_car_2.png}
%    \end{subfigure}
%    \begin{subfigure}{.2\textwidth}
%        \includegraphics[height=3cm]{bilder/new_car_3.png}
%    \end{subfigure}
%    \begin{subfigure}{.2\textwidth}
%        \includegraphics[height=3cm]{bilder/new_car_4.png}
%    \end{subfigure}
%\end{figure*}

\vspace*{-2mm}
\subsection{Drucken}
\vspace*{-1mm}
Zum Drucken kann man seine Dokumente auf einen speziellen Druck-Server\footnote{\url{https://www.urz.uni-heidelberg.de/de/service-katalog/drucken/oeffentliche-drucker-und-kopierer}} laden und dann an (fast) jedem der über den Campus verstreut aufgestellten Kopiergeräte ausdrucken. Das zum Drucken benötigte Gut\-ha\-ben wird über die Campus Card abgerechnet, die man in den Mensen und in der \gls{UB} aufladen kann. Für größere Aufträge gibt es einen Druckerraum im Keller des \gls{URZ}, der von der Firma Ricoh betrieben wird. Darüber hinaus bietet das URZ einen Poster-Druckservice\footnote{\url{https://www.urz.uni-heidelberg.de/de/poster}} und einen 3D-Druckservice\footnote{\url{https://www.urz.uni-heidelberg.de/de/3d-druck}} an.

\vspace*{-2mm}
\subsection{CIP-Pools -- Internet ohne Notebook}
\vspace*{-1mm}
Rechnerräume gibt es im \gls{KIP} und im \gls{PI} im 1. Stock, in der Bibliothek im \gls{Mathematikon} und natürlich im \gls{URZ}, in letzterem sogar mehrere. Für Veranstaltungen stehen im Mathematikon auch noch weitere CIP-Pools bereit, diese sind aber üblicherweise nicht frei zugänglich.

\subsection{WLAN -- Internet mit Notebook}

\begin{figure}[h]
    \centering
    \vspace{-5mm}

    \ifthenelse{\boolean{druckversion}}{
        \includegraphics[width=\linewidth]{eduroam_sw.pdf}
    }{
        \includegraphics[width=\linewidth]{eduroam.pdf}
    }
    \vspace{-5mm}

\end{figure}

% \begin{figure*}[t]
%     \centering
%     \includegraphics[width=.8\textwidth]{bilder/computers_vs_humans.png}
% \end{figure*}



Am einfachsten und komfortabelsten ist der WLAN-Zugang über das e\-du\-roam-Netz. Die Uni Heidelberg beteiligt sich an der eduroam-Initiative und bietet jeder Rechenzentrums-Nutzerin einen entsprechenden Zugang. Damit könnt ihr nicht nur an vielen Stellen auf dem Campus kabellos ins Internet, sondern auch an mehreren hundert Hochschulen und Forschungseinrichtungen in über 60 Ländern – einfach so.

Wie ihr euer Notebook konfigurieren müsst, um eduroam zu nutzen, ist auf den Webseiten des URZ\footnote{\url{https://www.urz.uni-heidelberg.de/de/eduroam}} für die unterschiedlichsten Betriebssysteme sehr detailliert erklärt.

Neben eduroam bietet das URZ an einigen Stellen noch das Netzwerk „UNI-HEIDELBERG“, das man mit einem VPN-Zugang nutzen kann, sowie das unverschlüsselte „UNI-WEBACCESS“, bei dem man sich in einem Captive Portal authentifiziert.

Sollte euch der WLAN-Zugang nicht mehr ausreichen, gibt es an einigen Orten auch kabelgebundene Zugänge mit mehr Bandbreite. Wenn ihr jetzt schon feuchte Hände bekommt und euch die nigelnagelneue externe Platte mit Inhalten aus zweifelhaften Quellen vollladen möchtet, seid auf die Benutzerordnung\footnote{\url{https://www.urz.uni-heidelberg.de/de/dokumente/verwaltungs-und-benutzungsordnung-vbo/download}} verwiesen. Nach zu viel Traffic dreht euch das \gls{URZ} den Hahn ab und schaltet ihn erst wieder frei, wenn ihr bestätigt, dass ihr keine bösen Dinge damit anstellt.

\subsection{VPN}
Um auf manche Dienste zugreifen zu können, müsst ihr euch im internen Uni-Netz befinden. Weil das von zu Hause irgendwie schwer ist, braucht ihr einen VPN-Tunnel. Damit stellt euer Rechner über einen Internetzugang eine Verbindung zur Uni her und es ist so, als würdet ihr mit einem LAN-Kabel in der Uni sitzen. Eine ausführliche Anleitung findet ihr online\footnote{\url{https://www.urz.uni-heidelberg.de/de/vpn}}, die neben allgemeinen Informationen auch Anleitungen zur Installation des \emph{Cisco AnyConnect VPN Client}, enthält. Hier aber dennoch die Kurzfassung:

Unter Linux könnt ihr einfach \emph{openconnect} verwenden. Hier könnt ihr z.B. unter Ubuntu, Debian oder Mint das Paket \emph{network-manager-openconnect-gnome} nutzen und dort \emph{vpnsrv0.urz.uni-heidelberg.de} als Gateway und \emph{Deutsche\_Telekom\_Root\_CA\_2.pem} als CA Certificate einstellen, hier eine Anleitung\footnote{\url{https://youtu.be/_vOrUASbegs}}.

Unter Windows \ldots benutzt einfach kein Windows. Falls doch, nutzt den Cisco AnyConnect Client und folgt dieser Anleitung\footnote{\url{https://youtu.be/L6mlUyoEJ9I}}. Es gibt auch Alternativen, diese findet ihr auch auf den Seiten des Rechenzentrums.

\subsection{Lehre -- Studium -- Forschung (LSF)}
Das LSF\footnote{\url{https://lsf.uni-heidelberg.de/}} ist das Campus-Management-System der Uni. Hier findet ihr neben dem Vorlesungsverzeichnis und Informationen zu Gebäuden und Personen auch eure Bescheinigungen, beispielsweise für BAföG o.ä., euer Stammdatenblatt könnt ihr da auch ausdrucken. Das braucht ihr, um später nachweisen zu können, dass ihr auch wirklich studiert habt.

Für einige Funktionen braucht ihr eine TAN. Diese müsst ihr euch einmal als Liste ausdrucken. Das ist ein bisschen umständlich, sollte aber mit der Anleitung, die ihr bei eurer Einschreibung bekommen habt, kein Problem sein.

\subsection{Softwarelizenzen}

Über das URZ kann man als Studentin auch Lizenzen für verschiedene Programme und Betriebssysteme bekommen. Zum Beispiel Microsoft Office 365 Pro Plus. Die Uni ist Teil eines Abkommens von dem Land Baden-Württemberg mit Microsoft, wodurch alle Studentinnen sich das Office 365 Pro Plus Paket für \EUR{3.99} im Jahr holen können. Auch erwähnenswert ist Matlab, was für numerische Anwendungen ein praktisches Programm ist. Die gesamte aktuelle Liste findet ihr auf der URZ-Webseite\footnote{\url{https://www.urz.uni-heidelberg.de/de/service-katalog/software-und-anwendungen}}.

Außerdem gibt es diverse andere externe Dienstleisterinnen, die extra für Studentinnen besondere Konditionen anbieten. Zum Beispiel könnt ihr bei GitHub als Studentin unbegrenzt viele private Repositories kostenlos anlegen. Häufig ist für so etwas die \emph{@stud.uni-heidelberg.de}-Adresse notwendig.
