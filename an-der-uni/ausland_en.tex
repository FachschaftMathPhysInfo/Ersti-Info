% !TEX ROOT = ../ersti.tex
\section{Semester Abroad/Studium im Ausland}
It is mentioned time and time again how many great experiences and possibilities a semester abroad entails. People tell reverently about the great time they had at University XYZ. But how do you get there?

%In den Bachelor-Studienordnungen der Physik und der Mathe heißt es, „Bei der Anerkennung von Studienzeiten, Studien- und Prüfungsleistungen, die außerhalb Deutschlands erbracht wurden, sind die von Kultusministerkonferenz und Hochschulrektorenkonferenz gebilligten Äquivalenzvereinbarungen sowie Absprachen im Rahmen von Hochschulpartnerschaften zu beachten. Bei Zweifeln an der Gleichwertigkeit kann die Zentralstelle für ausländisches Bildungswesen gehört werden.“ Das ist nicht sonderlich hilfreich, wenn man sich nicht unbedingt durch Tonnen von Gesetzestext quälen will. %
On the faculty homepages you can find a list of universities with which there are ongoing exchange programs (meaning that classes done there will get accepted here), as well as some general information and a couple of reports by former exchange students, which you should look at, if you are interested in the topic. The equivalency of achievements does not have to be checked by you, the faculty has the duty to prove that. So don't let yourself get confused and if there's any doubt, just take the course you want while abroad, maybe they won't (be able to) prove you wrong.

%Allerdings sollte ein Auslandssemester im Moment nicht euer erstes Problem sein -- während der ersten beiden Semester ist es ganz einfach fachlich nicht möglich. Bis zur Einführung der Bachelor- und Masterstudiengänge konnte man (wenigstens in der Mathe) an diesen Programmen nur nach abgeschlossenem Grundstudium teilnehmen, inzwischen wird der Zeitraum letztes Jahr Bachelor/erstes Jahr Master empfohlen. Das bedeutet, dass ihr nach dem zweiten Semester damit beginnen solltet, euch umzuschauen, insbesondere in Bezug auf Bewerbungsvoraussetzungen und -fristen.%

AuslandsBAföG has to be applied for seperately, who's responsible for it depends on the country you plan to go to, a list of all the responsible offices can be found, for example, on \url{AuslandsBAfoeG.de}. In Heidelberg you can also go directly to the department for international relations \footnote{\url{http://www.uni-heidelberg.de/studium/kontakt/auslandsamt/}} which is located in the Seminarstraße 2. You have to go to room 139.

\paragraph{Opening hours} Mo. -- Thu. 10 -- 15 \qquad Fr. 10 -- 13

Another very important thing to know is how to request a semester of leave ("Urlaubssemester"), so that your "Fachsemester" count (important for everyone receiving BAföG) doesn't increase. Instead the "Hochschulsemester" count will continue to rise. You can then (after checking in with the examination office) get the points you got awarded at the partner university \gls{LP} recognized by the University of Heidelberg.

The student administration in Seminarstraße 2 is responsible for semesters on leave. You can find additional information and the form on:
\url{https://www.uni-heidelberg.de/studium/imstudium/formalia/beurlaubung.html}

