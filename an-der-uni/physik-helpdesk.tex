\section{Physik-Helpdesk}
\label{sec:physik-helpdesk}

Seit einigen Jahren gibt es in der Physik einen Helpdesk (bis vor einem Jahr hieß dieser „Studentischer Arbeitsraum“). Dort findest du eine Tutorin aus einem höheren Semester, die dir bei Fragen zu Übungszetteln, zu Vorlesungsinhalten und zum Physikstudium ganz allgemein weiterhelfen kann. Die Tutorin ist kein Quell für Musterlösungen, sondern kann dir durch ihre eigene Studienerfahrung bei Problemen und Unklarheiten aller Art weiterhelfen und Tipps geben. Insgesamt lässt sich sagen, dass der Physik-Helpdesk in dieser Hinsicht ein niederschwelliges Zusatzangebot zu den Übungsgruppen ist. Außerdem findest du beim Helpdesk auch eine Auswahl an Lehrbüchern als kleine Präsenzbibliothek, und kannst dir sehr preiswert einen Kaffee oder einen Tee machen.

Zu finden ist der Physik-Helpdesk im \gls{KIP} im zweiten Stock vor den Seminarräumen. Von der zweiten bis zur vorletzten Vorlesungswoche sitzt dort montags bis freitags von 13 bis 19~Uhr eine Tutorin, die du auch durch ein großes Plakat an der Wand erkennen kannst. Die aktuellen Infos zum Helpdesk sollten auch immer hier\footnote{\url{https://uebungen.physik.uni-heidelberg.de/arbeitsraum}} zu finden sein.

Falls du also mal an einem Übungszettel verzweifeln solltest, komm vorbei und lass dir helfen.
