% !TEX ROOT = ../ersti.tex
\section{heiSKILLS}

heiSKILLS unterstützt euch dabei, eure Kompetenzen im Studium und darüber hinaus zukunftsfähig zu gestalten und zu erweitern. Das heiSKILLS Kompetenz- und Sprachenzentrum vereint hierfür die Beratungs- und Veranstaltungsangebote des Zentralen Sprachlabors, des Career Service sowie Study Skills für das Lernen, um euch so für euer Studium zu stärken und euch auf eure berufliche Zukunft vorzubereiten.

\begin{itemize}
  \item Möchtet ihr wissenschaftliches Schreiben üben oder eure Kompetenzen im kritischen Denken erweitern?
  \item Leitet ihr ein Tutorium und wollt euch fit machen, wie ihr Kommilitoninnen beim Lernen unterstützen könnt?
  \item Möchtet ihr Fremdsprachenkenntnisse ausbauen oder neu erwerben und benötigt international anerkannte Sprachnachweise für euer Studium oder einen Auslandsaufenthalt?
  \item Benötigt ihr eine starke Stimme für euren zukünftigen Beruf, etwa als Lehrerin, und möchtet zudem eure rhetorischen Fähigkeiten fördern?
  \item Wollt ihr euch im Rahmen von Karriereveranstaltungen vernetzen, informieren und mit unseren Expertinnen über mögliche Berufsperspektiven und -optionen sprechen oder euch in einer individuellen Karriereberatung hinsichtlich beruflicher Perspektiven coachen lassen?
  \item Seid ihr interessiert daran euch während des Studiums berufsrelevante Zusatzqualifikationen wie z.\,B.\, betriebswirtschaftliche Grundlagen, Projektmanagement oder Medienproduktion zur Schärfung Eures Kompetenzprofils anzueignen?

\end{itemize}

Möglichkeiten hierzu, aber auch zu vielem mehr findet Ihr bei heiSKILLS.
Unsere Angebote findet Ihr im LSF\footnote{\url{https://lsf.uni-heidelberg.de/qisserver/rds?state=wtree&search=1&trex=step&root120221=140004|140483&P.vx=mittel}} und auf den Abteilungswebseiten\footnote{\url{https://www.uni-heidelberg.de/careerservice/veranstaltungen}, \url{https://www.uni-heidelberg.de/slk/nutzbar/}}.
Falls Ihr Fragen habt oder Vorschläge und Ideen zu weiteren Angeboten habt, die ihr gerne sehen würdet, gebt einfach Bescheid unter \email{heiSKILLS@uni-heidelberg.de}. Wir freuen uns, von euch zu hören und euch in unseren Kursen und Veranstaltungen zu sehen!
