% !TEX ROOT = ../ersti.tex
% \vfill\eject

\section{Werk- und Experimentierraum}
\label{werk-und-experimentierraum}

Um deinen Erfindergeist und dein handwerkliches Geschick zu fördern, wurde der studentische Werk- und Experimentierraum gegründet. Dieser Raum bietet dir die Möglichkeit, unter Aufsicht und mit der Unterstützung erfahrener Tutorinnen eigene Projekte selbstständig durchzuführen und dabei die universitären Strukturen frühzeitig kennenzulernen und zu nutzen. Zudem werden regelmäßig Workshops zu spannenden Projekten wie dem Bau von Fahrzeugen organisiert.

Der Raum ist mit einer Vielzahl an Werkzeugen und Geräten ausgestattet: von Elektronik- und mechanischen Werkzeugen über 3D-Drucker, Arduinos, Raspberry Pi und Asuros inklusive kompatibler Sensoren und Module bis hin zu Holz- und Metallverarbeitungswerkzeugen und einer Nähmaschine.

Um den Werkraum nutzen zu können, ist eine Sicherheitseinweisung erforderlich, die jedes Semester angeboten wird. Informationen zu Öffnungszeiten, Anmeldung, Ansprechpartnerinnen und dem Standort findest du auf der Webseite\footnote{\url{https://www.kip.uni-heidelberg.de/mitarbeit/werkraum%7D%7D.

Der Raum steht dir offen, auch wenn du keine Physikstudentin bist. Wenn du Lust hast, an eigenen Projekten zu arbeiten, etwas Cooles zu bauen oder einfach etwas Neues zu lernen, ist der Werkraum genau der richtige Ort für dich!
