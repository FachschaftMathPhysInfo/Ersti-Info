% \documentclass[12pt]{article}
% \usepackage{ngerman}
% \usepackage[utf8]{inputenc}
% \usepackage[arrow, matrix, curve]{xy}
% \usepackage{ragged2e}
% \usepackage{array}
% \usepackage{amssymb}
% \usepackage{enumitem}
% \usepackage{amsmath}
% \usepackage{amsfonts}
% \usepackage{amssymb}
% \usepackage{makeidx}
% \usepackage{mathtools}
% \usepackage{listings}

% \usepackage[a4paper,left=2cm,right=2cm,top=2cm,bottom=2.5cm]{geometry}
% \setlength{\parindent}{0pt}

% \usepackage{tabularx}
\newcolumntype{L}[1]{>{\raggedright\arraybackslash}p{#1}} % linksbündig mit Breitenangabe
\newcolumntype{C}[1]{>{\centering\arraybackslash}p{#1}} % zentriert mit Breitenangabe
\newcolumntype{R}[1]{>{\raggedleft\arraybackslash}p{#1}} % rechtsbündig mit Breitenangabe

% \begin{document}

\section{Digitale Tools im Corona-Alltag}

Damit ihr auf einen Blick sehen könnt, welches Tool wofür verwendet wird, findet ihr hier eine kurze Zusammenfassung. Tools, die als Funktion Onlinevorlesungen aufweisen, wurden im vergangenen Semester verwendet, es ist also möglich, dass in diesem Semester auch andere Tools verwendet werden. Welche Tools eure Professorinnen und Tutorinnen letztendlich verwenden werden, werden sie euch dann rechtzeitig mitteilen. 

\def\nl{\vspace{12pt}\\}
\begin{table}[h]
	\begin{tabular}{L{2cm} L{4cm} L{9cm}}
		\toprule
		\textbf{Tool} & \textbf{Was ist es?} &  \textbf{Wofür wird es verwendet?} \\ 
		\midrule
		\href{https://lsf.uni-heidelberg.de/}{Lehre-Studium-Forschung (LSF)} &  Campus-Management-System der~Uni & Alles, was mit Verwaltung zu tun hat: Vorlesungsverzeichnis, Informationen zu Personen, Gebäuden, Rückmeldung durchführen, Bescheinigungen ausdrucken. In anderen Studiengängen auch Anmeldungen für Veranstaltungen und Klausuren\nl
		
		\href{https://moodle.uni-heidelberg.de}{Moodle} & e-Learningplattform & Lernmaterialien, Zettelabgabe, Punkteeinsicht, Klausuranmeldung, organisatorische Informationen \nl

		\href{https://uebungen.physik.uni-heidelberg.de}{Übungs-gruppen-verwaltung} & Tool zum Organisieren von Übungsgruppen der Physik & Eintragen in Übungsgruppen, Anmeldung zu Veranstaltungen, Punkte für Übungszettel, Einsicht der Klausurnote \nl

		\href{https://muesli.mathi.uni-heidelberg.de}{MÜSLI} & Tool zum Organisieren von Übungsgruppen in der Mathe und Info & Eintragen in Übungsgruppen, Anmeldung zu Veranstaltungen, Punkte für Übungszettel, Klausuranmeldung, Einsicht der Klausurnote \nl

		\href{https://mampf.mathi.uni-heidelberg.de}{MaMpf} & Mathematische Medienplattform & Vorlesungsvideos, Beispielvideos, Quizzes, weitere Lernangebote \nl

		\href{https://heiconf.uni-heidelberg.de}{heiCONF} & Uni-internes Videokonferenztool & Onlinevorlesungen, Seminare, Sprechstunden, Tutorien \nl

		Zoom & Videokonferenztool & Onlinevorlesungen, Seminare, Sprechstunden, Tutorien \nl

		Cisco Webex & Videokonferenztool & Onlinevorlesungen, Seminare, Sprechstunden, Tutorien \nl 
		
		Microsoft Teams & Videokonferenztool & Onlinevorlesungen, Seminare, Sprechstunden, Tutorien \nl 
	
		Twitch & Live-streaming-Videoportal & Onlinevorlesungen (in Kombination mit Chat-Tool) \\
					
		\bottomrule
	\end{tabular}
	
\end{table}