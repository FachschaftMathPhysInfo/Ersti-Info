\section{Ein paar Worte zu Corona \ldots}

Für euch kommt in diesem Semester gleich zu Beginn eures Studiums noch eine Schwierigkeit hinzu, die ihr alle schon kennt und die ihr vielleicht auch fürchtet: Corona.
Auch wenn für euch erst einmal alles neu ist -- wie bei all den Erstis vor euch auch -- und dann doch ein paar Dinge  anders laufen, ist es für euch ebenso wichtig, am Anfang möglichst viele Kommilitoninnen kennenzulernen. Denn auch für euch gilt: Nur gemeinsam sind wir stark! Es kommen viele Herausforderungen auf euch zu, die gemeinsam leichter zu bewältigen sind und zudem sogar Spaß machen können.

Zu Beginn sei hier das berühmt-berüchtigte Zettelrechnen erwähnt, das ihr demnächst kennenlernen werdet. Jede Woche werdet ihr pro Veranstaltung ein Übungsblatt bearbeiten und abgeben. Alleine verzweifelt man schnell und braucht sehr viel Zeit, gemeinsam kommt man schneller ans Ziel und wenn es auch manchmal vielleicht sehr knifflig ist, haben mehr Köpfe meistens mehr gute Ideen. Sich gegenseitig zu unterstützen und gemeinsam zu lernen, gehört zum Studium einfach dazu. Es ist also wichtig, gleich zu Anfang Kontakte zu knüpfen.

Nutzt unbedingt das Rahmenprogramm des Vorkurses, um Kommilitoninnen zu treffen.  Im Vorkurs organisieren wir für euch, falls möglich, Wanderungen und Spieleabende in Präsenz.
Eure Tutorien werden, sofern irgendwie möglich, in Kleingruppen stattfinden.
Und habt ihr erst einmal nette Menschen getroffen, können wir euch nur empfehlen, mit den anderen etwas zu machen -- auch wenn im Winter das Outdoorprogramm sehr eingeschränkt ist. Trefft euch trotzdem, zum Kochen, (kleine) Spieleabende oder online (z.\,B.\,Jackbox, Montagsmaler, Codenames, \ldots), zum Telefonieren oder einfach nur vor der Kamera Abendessen, damit ihr euch nicht so alleine fühlt. Denn das solltet ihr im Studium nicht und müsst es auch nicht.

Insbesondere mit euren Zettelpartnerinnen werdet ihr vermutlich viel Zeit verbringen, falls möglich in Präsenz, ansonsten online. So lassen sich erste soziale Kontakte knüpfen und vielleicht sogar kleine Lerngruppen bilden. Nutzt diese Gruppen neben dem produktiven, inhaltlichen Studieren unbedingt auch für soziale Interaktionen, indem ihr zum Beispiel einen gemeinsamen Wein und Käse-Abend veranstaltet oder zusammen Fernsehserien schaut.

Wir drücken euch für euren Studienbeginn trotz der schwierigen Randbedingungen die Daumen, hoffen, ihr könnt euer Studium trotzdem genießen, und wünschen einen möglichst guten Semesterstart!
