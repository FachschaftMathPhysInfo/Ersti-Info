\section{Bachelor Allgemein}
In den ersten Semestern besuchen die meisten Studis die sog. \emph{Pflichtvorlesungen}. Gerade die ersten beiden Semester sind i.d.R. mit diesen Grundvorlesungen gut gefüllt. Ab dem dritten Semester belegen dann viele die ersten \emph{Wahlpflichtveranstaltungen}, bei denen du für bestimmte Module (s.u.) aus mehreren Lehrveranstaltungen die wählst, die dir am ehesten zusagen. Im Bachelor gibt es vier Bereiche: \emph{Pflicht-}, \emph{Wahlpflicht-} und der Bereich \emph{Fachübergreifende Kompetenzen (FüK)}, sowie den \emph{Wahlbereich} (Physik) bzw.~ein \emph{Anwendungsgebiet} (Mathe/Info).

\begin{itemize}
	\item Pflichtbereich: Grundlagen des Studienfachs und fachliche Methodik
	\item Wahlpflichtbereich: Spezialisierung auf einem Gebiet
	\item FüK: Soft Skills und fachübergreifendes Wissen
	\item Anwendungsgebiet: Interdisziplinäre Anwendung eures Fachwissens
\end{itemize}

\paragraph*{Module} sind \glqq{}thematisch und zeitlich abgeschlossene Lehr- und Lerneinheiten\grqq{}, in die euer Studium gegliedert ist. Verständlich: Das sind

\begin{itemize}
	\item Vorlesungen, meist mit einer Klausur am Ende,
	\item Seminare, in denen eure Prüfung aus einem Vortrag besteht,
	\item Praktika, in denen ihr lernt, euer Wissen anzuwenden.
\end{itemize}

Teilweise bestehen Module aus mehreren Veranstaltungen. Dann müsst ihr zum Abschließen des Moduls eben nicht nur eine Prüfung bestehen oder einen Vortrag halten, sondern eben alle\footnote{rechtlich heißt es zwar „ein Modul, eine Prüfung“, aber man kann da Ausnahmen machen} Veranstaltungen des Moduls bestehen.

Zur Vertiefung und Übung der Themen gibt es in fast allen Vorlesungen jede Woche einen Übungszettel mit Aufgaben zum aktuellen Thema. Dabei wiederholt und übt man den Vorlesungsstoff und erarbeitet sich die notwendigen Punkte für die Klausurzulassung (i.d.R. reichen 50\% aller Punkte). Das jeweilige System wird euch immer am Anfang des Semesters in jeder Vorlesung erklärt.

Für bestandene Module bekommt ihr dann \gls{LP}\footnote{manchmal auch \gls{CP} genannt}, von denen ihr in eurem Studium insgesamt 180 sammeln müsst (mit einigen Bedingungen verknüpft), um euren Bachelor zu bekommen. Die Anzahl der Punkte pro Modul errechnet sich aus einem von den Dozierenden jeweils festgelegten Schlüssel, der die benötigte Zeit für den Besuch der Vorlesung, die Übungsaufgaben und das Selbstlernen mit Vor- und Nachbereitung der Vorlesungen widerspiegeln soll. Ein \gls{LP} entspricht etwa 30 Stunden Arbeit im Semester, also gute zwei Stunden pro Woche. Da es natürlich stark von euch abhängt, wie lange ihr für das alles braucht und wie viel Zeit ihr wirklich investieren wollt, kann das nur eine grobe Abschätzung sein, ist aber eine gute Orientierung, wenn ihr überlegt, wie viel ihr euch im Semester aufbürden wollt.

Genaue Informationen zu den Modulen, Modellstudienpläne und die Prüfungsmodalitäten findet ihr auf den Seiten der Uni\footnote{\url{http://www.uni-heidelberg.de/studium/download/stud_pruef.html}} in eurem jeweiligen Modulhandbuch bzw. eurer Prüfungsordnung. Bei Unklarheiten geben euch eure Prüfungssekretariate rechtsverbindliche Auskünfte und beantworten Fragen zur Prüfungsordnung. 
\begin{itemize}
	\item Physik: \pruefsekphysik
	\item Mathe: \pruefsekmathe
	\item Info: \pruefsekinfo
\end{itemize}

% === Den folgenden Abschnitt würde ich komplett weglassen.
% === Diese Info bekommen die Leute im Prüfungssekretariat
% === und sie ist im ersten Semester i.d.R. nicht relevant.

% Falls die euch auch nicht mehr helfen können, ihr eine offizielle Unterschrift braucht oder Einzelabsprachen treffen wollt, ist die letzte Instanz, die alles entscheidet, der Prüfungsausschuss. In der Physik ist das \pruefausschussvorsitzphysik, in der Mathe \pruefausschussvorsitzmathe, in der Informatik \pruefausschussvorsitzinformatik.
