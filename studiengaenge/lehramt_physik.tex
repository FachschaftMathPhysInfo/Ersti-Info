%\newpage\Large\mathphyssubsubsec{Lehramt Physik}\small
\section{Physik 50\%}

Der 50\%-Bachelorstudiengang Physik ersetzt seit dem Wintersemester 2015/16 den Physik-Lehramtsstudiengang. Hier hast du die Wahl zwischen Physik 50\% mit und ohne Lehramtsoption. 
Die Lehramtsoption soll dir die notwendigen fachlichen Kompetenzen für den Physikunterricht beibringen und den Übergang zum Master of Education ermöglichen.
Wenn du den 50\%-Bachelor nutzt, weil du an zwei Fächern interessiert bist, aber nicht das Berufsziel Lehrerin anstrebst, solltest du ohne Lehramtsoption studieren. Sofern du die Bachelorarbeit im Fach Physik schreibst, ist anschließend der Übergang in den Master of Science Physik möglich.

Um dir einen Überblick über den Studienverlauf zu verschaffen, findest du im Modulhandbuch Modellstudienpläne. Dort kannst du auch die Varianten für 100\% Physik, 50\% Physik und 50\% Physik mit Lehramtsoption vergleichen.

\begin{figure*}[b]
    \centering
    \includegraphics[width=.95\textwidth]{bilder/teaching_physics.jpg}
\end{figure*}

\subsection{Mit Lehramtsoption}

Gemäß Prüfungsordnung darfst du den 50\%-Bachelor Physik mit Lehramtsoption mit allen anderen 50\%-Studiengängen an der Universität Heidelberg, die für einen Master of Education qualifizieren, kombinieren. Allerdings ist die Kombination mit Mathematik vorteilhaft, da du nur so mathematische Grundlagen erwirbst (siehe dazu den Abschnitt \nameref{mathegrundlagen}) und mit der Mathematik die zeitliche Überschneidungsfreiheit von Pflichtveranstaltungen noch am ehesten gewährleistet ist.

Unabhängig von der Fächerkombination darfst du am Ende dieses Studiums die Bachelorarbeit in Physik schreiben; du musst aber nicht. Du kannst dein erstes Lehramtsfach, in dem du die Bachelorarbeit schreiben wirst, auch unkompliziert im Laufe deines Studiums tauschen.

Während der 50\%-Bachelor ohne Lehramtsoption auf einen möglichen Master of Science in Physik vorbereiten soll, folgt auf den 50\%-Bachelor mit Lehramtsoption der Master of Education, welcher fachlich weniger in die Tiefe geht, dafür aber Inhalte der Bildungswissenschaften und Fachdidaktik vermittelt. Um diesen unterschiedlichen Ansprüchen der beiden 50\%-Bachelorstudiengänge gerecht zu werden, überarbeitet die Fakultät aktuell den Lehramtsstudienplan. Ab dem Wintersemester 2023/24 gibt es die neue dreiteilige Vorlesungsreihe \vl{Moderne Physik für Lehramt} (\gls{MoPhy}). Diese besteht aus drei aufeinanderfolgenden Vorlesungen, wobei die ersten beiden den Stoff aus den Vorlesungen \vl{Experimentalphysik 3}, \vl{4} und \vl{5} (\gls{Ex}), sowie \vl{Theoretische Physik 3} und \vl{4} (\gls{Theo}\footnote{Die Abkürzung „Theo“ kann zu Verwirrungen mit Informatikern führen, da diese das als \vl{Theoretische Informatik} verstehen.}) behandeln. Die \gls{MoPhy} 3 geht darüber hinaus und bietet eine Einführung in die Umwelt- und Astrophysik.

In den ersten beiden Semestern hört man regulär die Ex~1 und~2. Bei der Ex~1 handelt es sich um die Orientierungsprüfung für den Studiengang Physik, die du am besten im ersten, spätestens aber im dritten Semester bestehen musst. Im dritten und vierten Semester geht es dann nicht mit der Ex~3 und 4 weiter, sondern du startest die MoPhy~1 und 2. Parallel dazu hörst du die Theo~1 und~2. Im sechsten Semester kannst du dich dann entscheiden: aktuell hast du die Wahl zwischen Ex~4 und Theo~4, je nachdem welche Vorlesung dich mehr interessiert. Ergänzt wird diese Vorlesung durch einen Programmierkurs. Lies am besten nochmal die aktuellen Regularien durch, wenn du soweit bist.

Eine weitere Neuerung betrifft dich bereits im ersten Semester: du kannst dir sowohl den \vl{Vorkurs} als auch den \vl{Basiskurs} für dein Lehramtsstudium anrechnen lassen. 

Ein Großteil der Arbeit ist auf die Vorlesungszeit konzentriert; in der vorlesungsfreien Zeit können Labor- und Schulpraktika durchgeführt werden. Während das \vl{Physikalische Anfängerpraktikum für Lehramt} (APL) für die Semesterferien nach dem zweiten Semester vorgesehen ist, kannst du das erste Schulpraktikum, die sogenannte \vl{Berufsorientierende Praxisphase 1} (\gls{BOP}) nach Belieben zwischen dem ersten und fünften Semester absolvieren. Wenn du das Praktikum direkt nach dem ersten Semester durchführen möchtest, musst du dich jedoch rechtzeitig zu Beginn des ersten Semesters anmelden.

Das 50\%-Studium mit Lehramtsoption ist noch weit entfernt davon, ausgereift zu sein, und viele Feinheiten hängen auch sehr vom Kombinationsfach ab, sodass man es als Lehramtsstudentin leider nicht immer leicht hat. Wir als Fachschaft stehen euch dabei neben anderen Anlaufstellen soweit wie möglich als Ansprechpartnerin zur Verfügung (siehe \autopageref{lehramtkontakte}). Im Laufe des Vorkurses wird es auch ein spezielles 50\%lerinnen-Treffen zusammen mit einer 50\%-Studentin aus einem höheren Semester geben, sodass ihr bereits dort wichtige Tipps und Erfahrungen mit auf den Weg bekommt.

\subsection{Ohne Lehramtsoption}

Der polyvalente Bachelor mit 50\% Physikanteil soll für den Master of Science Physik qualifizieren. Das heißt, dass du die grundlegenden Vorlesungen \vl{Experimentalphysik 1-5} (\gls{Ex}) und \vl{Theoretische Physik 1-4} (\gls{Theo}) verpflichtend hören musst. Ein Übergang in den Master of Science Physik ist jedoch nur möglich, wenn du die Bachelorarbeit im Fach Physik schreibst. Davon abgesehen macht es für den Studienverlauf keinen Unterschied, ob Physik dein erstes oder zweites Fach ist.

Der Modellstudienplan sieht vor, den Experimentalphysikzyklus im ersten Semester zu beginnen und den Theoriezyklus ein Jahr später. Die \gls{Ex}~1 sollte man auf keinen Fall nach hinten schieben, da diese die Orientierungsprüfung ist, die bis zum dritten Semester bestanden werden muss.
Konkret bedeutet das, dass man im ersten Semester die \gls{Ex}~1 belegt, dazu kommen Veranstaltungen aus dem anderen Fach.
Für den Fall, dass du als anderes Fach Mathematik gewählt hast, hörst du im ersten Semester noch \vl{Analysis 1} (\gls{Ana}) und \vl{Lineare Algebra 1} (\gls{LA}).

Der Modellstudienplan ist nicht verpflichtend. Wer interessiert ist, kann auch bereits im ersten Semester die \gls{Theo}~1 besuchen. Es ist auch kein Problem, die Vorlesung nach den ersten Wochen wieder fallen zu lassen, wenn man merkt, dass der Arbeitsaufwand zu groß ist und man seine Energie lieber auf die verbleibenden Vorlesungen (inklusive der Orientierungsprüfung!) fokussieren möchte.

\subsection{Mathematische Grundlagen}
\label{mathegrundlagen}

„Mathematik ist die Sprache der Physik“ heißt es so schön und das ist in der Tat korrekt. Man könnte sogar weitergehen und sagen, dass die Menschheit nur deshalb begonnen hat, Mathematik zu betreiben, weil sich damit die Natur beschreiben lässt. Dies heißt aber auch, dass alle, die Physik betreiben -- sei es an Schulen, Universitäten oder in der Wirtschaft -- ein Grundverständnis für Mathematik benötigen. Für diejenigen unter euch, die Mathematik als zweites Fach gewählt haben, ist das Folgende nicht relevant, da die dort vorgesehenen Mathematik-Vorlesungen sicher mehr als nur ein Grundverständnis für Mathematik beibringen. Für diejenigen unter euch, die mit Lehramtsoption studieren, vermittelt die MoPhy (hoffentlich) die nötigen mathematischen Kompetenzen.

Vieles der Mathematik, die in den ersten Semestern gebraucht wird, wird in den Physik-Vorlesungen behandelt. Grund dafür ist, dass die Fakultät für Mathematik und Informatik die Inhalte ihrer Vorlesungen natürlich an ihren eigenen Zielen und nicht an denen des Physik-Studienganges ausrichtet. So beinhaltet beispielsweise die Vorlesung \gls{Theo}~1 einen großen Mathematik-Teil, in dem unter anderem das Lösen von Differentialgleichungen behandelt wird.

Allerdings sind nicht ohne Grund für den 100\%-Bachelorstudiengang Physik die Vorlesungen \gls{LA}~1 sowie wahlweise \vl{Höhere Mathematik für Physiker 2} und \vl{3} (\gls{HoMa}) oder \gls{Ana}~2 und 3 vorgeschrieben. Zum Beispiel ist die theoretische Beschreibung der Quantenmechanik ohne Kenntnis über Eigenvektoren aus \gls{LA}~1 nicht zu verstehen. Das Beste wäre also, wenn all jene, die nicht Mathematik als zweites Fach gewählt haben, zumindest \gls{LA}~1 vor \gls{Theo}~4 hören (oder ein Buch lesen) würden. Diese Vorlesungen sind im Modellstudienplan allerdings nicht vorgesehen, wodurch dies quasi freiwillige Zusatzarbeit wäre, da das 50\%-Studium keinerlei Spielraum für zusätzliche Leistungspunkte aus Wahlfächern vorsieht.
