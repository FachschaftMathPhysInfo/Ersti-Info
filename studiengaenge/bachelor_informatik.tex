% !TEX ROOT = ../ersti.tex
\section{Informatik 100\%}

\begin{figure*}[th]
    \begin{subfigure}{.23\textwidth}
	    \includegraphics[height=5cm]{bilder/cant_sleep_1.png}
    \end{subfigure}
    \hfill
    \begin{subfigure}{.23\textwidth}
	    \includegraphics[height=5cm]{bilder/cant_sleep_2.png}
    \end{subfigure}
    \hfill
    \begin{subfigure}{.23\textwidth}
	    \includegraphics[height=5cm]{bilder/cant_sleep_3.png}
    \end{subfigure}
    \hfill
    \begin{subfigure}{.23\textwidth}
	    \includegraphics[height=5cm]{bilder/cant_sleep_4.png}
    \end{subfigure}

\end{figure*}

\subsection{Die ersten Semester}

Im ersten Semester hört ihr nach Studienverlaufsplan (s. Modulhandbuch) die „Einführung in die Praktische Informatik“ (\gls{IPI}), „Einführung in die Technische Informatik“ (\gls{ITI}), den „Programmierkurs“ (\gls{IPK}) und eine Mathematikvorlesung. Die IPI vermittelt Grundkenntnisse und -konzepte der Informatik anhand mindestens einer Programmiersprache. Im IPK vertieft ihr eure Programmierkenntnisse mit einer weiteren und in ITI lernt ihr Grundlegendes über Rechnerarchitektur und die dazugehörigen logischen Schaltungen. Aufgepasst! Die IPI ist in der Informatik eure einzige \emph{Orientierungsprüfung} und ihr müsst sie \emph{bis zum Ende des dritten Semesters bestanden} haben.


\subsection{Später dann\dots{}}

\dots{} werden einzelne Themengebiete eröffnet und vertieft. Die Namen der Vorlesungen sprechen zum größten Teil für sich, zum Beispiel „Datenbanken“ (\gls{IDB1}) oder „Betriebssysteme und Netzwerke“ (\gls{BeNe} bzw. IBN). Außerdem besucht ihr ein Proseminar, ein Seminar sowie ein An\-fän\-ger-- und ein Fortgeschrittenen--Praktikum (\gls{AP} und \gls{FP}). Diese unterscheiden sich von Vorlesungen, da von der Modulbeschreibung kein inhaltliches Thema vorgegeben wird, sondern lediglich die Veranstaltungsform.

\paragraph*{In Seminar und Proseminar} erklärt ihr jeweils den anderen Studis in einem Vortrag ein euch zuvor zugeteiltes Thema aus einem zusammenhängenden Themenbereich. Bei einem Proseminar wird dabei eher auf die Präsentation an sich geachtet, wobei es im Seminar normalerweise stärker auf den Inhalt ankommt. Das ist aber von der Lehrkraft abhängig und wird euch jeweils am Anfang erklärt.

\paragraph*{Praktika} sind i.d.R. Projekte mit einigermaßen freier Zeiteinteilung, die am Ende des Semesters fertig sein müssen und in Gruppen bearbeitet werden. Die Praktika wechseln von Zeit zu Zeit, es gibt aber üblicherweise jedes Jahr Angebote in den Bereichen Technische Informatik und Software Engineering.

\subsection{Mathe und so\dots}

Informatikstudium in Heidelberg heißt, ihr hört vergleichsweise viel Mathe. Dabei habt die Wahl zwischen mehreren Varianten, in der Hauptsache gibt es aber diese beiden Wege, bei denen ihr die für euer weiteres Studium relevanten Grundkenntnisse vermittelt bekommen sollt:

\begin{itemize}
	\item Mathe bei den Mathematikern, also „Lineare Algebra 1“ (\gls{LA}) und „Analysis 1“ (\gls{Ana}). LA und Ana können entweder beide im ersten oder LA im ersten und Ana im dritten Semester gehört werden.
	\item „Mathematik für Informatiker 1“ (\gls{MafIn} bzw. IMI, entspricht etwa LA 1) im ersten und „Mathematik für Informatiker 2“ (ersetzt Ana 1) im zweiten Semester. Weiteres dazu im Abschnitt \autoref{mafin}.
\end{itemize}

Danach hört ihr „Einführung in die Numerik“ (\gls{Num0}) und eins der drei Module „Analysis 2“, „Mathematische Logik“ oder „Einführung in die Wahrscheinlichkeitstheorie und Statistik“ (\gls{WTheo0}), sowie \emph{freiwillig} ein weiteres anrechenbares Mathemodul. Die Mathematikmodule unterscheiden sich stark von der Schulmathematik. Unterschätzt den Aufwand für die Vorlesungen nicht! Wollt ihr euer Informatikstudium mathematisch ausrichten (z.B. Richtung Wissenschaftliches Rechnen), empfehlen wir, die Mathevorlesungen LA und Ana bei den Mathematikern zu hören.

\subsection{Programmieren. Und dann\dots}

\begin{figure*}[b]
    \centering
    \begin{subfigure}{.23\textwidth}
	    \includegraphics[height=5cm]{bilder/backing_up_1.png}
    \end{subfigure}
    \begin{subfigure}{.23\textwidth}
	    \includegraphics[height=5cm]{bilder/backing_up_2.png}
    \end{subfigure}
    \begin{subfigure}{.23\textwidth}
	    \includegraphics[height=5cm]{bilder/backing_up_3.png}
    \end{subfigure}
    \begin{subfigure}{.23\textwidth}
	    \includegraphics[height=5cm]{bilder/backing_up_4.png}
    \end{subfigure}
\end{figure*}

Programmieren ist eine wichtige Fertigkeit, die ihr außerhalb von IPI und IPK hauptsächlich im Selbststudium erlernen oder vertiefen werdet. Solide Kenntnisse von unixoiden Systemen, wie Linux und OSX, sind immer Gold wert. Im Informatikstudium wird \emph{fast ausschließlich mit Linux} gearbeitet. Die Uni hilft aber auch noch ein bisschen: Neben den Pflichtkursen IPI und IPK gibt es in der Num0 praktische Übungen, in denen ebenfalls programmiert wird. Eigeninitiative ist aber dennoch wichtig, mit interessanten Projekten macht das aber auch enorm viel Spaß.


\subsection{Anwendungsgebiet}

Neben der wunderbaren Welt der Informatik sollt ihr euch aber auch mit deren Anwendungsgebieten auseinandersetzen. Dazu müsst ihr (24 \gls{LP}) in mindestens einem anderen Fach erreichen. Entgegen der häufigen Vermutung hört ihr hier also einfach fachfremde Vorlesungen, um euren Horizont interdisziplinär zu erweitern, in denen ihr weder programmiert, noch sonstige Informatikkenntnisse vermittelt bekommt. Obwohl nur wenige Fächer im Modulhandbuch als Anwendungsgebiete vorgeschlagen werden, könnt ihr nach \emph{vorheriger} schriftlicher Bestätigung eures Prüfungssekretariats auch andere Fächer wählen, die euch interessieren. Dabei sind die Module von den Fachbereichen normalerweise fest vorgeschrieben, Nachfragen im Prüfungssekretariat des Anwendungsgebiets lohnt sich aber, da es manchmal doch einige Wahlmöglichkeiten gibt.


\subsection{Orientierungsprüfung}

Die „Einführung in die Praktische Informatik“ \emph{müsst} ihr, als eure Orientierungsprüfung, bis zum Ende des dritten Semesters bestanden haben.


\subsection{Prüfungen: wie und wieso?}

Um in den Vorlesungen \gls{LP} zu erhalten und das sprich das Modul abzuschließen, müsst ihr fast immer eine Klausur am Semesterende bestehen. Über die genauen Prüfungsmodalitäten informieren euch eure Dozierenden jeweils am Anfang des Semesters. Meist müsst ihr für die Klausurzulassung eine Mindestpunktzahl (i.d.R. 50\%) aus den Übungszetteln erreichen.

Grundsätzlich habt ihr pro Prüfung zwei Versuche (z.B. wenn ihr krank wart, oder beim ersten Versuch nicht bestanden habt). Einige Dozierende weichen aber von der Standardregelung ab, teilen euch dies allerdings direkt am Anfang des Semesters mit. \emph{Informiert euch also am Anfang des Semesters} immer genau über die Bedingungen! Solltet ihr eine Prüfung auch beim Zweitversuch nicht bestehen, so besteht in bis zu vier Fällen die Möglichkeit, die Prüfung \emph{auf Antrag beim Prüfungsausschuss} ein weiteres Mal zu wiederholen (gilt nicht für Orientierungsprüfung und Bachelorarbeit). Besteht ihr auch diesen Versuch nicht, verliert ihr euren Prüfungsanspruch endgültig und werdet exmatrikuliert.

\emph{Klausuranmeldungen sind immer verbindlich}. Überlegt euch also gut, ob ihr euch anmeldet. Solltet ihr euch sicher sein, die Klausur nicht zu bestehen, empfiehlt es sich, sich nicht anzumelden und die Vorlesung im nächsten Jahr noch einmal zu hören. Das Schöne an eurem Studiengang ist, dass die Noten der Grundpflichtmodule (IPI, IPK, ITI, LA 1 und Ana 1 bzw. MafIn 1 und 2) nicht in eure Abschlussnote zählen. Das heißt, dass auch wenn die ersten Semester mit der ganzen Mathematik vielleicht etwas hart sind, eure Abschlussnote darunter nicht unbedingt zu leiden hat.

\begin{figure}[h]
\centering
\includegraphics[width=.5\linewidth]{bilder/haskell.png}
\end{figure}

%\vspace{-\parskip}
