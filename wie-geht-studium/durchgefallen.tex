
\begin{figure*}[t]
    \centering
    \begin{subfigure}[b]{.18\textwidth}
        \includegraphics[width=\linewidth]{bilder/the_difference_1.png}
    \end{subfigure}
    \begin{subfigure}[b]{.18\textwidth}
        \includegraphics[width=\linewidth]{bilder/the_difference_2.png}
    \end{subfigure}
    \begin{subfigure}[b]{.18\textwidth}
        \includegraphics[width=\linewidth]{bilder/the_difference_3.png}
    \end{subfigure}
    \begin{subfigure}[b]{.18\textwidth}
        \includegraphics[width=\linewidth]{bilder/the_difference_4A.png}
    \end{subfigure}
    \begin{subfigure}[b]{.18\textwidth}
        \includegraphics[width=\linewidth]{bilder/the_difference_4B.png}
    \end{subfigure}
\end{figure*}

\vspace{-3mm}
\section{Durchgefallen -- Was tun?}

In eurem Studium wird es aller Voraussicht nach hin und wieder vorkommen, dass ihr durch die eine oder andere Klausur durchfallt. Das ist an und für sich auch kein großes Problem. Zuerst einmal ist es aber wichtig, zu wissen, was es eigentlich heißt, „durchgefallen“ zu sein.

Die meisten Leute setzen „durchgefallen“ mit „die Klausur nicht bestanden haben“ gleich. Das ist erst mal nicht ganz falsch, aber auch nicht ganz richtig. Ihr müsst nämlich zwischen der Klausur und Prüfungsleistung unterscheiden. In fast allen Grundvorlesungen werden eine Klausur am Anfang der Semesterferien und eine Nachklausur kurz vor Beginn des neuen geschrieben, die jeweils als ein einzelner Prüfungsversuch zählen.\footnote{Früher waren diese aneinander gekoppelt und galten als ein Prüfungsversuch, dafür gab es aber weniger Prüfungsversuche.}
Erst wenn ihr alle Prüfungsversuche, die euch für das Modul zustehen, nicht bestanden habt, habt ihr die entsprechende Prüfungsleistung für das Modul nicht bestanden. Was die konkrete Anzahl an Prüfungsversuchen angeht, müsst ihr euch gut informieren, diese variiert nämlich zwischen den Modulen. Besteht ihr alle (regulären) Prüfungsversuche nicht, könnt ihr einen Antrag beim Prüfungsauschuss stellen, und versuchen, vor dem Prüfungsausschuss zu begründen, warum ihr die Klausuren nicht bestanden habt. In der Physik habt ihr zusätzlich in zwei Modulen einen dritten Prüfungsversuch.\footnote{Das nennt sich dann „Jokerregelung“} Wie genau das geregelt ist, steht an entsprechender Stelle in der Prüfungsordnung. In der Informatik habt ihr die Chance in zusätzlich drei Modulen einen dritten Prüfungsversuch durchzuführen.

\subsection{Orientierungsprüfung}
Eine Ausnahme von dieser Regelung stellt die sogenannte Orientierungsprüfung dar. Diese Prüfung soll feststellen, ob ihr überhaupt geeignet seid, das Fach, das ihr studiert, zu studieren. Diese Prüfung müsst ihr, im Gegensatz zu allen anderen Prüfungen, bis spätestens zum Ende des dritten Semesters erbracht haben, und die Jokerregelung gilt hier nicht. In der Informatik habt ihr aber zum Beispiel, da es sich um ein Grundmodul handelt, sowieso vier Prüfungsversuche.

\subsection{Muss ich nochmal Zettel rechnen?}
In der Mathematik und Informatik sind Klausurzulassungen ein Jahr gültig. Dort musst du das also nicht. In der Physik kommt das auf die Dozentin deiner Veranstaltung an. Es gibt einige, die finden, dass man die Zettel nicht nochmal rechnen muss, einige wollen, dass du die Zulassung noch mal neu erwirbst. So oder so ist es unglaublich hilfreich, die Zettel trotzdem noch mal zu rechnen. Zwei mal die gleiche Veranstaltung bei zwei verschiedenen Professorinnen ist eben doch ein Unterschied, und dass du Nachholbedarf hast, hast du ja schon gezeigt ;)
