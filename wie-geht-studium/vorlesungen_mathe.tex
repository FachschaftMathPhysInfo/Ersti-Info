\section*{Vorlesungen Mathematik}

\subsection{Lineare Algebra 1}
\label{la1}
Die \vl{Lineare Algebra 1} (\gls{LA}) hören Mathematikerinnen, Physikerinnen und (eventuell) Informatikerinnen zusammen, bereitet euch also auf eine sehr große Vorlesung vor. Neben den vielen Grundlagen, die euch hier vermittelt werden, handelt es sich inhaltlich um die Vektorrechnung, wie ihr sie bereits aus der Schule kennt. Diese wird jedoch viel allgemeiner und abstrakter als bisher eingeführt, was am Anfang etwas umständlich erscheint, aber wie sich später offenbart, viele Vorteile bringt. Im weiteren Verlauf kommen auf der Vektorrechnung aufbauend noch lineare Operatoren und Innenprodukträume hinzu, die euch Begriffe und Sätze wie Determinanten, Eigenwerte oder den Spektralsatz näher bringen. Die Inhalte werden euch euer ganzes Studium begleiten, da praktisch alle höheren Vorlesungen auf das mächtige Werkzeug der \vl{Linearen Algebra} zurückgreifen.

\vspace{-2mm}
\subsection{Analysis 1}
\label{ana1}
Die \vl{Analysis 1} (\gls{Ana}) muss von den Mathematikerinnen gehört werden, jedoch sind auch Physikerinnen und Informatikerinnen nicht unbedingt schlecht beraten, daran teilzunehmen. Hier lernt ihr richtige, fundierte Mathematik in all ihrer Schönheit und Abstraktion, was sehr abstrakt sein kann. Diese Vorlesung hat mit dem Matheunterricht aus der Schule ähnlich viel gemeinsam wie mit dem Sportunterricht.

Ihr lernt das formell richtige Argumentieren und Beweisen und erhaltet einen Einblick darin, was das Gebäude der Mathematik eigentlich ausmacht und wie dieses aufgebaut ist. Inhaltlich beginnt sie mit der Konstruktion der reellen Zahlen, führt über Folgen und Reihen zur Stetigkeit von (reellen) Funktionen und schließlich zur Differential- und Integralrechnung. Der Arbeitsaufwand für die Vorlesung schwankt (je nach Prof) zwischen lächerlich und immens, auch hier sind zehn oder mehr Stunden für einen Zettel nicht unbedingt Seltenheit, es lohnt sich Zettelgruppen zu bilden und gemeinsam zu rechnen. Was das unglaublich frustrierende und hilflose Gefühl betrifft, man würde nichts verstehen und wäre völlig falsch in seinem Studiengang, keine Angst, das haben alle. Wenig härtet so gut gegen Frust ab wie eine erbarmungslose Mathevorlesung im ersten Semester. Aber lasst euch davon nicht täuschen, dass Begriffe, die ihr meint aus der Schule zu kennen, unnötig umständlich eingeführt werden. Es hat alles durchaus seine mathematische Berechtigung und schafft ein Fundament, auf das ihr später aufbauen werdet. Sich ein sauberes, formelles Vorgehen und Denken anzugewöhnen, ist unabdingbar.
