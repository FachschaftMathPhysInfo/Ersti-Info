\section*{Vorlesungen Informatik}

\subsection{Einführung in die Praktische Informatik}
\label{info1}
Die \vl{Einführung in die Praktische Informatik} (\gls{IPI}) ist für alle Informatik- und Mathe-Bachelors verpflichtend. Einstieg bilden einige Grundstrukturen und Abläufe der Informatik, die dann zur Problemlösung herangezogen werden. Dies geschieht meist in einem C++-Programm. Idealerweise hat man am Ende genug Herangehensweisen angehäuft, um Aufgaben vor dem geistigen Auge zu modellieren und später in richtigen Code umsetzen zu können. Vor allem Mathematikerinnen sollten die Vorlesung nicht unterschätzen, auch wenn der geringe Aufwand dazu verleitet. Spätestens mit der \vl{Einführung in die Numerik} (\gls{Num0}) muss wieder programmiert werden --- euch darum drücken könnt ihr also nicht. Auch für Lehrämtlerinnen kann es interessant sein, die Vorlesung zu hören, um sich später den Einstieg in die Numerik zu vereinfachen. Lehrämtlerinnen in Mathematik können sich diese Vorlesung allerdings nicht (für Mathematik) anrechnen lassen!

\vspace{-3mm}
\subsection{Programmierkurs}
\label{ipk}
Der \vl{Programmierkurs} (\gls{IPK}) ist für alle Informatik-Studentinnen eine Pflichtveranstaltung. Je nach Angebot wird er als Vorlesung mit zwei Semesterwochenstunden oder als Blockkurs in den Ferien gelesen. Im Gegensatz zur praktischen Informatik lernt ihr hier keine informatischen Konzepte, sondern das Programmieren in der Sprache C++. Man sollte diesen Kurs nicht auf die leichte Schulter nehmen, denn die Klausur wird entweder an einem PC mit Linux durchgeführt und eure Programme müssen korrekt laufen oder ihr müsst auf Papier programmieren. In der Regel lernt ihr zunächst die grundlegenden Datentypen, Operationen und verschiedene Kontrollstrukturen, und am Ende wird objektorientiert programmiert, \zB mit Vererbung, Templates und den Methoden aus der Standard-Bibliothek.

\vspace{-3mm}
\subsection{Einführung in die Technische Informatik}
\vspace{-1mm}
\label{info2}
Die \vl{Einführung in die Technische Informatik} (\gls{ITE}) ist eine Pflichtvorlesung für alle Informatik-Studentinnen. Hier lernt ihr Konzepte aus der binären Logik, mit denen Prozessoren arbeiten. Nebenbei beschäftigt ihr euch mit verschiedenen Schaltungen und mit Rechenschemata in verschiedenen Zahlensystemen (z.B. Binärsystem, Umwandlung vom Oktal- ins Hexadezimal-System und umgekehrt, Addition und Multiplikation auf diesen Systemen usw.). Im letzten Drittel der Vorlesung lernt ihr etwas Allgemeines zu Rechnerarchitekturen. Insgesamt ist es für das erste Semester eine sehr schöne Vorlesung, die euch sanft ins Informatik-Studium einführt.


\subsection{Mathematik für Informatiker}
\label{mafin}
Die Mathematikausbildung für Informatikerinnen ist etwas kompliziert. Ihr habt die Auswahl zwischen den Vorlesungen \vl{Lineare Algebra~1} (\gls{LA}) und \vl{Analysis~1} (\gls{Ana}) oder den Vorlesungen \vl{Mathematik für Informatiker~1} und \vl{2} (\gls{MafIn}). Ihr entscheidet euch also bereits im ersten Semester für eine der beiden Varianten. Deshalb solltet ihr euch schon jetzt mit diesem Thema auseinandersetzen.

\subsubsection{Was spricht für MafIn?}
Die Vorlesung ist extra auf Informatikerinnen zugeschnitten und versucht, euch den Stoff besser verständlich näher zu bringen. Die \gls{MafIn}~1 orientiert sich dabei an den grundlegenden Inhalten der \gls{LA}~1, die \gls{MafIn}~2 an denen der \gls{Ana}~1. Grundsätzlich ist die Vorlesung solide konzipiert und kann euch die wichtigen Mathematik-Grundlagen vermitteln. Wie bei jeder Vorlesung hängt natürlich auch hier das Konzept und die Qualität stark vom jeweiligen Lehrpersonal ab, das sich in den letzten Jahren nicht geändert hat. Ihr solltet euch aber selbst ein Bild von der Vorlesung machen, ob sie euch zusagt oder nicht. Lasst uns gerne Feedback zur Vorlesung zukommen.

\subsubsection{Was spricht für Lineare Algebra 1 und\\Analysis 1?}
Wie oben beschrieben sind diese Vorlesungen für Mathematikerinnen konzipierte Veranstaltungen. Diese bieten eine präzisere Herangehensweise an, vor allem bei der Formulierung von Definitionen, Sätzen und insbesondere Beweisen. Ihr bekommt dadurch eine solidere Mathematikausbildung, die ihr in manchen Bereichen der Informatik auch dringend benötigt. Die \gls{LA} 1 und \gls{Ana} 1 bedeutet für euch aber ziemlich sicher auch mehr Arbeit und Zeitaufwand, da ihr auch eine intensivere und vertiefendere Ausbildung bekommt. Es heißt auch, dass ihr im ersten Semester neben \gls{IPI}, \gls{LA}~1 und \gls{Ana}~1 kaum noch Zeit habt, die \vl{Einführung in die Technische Informatik} (\gls{ITE}) zu hören.\\

Schlussendlich müsst ihr selbst wissen, was für euer Studium und euren Studienplan das beste ist. Informiert euch möglichst zu Beginn eures Studiums, was für euch besser passt. Im Zweifel könnt ihr euch auch in beide Vorlesungen rein setzen und die Art der Vorlesungen vergleichen. Macht dies aber nicht länger als ein paar Wochen, da der Mehraufwand gerade im ersten Semester sehr hoch ist. Im Zweifel bekommt ihr in der Regel während des Vorkurses auch noch einen Vortrag zum Studienaufbau, bei dem ihr noch Fragen stellen könnt. Ansonsten freuen wir uns, wenn ihr mit Fragen und Anregungen zu dieser schwierigen Entscheidung zu uns\footnote{Mehr Infos zur Fachschaft im \autoref{diefsmathphys}.} kommt.
