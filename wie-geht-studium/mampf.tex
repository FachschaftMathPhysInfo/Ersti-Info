
\section{MaMpf}
\label{mampf}

\noindent So mancher Digital Native treibt sich lieber im Netz als in Bibliotheken herum, springt lieber zwischen Tabs als Buchseiten hin und her. Ein Glück, dass man auch im Neuland Mathe lernen kann!

\begin{wrapfigure}[7]{r}{2cm}
\vspace{-3mm} % bessere Position für das textumflossene Bild
\hspace*{-7mm}
    \ifthenelse{\boolean{druckversion}}{
	  \includegraphics[width=2.5cm]{bilder/mampf-logo_512x512_mon.png}
    }{
        \includegraphics[width=2.5cm]{bilder/mampf-logo_512x512.png}
    }
\end{wrapfigure}

Seit 2016 gibt es nämlich MaMpf\footnote{\url{https://mampf.mathi.uni-heidelberg.de/}} („Mathematische Medienplattform“). Die Plattform wurde von Dr. Denis Vogel initiiert und hat sich inzwischen zu einem stattlichen E-Learning-Projekt gemausert. Zu den umfassenden MaMpf-Lernangeboten, von denen sich viele speziell an Erstis richten, gehören: 
 
\subsection{Vorlesungsvideos}
Auf MaMpf kannst du dir eine Vielzahl an Videos aktueller und vergangener Vorlesungen und dabei entstandener Aufschriebe anschauen. Auch wenn deine Dozentin selbst keine Videos produziert und hochlädt, können die Videos eine wertvolle Ressource sein, beispielsweise wenn du eine Vorlesung verpasst hast, irgendwo im Beweis hängst oder dich eine Definition vollkommen verwirrt. Die für dich relevanten Videos und Passagen findest du ganz einfach über die Suchfunktion oder die Tags bei den Videos. An die Stelle im Video, die dich interessiert, springst du dann über die Vorlesungsgliederung im MaMpf-eigenen THymE-Player („The Hypermedia Experience“).

Es gibt auch ein „Annotation Tool“, mit dem du im Video private Notizen anlegen kannst (sofern die Dozentin das Feature freigeschaltet hat). Wenn du möchtest, kannst du deine Notizen auch öffentlich zugänglich machen und mit der Dozentin teilen, \zB Fehler in der Vorlesung.

\subsection{Beispielvideos}
Ein gut erklärtes Beispiel bringt manchmal mehr fürs Verständnis als das stundenlange böse Anstarren einer Definition oder eines Satzes. Darum gibt es auf MaMpf Videos („Worked Examples“), in denen typische Beispiele zu wichtigen Algorithmen und Methoden detailliert vorgerechnet und zentrale Begriffe erklärt werden. Oft werden in diesen Videos auch nützliche Tipps gegeben und hilfreiche Tricks gezeigt.

\subsection{Quizzes}
Multiple-Choice-Fragen sind Teil vieler Matheklausuren der ersten Semester. Da Übung bekanntlich den Meister macht, tummeln sich auf MaMpf über 2500 Fragen, an denen du dein Können erproben kannst. Zu vielen dieser Fragen bekommst du nicht nur die Rückmeldung \emph{wahr}/\emph{falsch}, sondern auch eine ausführliche Erklärung in Videoform. Daher eignen sich die Quizzes sowohl zum Üben und Wiederholen als auch zum Vertiefen.

\subsection{Weitere Lernangebote}
Vorlesungsvideos, Beispielvideos und Quizzes sind längst nicht alles, was MaMpf zu bieten hat.
Auf der Medienplattform stehen dir außerdem Wiederholungsvideos, angeleitete Beweise, Übungsblätter, Vorlesungsskripte und die intelligente mathematische Datenbank ErDBeere\footnote{„Erkenntnisfördernde Datenbank zur Beispielerfassung und -entwicklung“} zur Verfügung. Zudem kannst du MaMpf als Nachschlagewerk verwenden und über die Tag-Netze Zusammenhänge erkunden. Wenn du mehr über MaMpf erfahren möchtest, kannst du dem MaMpf-Blog\footnote{\url{https://mampf.blog}} einen Besuch abstatten.  

\subsection{MaMpf nutzen}
Um MaMpf nutzen zu können, musst du dich nur mit deiner E-Mail-Adresse registrieren. In den Profileinstellungen abonnierst du dann die Module deiner Wahl, z.B. \vl{Analysis~1} und \vl{Lineare Algebra~1} ;-) und navigierst über das blaue Drop-Down-Menü am Seitenanfang zwischen ihnen. Wenn du ein Modul in diesem Menü ausgewählt hast, gelangst du über ein weiteres Menü auf der linken Bildschirmseite zu allen erwähnten Ressourcen. Wenn du Fragen oder Feedback zu Mampf hast, kannst du den neuen Feedback-Button neben der Suchleiste nutzen.
