% !TEX ROOT = ../ersti.tex
\section*{Vorlesungen Physik}


\subsection{Experimentalphysik 1: \\Mechanik und Wärmelehre}
\label{ex1}
Die Experimentalphysik 1 (\gls{Ex} 1) ist eigentlich eine sehr angenehme Vorlesung, um ins erste Semester zu starten, zumindest sofern ihr in der Schule irgendwann mal Physik hattet. Echte Verständnishürden stellen sich gerade im Mechanikteil im Allgemeinen nicht. Im Großen und Ganzen stellt sie einen Schnelldurchlauf durch die Entwicklung der Mechanik dar, von den Anfängen (diese liegen, je nach Prof, zwischen den alten Griechen und Newtons Axiomen) bis zu etwas ausgefeilteren Sachen, die sich aber alle noch im Rahmen des Schulstoffs bewegen sollten. Was diese Vorlesung dennoch von der Schule unterscheidet, ist die \emph{Art} der Präsentation, größtenteils frontaler Vortrag, gespickt mit vielen Experimenten (manchmal durchaus eine Art „Knoff-Hoff-Show“) und sehr viel schneller und mathematischer als ihr das von eurer Lehrerin gewohnt seid. Trotzdem ist dies die Vorlesung mit dem wahrscheinlich größten Unterhaltungswert eurer ganzen Universitätskarriere.



\subsection{Theoretische Physik 1: \\Mechanik und mathemat. Methoden}
\label{theo1}
Die Theoretische Physik 1 (\gls{Theo} 1) beschäftigt sich mit der Newton'schen Mechanik und nützlichem mathematischen Werkzeug.

Hier werdet ihr im ersten Semester einige Zusammenhänge und Techniken einfach „vorgesetzt“ bekommen, ohne sie völlig zu verstehen. Der genaue Grund, warum ihr das, was ihr da tut, eigentlich dürft, wird im Normalfall erst in einer der späteren Mathevorlesungen klar, daran solltet ihr aber nicht verzweifeln. Dieses Vorgehen ist in der Physik nicht ungewöhnlich, was einer der Angriffspunkte von Witzen der Mathematikerinnen über Physikerinnen ist \dots

Ihr erhaltet hier aber nicht nur die mathematischen Techniken, die ihr in eurem Studium brauchen werdet und von denen ihr in vielen Fällen noch nie was gehört habt, sondern euch wird auch eine theoretische Beschreibung der Mechanik vorgestellt. Diese unterscheidet sich im ersten Semester, bis auf einige seltsame Symbole und unglaubliche Umständlichkeit noch nicht sehr von der in der Experimentalphysik 1, ab dem zweiten Semester tun sich zwischen den Sichtweisen jedoch Abgründe auf und ihr werdet verstehen, warum auf diese Umständlichkeit bestanden wurde.

Der Anspruch dieser Vorlesung an Verständnis und Wissen ist deutlich größer als in der Ex 1, womit ihr auch einen erheblich höheren Aufwand für die Bewältigung der Arbeitszettel einplanen könnt, je nach eigenem Interesse, Wissen und Perfektionismus sind zehn Stunden durchaus eine realistische Einschätzung für einen Zettel. Die Übungsgruppenleiterinnen sind hier vor allem ältere Studierende, was den Vorteil hat, dass diese sich noch an ihre eigenen Probleme in ihrer Theo 1 erinnern können und euch das Ganze verständlicher erklären können als die Theo-Professorin es kann.

\begin{figure}[b!]
	\centering{
    \includegraphics[width=\linewidth]{bilder/log_scale.png}
}
\end{figure}

