% !TEX ROOT = ../ersti.tex
\section*{Vorlesungen Physik}


\subsection{Experimentalphysik 1: \\Mechanik und Wärmelehre}
\label{ex1}
Die \vl{Experimentalphysik 1} (\gls{Ex} 1) ist eigentlich eine sehr angenehme Vorlesung, um ins erste Semester zu starten, zumindest sofern ihr in der Schule irgendwann mal Physik hattet. Echte Verständnishürden stellen sich gerade im Mechanikteil im Allgemeinen nicht. Im Großen und Ganzen stellt sie einen Schnelldurchlauf durch die Entwicklung der Mechanik dar, von den Anfängen (diese liegen, je nach Prof, zwischen den alten Griechen und Newtons Axiomen) bis zu etwas ausgefeilteren Sachen, die sich aber alle noch im Rahmen des Schulstoffs bewegen sollten. Was diese Vorlesung dennoch von der Schule unterscheidet, ist die \emph{Art} der Präsentation, größtenteils frontaler Vortrag, gespickt mit vielen Experimenten (manchmal durchaus eine Art „Knoff-Hoff-Show“) und sehr viel schneller und mathematischer als ihr das von eurer Lehrerin gewohnt seid. Trotzdem ist dies die Vorlesung mit dem wahrscheinlich größten Unterhaltungswert eurer ganzen Universitätskarriere.



\subsection{Theoretische Physik 1: \\Mechanik und mathemat. Methoden}
\label{theo1}
Die \vl{Theoretische Physik 1} (\gls{Theo} 1) beschäftigt sich mit der Newton'schen Mechanik und nützlichem mathematischen Werkzeug.

Hier werdet ihr im ersten Semester einige Zusammenhänge und Techniken einfach „vorgesetzt“ bekommen, ohne sie völlig zu verstehen. Der genaue Grund, warum ihr das, was ihr da tut, eigentlich dürft, wird im Normalfall erst in einer der späteren Mathevorlesungen klar, daran solltet ihr aber nicht verzweifeln. Dieses Vorgehen ist in der Physik nicht ungewöhnlich, was einer der Angriffspunkte von Witzen der Mathematikerinnen über Physikerinnen ist \dots

Ihr erhaltet hier aber nicht nur die mathematischen Techniken, die ihr in eurem Studium brauchen werdet und von denen ihr in vielen Fällen noch nie was gehört habt, sondern euch wird auch eine theoretische Beschreibung der Mechanik vorgestellt. Diese unterscheidet sich im ersten Semester, bis auf einige seltsame Symbole und unglaubliche Umständlichkeit noch nicht sehr von der in der \vl{Experimentalphysik 1}, ab dem zweiten Semester tun sich zwischen den Sichtweisen jedoch Abgründe auf und ihr werdet verstehen, warum auf diese Umständlichkeit bestanden wurde.

Der Anspruch dieser Vorlesung an Verständnis und Wissen ist deutlich größer als in der \gls{Ex} 1, womit ihr auch einen erheblich höheren Aufwand für die Bewältigung der Arbeitszettel einplanen könnt, je nach eigenem Interesse, Wissen und Perfektionismus sind zehn Stunden durchaus eine realistische Einschätzung für einen Zettel. Die Übungsgruppenleiterinnen sind hier vor allem ältere Studierende, was den Vorteil hat, dass diese sich noch an ihre eigenen Probleme in ihrer \gls{Theo} 1 erinnern können und euch das Ganze verständlicher erklären können als die Theo-Professorin es kann.

\vspace{-2mm}
\subsection{Höhere Mathematik für Physiker}
\label{mathephysik}

Die Mathematikausbildung für Physikerinnen sieht vor, dass ihr die \vl{Lineare Algebra 1} (\gls{LA}) im ersten Semester hört. Laut Modulhandbuch könnt ihr euch dann vor dem zweiten Semester entscheiden, ob ihr mit \vl{Analysis 2} und \vl{3} (\gls{Ana}, was auch die Mathematikerinnen hören) oder mit \vl{Höhere Mathematik für Physiker (\gls{HoMa}) 2} und \vl{3} weitermacht.

\subsubsection{Was spricht für die HöMa?}
Die Vorlesung ist extra auf euch als Physikerinnen zugeschnitten und legt ihren Schwerpunkt auf die Vorlesungen \gls{Ana} 1-3 in strafferer Form. Während Mathematikvorlesungen einer gewissen Freiheit unterliegen und es durchaus vorkommen kann, dass die Dozentin einen Schwerpunkt auf ihr Forschungsgebiet legt, hört ihr in HöMa größtenteils nur jene Dinge, die in der Physik auch verwendet werden. Außerdem kommen Beispiele gerne aus der Physik und liegen euch deshalb vielleicht näher. Trotzdem handelt es sich nicht um eine Schmalspurversion, sondern um eine vollwertige Mathematikvorlesung, die auch für zukünftige Theoretikerinnen nicht ungeeignet ist.

\subsubsection{Was spricht für die Analysis?}
Die Analysis bietet als eine für Mathematikerinnen konzipierte Veranstaltung eine präzisere Formulierung der Definitionen, Sätze und insbesondere Beweise. Somit wird es möglich, die mathematischen Hintergründe in der Physik besser zu durchdringen und weitergehende Verbindungen der Gebiete zu erkennen. Dieses tiefere Verständnis kann unter anderem in der theoretischen Physik oder auch in weiterführenden Matheveranstaltungen von Vorteil sein und lässt euch insgesamt mehr Freiheiten im weiteren Studienverlauf, besonders bezüglich der Mathematik. Zudem ist je nach Dozentin und Forschungsbereich auch eine gewisse Schwerpunktlegung (vor allem in der \gls{Ana} 3) möglich.\\

Auf den ersten Blick mag es verwundern, dass ihr in die \gls{Ana} 2 einsteigen sollt, ohne die erste Vorlesung dazu gehört zu haben. Dies ist theoretisch zumindest möglich, jedoch vermutlich mit ein wenig Mehraufwand verbunden. Trotzdem können mathematisch Ambitionierte natürlich auch die \gls{Ana} 1 im ersten Semester hören, da diese eine schöne Einführung in den Themenbereich Analysis und die damit verbundenen Methoden darstellt. Das kann euch vor allem in weiterführenden Vorlesungen weiterhelfen; auch wird es eine große Erleichterung für die \gls{Ana} 2 sein, wenn ihr schon ein wenig mehr mit der Materie und der Dozentin vertraut seit. Andererseits werdet ihr mit dem Kursprogramm auch so schon stark ausgelastet sein. Wenn ihr es mit vier Vorlesungen versuchen wollt, solltet ihr euch nach zwei bis drei Wochen entschieden haben, ob ihr das im ersten Semester durchhaltet oder nicht, da es für den Übungsbetrieb ziemlich blöd ist, wenn in der Mitte des Semesters viele Leute aussteigen.

\begin{figure}[b!]
	\centering{
    \includegraphics[width=\linewidth]{bilder/log_scale.png}
}
\end{figure}

