% !TEX ROOT = ../ersti.tex

\section{Zettelrechnen/Übungszettel}
\label{uebungsgruppen}

%\subsection{Übungszettel}
\noindent In den meisten Vorlesungen müsst ihr, um zur Klausur zugelassen zu werden, Übungszettel rechnen. Das bedeutet, jede Woche werden irgendwo\footnote{Wo, wird in der ersten Vorlesung bekannt gegeben} Übungsaufgaben hochgeladen, die ihr euch ausdrucken und dann rechnen sollt. Es ist sinnvoll, die Aufgaben in Gruppen zu bearbeiten, da man gemeinsam bekanntlich besser lernt und sich gegenseitig helfen kann. Dazu kann man sich gut im Mathematikon in den Teeküchen und Sitznischen an der Südseite des Gebäudes, im \gls{KIP} im Foyer und den Sitzecken an der Westseite sowie in INF 308 und 252 treffen. Teilweise kann man sogar noch nach den Schließzeiten drinnen sitzen bleiben und weiterrechen, auch wenn man das Gebäude dann nur noch verlassen und nicht mehr betreten kann. Weil die Übungszettel erfahrungsgemäß nicht so einfach sind und meistens viele Fragen auftreten, gibt es Übungsgruppen.

\subsection{Übungsgruppen}
In fast allen Vorlesungen werden vorlesungsbegleitend sogenannte Übungsgruppen angeboten. Diese Übungsgruppen sind dazu da, euch bei euren Problemen mit der Vorlesung zu helfen und die Zettel nachzubesprechen. Dazu seid ihr natürlich nicht auf euch allein gestellt, sondern euch wird eine Tutorin zur Seite gestellt, die euch bei allen auftretenden Fragen kompetente Hilfe bietet. Hier unterscheiden sich die Mathe/Informatik und die Physik: In der Physik werden die Übungsgruppen meistens mindestens von Masterstudentinnen, häufig auch von Doktorandinnen oder gar Profs gehalten. Dementsprechend studierendenfern sind diese oftmals, aber natürlich gibt es auch Tutorinnen, die geradezu geniale Übungsgruppen halten. Grundsätzlich gilt es aber, so viele Fragen wie möglich zu stellen, auch um der Tutorin Feedback zu eurem Kenntnisstand zu geben. In der Mathe und Info werden die Übungsgruppen in den meisten Fällen von Studierenden höherer Semester gehalten, es ist durchaus nicht unüblich, eine Tutorin zu haben, die die Vorlesung selbst erst vor zwei Semestern gehört hat.

\subsection{Was bringt mir das?}
Die Übungszettel und die dazugehörigen Übungsgruppen sind erfahrungsgemäß der Ort, an dem euch der Stoff der Vorlesung nahe gebracht wird und ihr anfangt zu verstehen, was eigentlich in der Vorlesung vor sich geht. Eigentlich ist die Übungsgruppe also dazu da, die Vorlesung mit euch nachzuarbeiten, an den Stellen, an denen nicht klar ist, was passiert, Hilfe zu bieten, mit euch die Zettel zu besprechen und einfach nochmal eine andere Darstellung zu liefern. Leider kommt es viel zu oft vor, dass in der Übung nur die Aufgaben „runtergerechnet“ werden, man irgendwann nicht mehr aufpasst und am Ende auch nicht mehr weiß als vorher. Wenn das passiert, fragt penetrant nach dem Sinn der Aufgabenstellung, nach dem Zusammenhang mit der Vorlesung oder nach was euch sonst noch so durch den Kopf geht. Das ist eure einzige Chance, noch etwas aus der Rechnerei zu lernen.

\subsection{Anmeldung}
Ganz häufig stellen sich anfangende Studierende die Frage, ob sie sich zu den Übungen anmelden müssen. Die Antwort ist meistens \emph{ja}. In der Physik nutzt ihr dazu das physikinterne Übungsgruppensystem\footnote{\url{https://uebungen.physik.uni-heidelberg.de/uebungen/}}. Das System ist leider nicht besonders leistungsfähig. Da kann es schon mal vorkommen, dass es zu Zeiten, in denen die Übungsgruppenanmeldung großer Vorlesungen freigeschaltet werden, abstürzt. Das ist aber auch kein Drama und insbesondere kein Grund, das Rektorat zu alarmieren\footnote{wann das wohl passiert ist \dots}, denn nach einigen Minuten hat sich das System dann auch wieder gefangen. Die Mathe geht wie immer ihren eigenen Weg  und hat mit dem MÜSLI\footnote{Mathematisches Übungsgruppen- und Scheinlisten-Interface: \url{https://muesli.mathi.uni-heidelberg.de}} ihr eigenes System. Das ist deutlich leistungsfähiger, deutlich besser und viel schöner, aber man kann als Physikerin ja nicht alles haben. Eigentlich ist es Pflicht, für Mathematik und Informatik das MÜSLI für die Anmeldung zu Veranstaltungen zu nutzen, aber das wird auch gerne mal von Dozentinnen ignoriert. Die Informatik wiederum nutzt in 90\% aller Fälle Moodle\footnote{\url{https://moodle.uni-heidelberg.de/}}, das E-Learning-System der Uni Heidelberg. 90\% aller Nutzerinnen sind sich einig, dass dieses System ganz doll Grütze ist, aber davon hat sich die Informatik noch nie beeinflussen lassen.

\subsection{Was mache ich mit der Arbeitslast?}
Dass ihr die Zettel in Gruppen rechnen sollt, ist kein Versuch, hochgepriesene Softskills zu entwickeln. Es geht darum, dass die Zettel manchmal einfach extrem lang brauchen können, auch wenn durchschnittlich vier Aufgaben nach sehr wenig klingt. Man nehme noch dazu, dass der Vorlesungsinhalt zumindest ein wenig begriffen sein muss, bevor man losrechnen kann; es könnte klar werden, dass alleine arbeitend die Zettel nicht immer in einer Woche zu schaffen sind. \\
Arbeitet also bestenfalls gemeinsam an den Zetteln. Aufteilen ist eine valide Lösung, aber um alle Inhalte zu verstehen, lohnt es sich, jede Aufgabe zumindest nachzuvollziehen; zusammen bearbeiten ist natürlich am einfachsten.
Wenn eure Zettelpartnerin mitten im Semester die Vorlesung abbricht, solltet ihr bei eurer Tutorin nachfragen, ob ihr euch einer anderen Zettelgruppe anschließen könnt. \\
Trotz aller Gruppenarbeitsmaßnahmen werdet ihr wahrscheinlich früher oder später inhaltlich verwirrt, überarbeitet und sehr frustriert sein. Es lohnt sich also, eine effektive und Stress-minimierende Arbeitsstrategie zu wählen, damit wird dieser Zustand deutlich weniger schlimm. \\
Das ewige pausenlose drauflosrechnen bringt nicht viel; die zündenden Gedanken, aus denen die Lösung im Grunde besteht, treten wegen der Erschöpfung nicht mehr auf. Stellt euch also einen Wecker, der euch zu Pausen zwingt. Wie lang die Zeit zwischen den Pausen und wie lang die Pausen selbst sein sollen, müsst ihr für euch selbst herausfinden; 45 Minuten Arbeit, gefolgt von 15 Minuten Pause sollten aber ein guter Startpunkt sein. Schließlich wäre es ja gut, wenn die gesamte Zettelgruppe dasselbe System verfolgt. Das hat übrigens auch den Vorteil, dass ihr dann in Arbeitseinheiten zählen könnt, wie lang ihr für einen Zettel durchschnittlich braucht. Das kann sehr motivierend sein, dann braucht der nächste Zettel nicht mehr „irgendwie sehr lang, keine Ahnung wie viel genau, könnte schon unendlich sein“, sondern zum Beispiel acht Arbeitseinheiten.
Und noch etwas zum Thema (endliche) Zeit: setzt euch eine Uhrzeit, nach der ihr nicht mehr an Uni-Sachen arbeitet. Schlaf und Freizeit sind nämlich auch wichtig. Wenn ihr dann Ziele wie vollständig bearbeitete Zettel nicht erreicht, ist das vollkommen in Ordnung; ein richtiger Arbeitszeitenplan . \\
Ganz abgesehen davon, dass es komplett normal ist, Zettel zeitlich und inhaltlich nicht zu schaffen, ist das Wichtigste dabei, die Zettel zu überfliegen, sobald sie herauskommen. Damit könnt ihr planen, wie lang sie brauchen („wie viel inhaltliches Nacharbeiten ist nötig, wie schwer sieht das aus, gibt es Aufgaben, die ihr auf der Stelle abarbeiten könntet, wie zum Beispiel Rechenaufgaben“). Dann könnt ihr die freien Arbeitseinheiten schonmal auf die verschiedenen Übungsblätter aufteilen. Damit wird vermieden, dass der Zettel, der als Letztes abgegeben werden muss, immer am wenigsten bearbeitet wird. \\
Noch eine Sache zur Effektivität: zu Hause prokrastiniert man mehr und wird einfacher abgelenkt; Zettel sind keine Hausarbeiten; sucht euch lieber einen Platz im \gls{KIP} oder im \gls{Mathematikon}. Das gilt auch fürs Nacharbeiten von Vorlesungen, vor allem die, die man nicht besucht. Um Ablenkung zu minimieren, lohnt es sich auch, irgendeine Art papierenes oder digitales Heft mit sich zu führen. Darin können dann alle Gedanken, die nichts mit dem Thema zu tun haben, aufgeschrieben werden; so sind sie erstmal aus dem Kopf. Es hilft auch ein bisschen, wenn ihr euren Arbeitsplatz nur mit notwendigen Sachen befüllt.
