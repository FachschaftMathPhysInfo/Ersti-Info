% !TEX ROOT = ../ersti.tex
\section{Lehrbücher}

Bücher sind in erster Linie eine Geschmackssache. Die meisten Bücher, die hier aufgelistet werden, behandeln Elemente des Stoffes, der für die ersten zwei Semester gebraucht wird. Das richtige Buch für sich selbst zu finden, geht aber nur durch Ausprobieren! Jedem Lerntyp liegen unterschiedliche Herangehensweisen und damit auch unterschiedliche Bücher. Hier für sich selbst den richtigen Weg zu finden, kann durchaus seine Zeit dauern -- macht euch deshalb nicht verrückt wegen der Büchersuche, die Physik und Mathe ist letztendlich in allen Werken die Gleiche.

Es ist jedoch durchaus empfehlenswert, in einer ruhigen Minute mal ein Thema in verschiedenen Büchern nachzulesen. Dabei lernt man nicht nur viel über die eigene bevorzugte Herangehensweise, die unterschiedlichen Blickwinkel können auch dazu beitragen, ein Thema umfassender oder überhaupt erst zu verstehen.

Neben der Möglichkeit, sich in der Universitätsbibliothek Bücher auszuleihen, darf hier auch ein Hinweis auf den Lesesaal nicht fehlen: Im Erdgeschoss der UB befindet sich ein großer Bereich mit Arbeitsplätzen, in dem ein großer Teil der Lehrbücher als Ansichtsexemplare ausliegen. Dort kann man sich die verschiedenen Bücher in Ruhe genauer ansehen, außerdem wird dieser Bereich von manchen Studierenden auch als Lernumgebung sehr geschätzt, da dort absolute Ruhe und eine sehr konzentrierte Lernatmosphäre herrscht und man alle nötigen und unnötigen Nachschlagewerke direkt vor Ort hat. Ähnliches gilt übrigens auch für die Präsenzbibliotheken der Fakultäten (s.u.).

Falls ihr vorhabt, ein Buch zu kaufen, dann lasst euch Zeit dafür und leiht euch die Bücher lieber erst einmal aus -- alle hier vorgestellten Bücher sind im Bestand der Leihbibliothek. Beim ersten Durchblättern eines Buches kann man meistens nicht feststellen, ob einem die Art und Weise der Stoffvermittlung liegt. Das führt häufig im ersten Semester dazu, dass in Extremfällen kleine dreistellige Beträge für Fachliteratur ausgegeben werden, die anfangs als absolut notwendig erscheint und sich dann nach einiger Zeit doch als mehr oder weniger nutzlos erweist, weil einem die spezielle Herangehensweise des Buches zufällig nicht liegt~\dots 

Für viele Vorlesungen haben die Dozentinnen ein Skript erstellt, was häufig eine gute Hilfe bei der Nachbereitung des Stoffes ist. Einige dieser Skripte werden auch gedruckt und im Laufe des Semesters in den Vorlesungen ausgeteilt. Anschließend sind sie im Fachschaftsraum kostenlos erhältlich -- kommt einfach bei uns vorbei und fragt nach, ob das auch bei euren Vorlesungen der Fall ist.

Weiterhin solltet ihr auf jeden Fall die Skriptensammlung\footnote{\url{https://mathphys.info/w/skripte/}} der Fachschaft im Netz durchstöbern.

\subsubsection{Analysis}
\begin{description}
\item[Forster]{
		Die ersten zwei Bände behandeln recht knapp und kompakt den Stoff der ersten zwei Semester des Analysis-Kurses. Der dritte Band ist ebenfalls knapp geschrieben, allerdings sehr umfangreich, so dass meist nicht einmal die Hälfte des Buches im dritten Semester behandelt werden kann. Ein Standardbuch, da es auch sehr preisgünstig ist. Aber zum erstmaligen Lernen nur bedingt geeignet, dagegen zur Prüfungsvorbereitung relativ gut; viele Übungsaufgaben mit Lösungen in einem Extra-Band.}

\item[Königsberger]{
		Ein gut strukturiertes Standardbuch. Es wird nicht nur der Stoff der ersten beiden Semester behandelt, sondern darüber hinaus auch einige damit zusammenhängende oder weiterführende Themen. Es ist deutlich ausführlicher geschrieben als Forster und ist so nicht nur hervorragend zur Prüfungsvorbereitung geeignet, sondern auch begleitend zur Vorlesung.}

\item[Amann, Escher]{
		Ein sehr umfangreiches Buch, welches extrem in die Tiefe geht und eine schöne Querverbindung zur Linearen Algebra schlägt. Wer dieses Buch durchgearbeitet bekommt, hat wohl alles Wissenswerte gut genug verstanden. Allerdings ist es ziemlich kompliziert und damit für die meisten Studis als erstes Buch nicht geeignet.}
\end{description}


\subsubsection{Lineare Algebra}
\begin{description}
\item[Fischer]{
		Ein Standardwerk, das durch seinen günstigen Preis und seine kompakte Darstellung zum wohl meistgelesenen \gls{LA}-Buch geworden ist. Es ist empfehlenswert, wenn man sich nicht von der etwas abstrakten Darstellung abschrecken lässt. Es bringt den vollständigen Stoff der ersten zwei Semester in einem Band (natürlich profabhängig). Zur Prüfungsvorbereitung ist es relativ gut geeignet. Wem die Mathematik zum Studienbeginn sowieso schon zu abstrakt ist, dem sei eher der Beutelspacher empfohlen. Man sollte die älteren Auflagen (alles vor der 10.) meiden, da sie unübersichtlich sind.}

\item[Beutelspacher]{
		Die wohl zugänglichste Einführung in die Lineare Algebra. Der Autor verzichtet weitestgehend auf Formeln und versucht, die Ideen möglichst intuitiv und sprachlich zu vermitteln. Daher ist es für Studis, denen die mathematische Vorgehensweise im Studium zu abstrakt ist, sehr zu empfehlen. Problematisch ist allerdings, dass gerade diese Fähigkeit zum Abstrahieren eines der wichtigsten Ziele des ersten Semesters ist. Außerdem ist der Stoffumfang nicht sehr groß, wer also mal die ersten Hürden der Mathematik überwunden hat, sollte das Buch wechseln.}

\item[Bosch]{
		Ein gutes Lehrbuch, welches in Heidelberg oft begleitend zur Vorlesung verwendet wird. Es führt zuweilen relativ abstrakt in die Lineare Algebra ein und legt bereits dort Grundlagen für die weitergehenden Algebra-Vorlesungen, die man in anderen \gls{LA}-Büchern nur zum Teil findet.}
\end{description}

\vspace{-1mm}

\subsubsection{Experimentalphysik}

\begin{description}
\item[Demtröder]{
		Ein insgesamt vierbändiges Werk. Die Erklärungen sind gut und tiefgehend, dafür ist das Buch stellenweise sehr theoretisch. Beim ersten Lesen empfiehlt es sich, einige Paragraphen zu überspringen. Zum Lernen und zur Prüfungsvorbereitung ist es sehr empfehlenswert, doch muss man bei speziellen Themen und komplizierteren Formeln aufpassen, da selbst die dritte Auflage noch stark von Fehlern durchsetzt ist. Inzwischen hat sich das Buch trotzdem zu einer Art Standardwerk entwickelt, besonders in den höheren Experimentalphysikvorlesungen eignet sich das Buch bei vielen Profs sehr gut für die Vorlesungsnachbereitung.}

\item[Feynman]{
		Diese Bücher sind wunderschön zu lesen, da sie weniger aus Formeln, sondern hauptsächlich aus Erklärungen bestehen. Manche finden sie einfach genial, andere halten es nur für Gelaber. Es ist aber das einzige Buch, das wirklich versucht, Verständnis zu vermitteln und nicht nur Wissen. Zum Nachschlagen ist dieses Buch denkbar ungeeignet -- für die verzweifelten Studierenden, die gerade dabei sind, den Spaß am Studium zu verlieren, finden sich hier aber zahlreiche Passagen, die sehr anschaulich und auf eine unnachahmliche Art und Weise Physik vermitteln und so die Freude am Studium wiederbeleben können.}

\item[Tipler]{
		Das Buch enthält den Stoff der ersten drei Semester. Die Erklärungen sind sehr ausführlich, das Buch eignet sich daher hervorragend zum Lernen und zur Prüfungsvorbereitung. Es wird viel Wert auf Verständnis und Aufgaben gelegt und es ist einfach nett, im Tipler zu lesen. Allerdings werden die Themen nicht immer in der nötigen Tiefe behandelt. Es ist ein Buch zum Lernen, nicht zum Nachschlagen. Super ist der Aufgabenteil, zu dem es ein Lösungsheft mit ausführlichen Beschreibungen gibt.}
\end{description}

\vspace{-1mm}

\subsubsection{Theoretische Physik}

Bei Büchern der theoretischen Physik muss man leider immer damit rechnen, dass sie einen recht abstrakten Blick auf die Welt haben und didaktisch nicht dermaßen hervorragend sind, wie man sich das oft wünscht. Löbliche Ausnahmen sollen hier vorgestellt werden.

\begin{description}
\item[Fließbach]{
		Eines der einfacheren Bücher, allerdings auch nicht so umfangreich und auch nur bedingt für das erste Semester geeignet, da die Newton'sche Mechanik nur sehr knapp behandelt wird. Wer mit der theoretischen Physik Schwierigkeiten hat, findet hier ab dem zweiten Semester ein gutes Buch. Dazu passend gibt es auch ein Arbeitsbuch, in dem alles noch mal zusammengefasst und an Aufgaben erläutert wird.}

\item[Nolting]{
		Mehrbändige Theo-Reihe, die vor allem beim Lösen von Übungsaufgaben und bei der Klausurvorbereitung hilfreich ist. Sehr gut und nachvollziehbar strukturiert und eignet sich deshalb auch zum Wiederholen. Entspricht bis auf die Reihenfolge von der Vorgehensweise auch den Heidelberger Theorievorlesungen.}

\item[Bartelmann]{
		Diese Komposition aus Heidelberger Gefilden deckt alle für das Studium relevanten Bereiche der Theoretischen Physik in einem Band ab. Geschrieben von Dozenten, die für gute Lehre bekannt sind, wird das Buch von vielen Studierenden als gut lesbar empfunden. Ist in den Bibliotheken dermaßen vorhanden, dass sich sogar eine Kopie in die Altstadt verirrt hat.}
\end{description}

\vspace{-4mm}

\subsubsection{Mathematische Methoden}

Eine Vorlesung, die ihr zwar nicht mehr hört, doch trotzdem sind die Themen für Physikerinnen wichtig, da hier die Mathematik auf Gebrauchsniveau gehievt wird.

\begin{description}[style=unboxed]
\item[Boas: Mathematical Methods in the Physical Science]{
		Hier wird die Mathematik so gebracht, wie sie in der Physik gebraucht wird. Es ist wohl das beste Buch zu diesem Thema. In Boas werden zahlreiche für Physikerinnen wichtige Vorgehensweisen anschaulich erklärt und in vielen Beispielen ausführlich vorgerechnet. Außerdem enthält das Buch Aufgaben mit Lösungen. Vor dem Englisch braucht ihr keine Angst zu haben, denn mathematical english ist immer sehr viel einfacher als normal english. Sehr zu empfehlen, sowohl als Nachschlagewerk als auch, um verschiedene generelle Schwierigkeiten zu beheben.}

\item[Lang, Pucker: Mathematische Methoden in der Physik]{
		Dieses Buch wird zur Zeit von den Dozentinnen empfohlen und ist in etwa äquivalent zur Boas -- nur auf Deutsch und etwas günstiger in der Anschaffung. Teilweise etwas weniger ausführlich.}

\item[Otto: Rechenmethoden für Studierende d. Physik im ersten Jahr]{
		In diesem Buch ist die Mathe, die man in den ersten beiden Semestern eines Physikstudiums braucht, anschaulich und ausführlich erklärt. Dabei wird bewusst auf mathematische Beweise verzichtet und mehr auf die physikalische Interpretation eingegangen. Sehr gut geeignet für Studis, die sich am Anfang in Theo ein wenig von der Mathe überrumpelt fühlen.}
\end{description}

\subsubsection{Informatik allgemein}
In den Informatik-Vorlesungen wird allgemein bewusst sehr wenig Literatur verwendet, sondern hauptsächlich die Vorlesungsfolien und vor allem Online-Dokumentationen, insbesondere zu den Programmiersprachen. Meistens werden gute Wikipedia-Artikel, YouTube-Tutorials, Stack Overflow-Posts und Google-Books bzw. Google-Scholar-Links näher angeschaut. Wir haben euch zur Vorbereitung auf das erste Semester ein Vorkursskript geschrieben\footnote{\url{https://mathphys.info/vorkurs/programmier/vorkurs.pdf}}.

\begin{figure}[b]
\centering
\includegraphics[width=\linewidth]{bilder/random_number.png}

\end{figure}

\subsubsection{Praktische Informatik und Programmierkurs}
\begin{description}[style=unboxed]
\item[Bastian: Einführung in die praktische Informatik]{
	ist ein Vorlesungs-Skript, was ihr euch auch bei uns im Fachschaftsraum kostenlos in gedruckter Form abholen könnt. In der Praktischen Informatik wird meistens darauf verwiesen oder eine Abwandlung davon verwendet. Hier wird Schritt für Schritt die Programmiersprache C++ an zahlreichen praktischen Beispielen aus dem Themenbereich des wissenschaftlichen Rechnens erklärt und zeigt somit im Ansatz die Schnittmenge zwischen den Disziplinen Mathematik und Informatik. Das Skript von Prof.~Bastian ist online verfügbar\footnote{\url{https://conan.iwr.uni-heidelberg.de/old-site/teaching/info1_ws2014/info1-skript.pdf}}.}

\item[Stroustroup: The C++ Programming Language]{
	ist das Referenzwerk für die Programmiersprache C++. Ähnlich wie in einer Online-Dokumentation werden hier die Syntax, verschiedene Befehle und Funktionen anhand von Minimalbeispielen näher erklärt.}

\item[Knuth: The Art of Computer Programming]{ist eine große Bandreihe vom berühmten \TeX-Erfinder. Das erste Band beschreibt mit mathematischer Strenge die Performanz unterschiedlicher Datenstrukturen. Dabei spielt der Begriff der Laufzeitkomplexität eine große Rolle.}

\end{description}

% Es werden nur noch Bücher für Erstsemestervorlesungen empfohlen, da wie im Allgemeinen Informatik Teil erläutert, ohnehin mehr Folien und online Materialien verwendet werden als Lehrbücher.
% \subsubsection{Algorithmen und Datenstrukturen}
% \begin{description}[style=unboxed]

% \item[Sedgewick: Algorithms]{
% 	ist eine sehr einsteigerfreundliche Lektüre, in der verschiedene Algorithmen eingeführt und anhand einer Java-Implementation näher erläutert werden. Auf die eigene Implementation verschiedener Datenstrukturen wird hier leider nicht eingegangen, sondern es wird auf die in den Bibliotheksfunktionen bereits vorhandenen Strukturen zurückgegriffen. Die Programmbeispiele in Java sind sehr simpel gehalten und lassen sich somit relativ einfach in C++ oder in Python umformulieren. Dieses Buch kann vorweg als Begleitlektüre für die Vorlesung Praktische Informatik genommen werden und aufgrund der Java-Beispiele bereitet es die Leserin ebenfalls auf die Vorlesung Einführung in Software Engineering vor.} 

% \item[Cormen: Algorithms]{
% 	ist sozusagen die Algorithmen-Bibel. Jede Art von Sortier-, Textbearbeitungs-, Graphen- und Hashing-Algorithmus wird hier in Pseudocode und mit bildlich dargestellten Beispielen näher erklärt und mathematisch auf ihre Korrektheit geprüft. Es ist nicht nur eine sehr einsteigerfreundliche Lektüre sondern auch ein Nachschlagewerk für erfahrene Mathematikerinnen und Informatikerinnen.}
% \end{description}

\subsubsection{Technische Informatik}
\begin{description}[style=unboxed]
\item[Clements: The Principles of Computer Hardware]{
	ist ein übersichtlich gegliedertes Buch, welches die Vorlesungsinhalte im Großen und Ganzen abdeckt. Insbesondere wird auf die boolesche Algebra, unterschiedliche Schaltwerke, Rechnen in anderen Zahlensystemen und auf die Computerarchitektur eingegangen.}
\end{description}

% Es werden nur noch Bücher für Erstsemestervorlesungen empfohlen, da wie im Allgemeinen Informatik Teil erläutert, ohnehin mehr Folien und online Materialien verwendet werden als Lehrbücher.
% \subsubsection{Betriebssysteme \& Netzwerke}
% \begin{description}[style=unboxed]
% \item[Silberschatz: Operating Systems Concepts]{
% 	ist ein auffälliges Buch mit Dinosaurier auf dem Buchcover und enthält die ersten zwei Drittel der Vorlesung, und zwar den Betriebssystemeteil. Im Großen und Ganzen werden hier der Umgang mit Prozessen in einem Betriebssystem und die verschiedenen Dateisysteme erklärt. Die zuvor erklärte Theorie wird dann an realen Betriebssystemen wie Windows oder Linux dargestellt.}

% \item[Tanenbaum: Computer Networks]{
% 	ist ebenfalls ein Buch mit einem extravaganten Cover und deckt das letzte Drittel - den Netzwerketeil - der Vorlesung ab. Das Buch ist nach den Schichten der Internetprotokollfamilie unterteilt. Hier werden, wenn auch sehr textlastig, die verschiedene Techniken des Internets erklärt. Leider werden wichtige Algorithmen, die am Ende der Vorlesung drankommen, nur kurz oder gar nicht erwähnt.}
% \end{description}

% \subsubsection{Datenbanken}
% \begin{description}
% \item[Kemper: Datenbanksysteme]{
% 	ist eines der wenigen deutschsprachigen Bücher, die in der Informatiklehre relevant sind. Obwohl dieses Buch relativ dick ist, ist der Inhalt verständlich und kompakt beschrieben. Die typischen Themen wie das Entity-Relationship-Modell, SQL, B-Bäume und Anfrageoptimierung werden im vollen Umfang abgedeckt. Für fleißige Menschen gibt es auch ein Übungsbuch dazu.}
% \end{description}

% \subsubsection{Theoretische Informatik}
% \begin{description}[style=unboxed]

% %Ist das Buch noch relevant?
% \item[Vossen, Witt: Grundkurs Theoretische Informatik]{
% 	enthält jedes Thema, das in der Vorlesung Theoretische Informatik behandelt wird, außer die Registermaschine. Hier sollte man jedoch etwas aufpassen, denn die ersten zwei Drittel der Vorlesung (Berechenbarkeit und Komplexität) werden erst ab dem achten Kapitel des Buches beschrieben, und das letzte Drittel der Vorlesung (Formale Sprachen und Automatentheorie) wird ab dem ersten Kapitel des Buches beschrieben. Hier werden zwar, ähnlich wie in einer Mathevorlesung, die Themen relativ knapp und beweislastig beschrieben, aber regelmäßig mit Rechenbeispielen näher erläutert. Jedoch unterscheiden sich leider die Notationen von denen aus der Vorlesung.}

% \item[Cohen: Introduction to Computer Theory]{ 
% 	ist ein Buch, was nicht nur die Theoretische Informatik abdeckt, sondern auch die nachfolgenden Vorlesungen, wie Formale Sprachen \& Automatentheorie und Berechenbarkeit \& Komplexität. Dieses Buch zeichnet sich auch durch viele Beweis- und Rechenbeispiele aus und ist auch nota\-tionsmäßig sehr nahe an der Vorlesung.}
% \end{description}

% \subsubsection{Software-Engineering}
% \begin{description}[style=unboxed]

% %Ist das Buch relevant, wichtiger wäre vielleicht das Buch von Ludewig & Lichter?
% \item[Sommerville: Software Engineering]{
% 	ist ein sehr text-, tabellen- und diagrammlastiges Buch, was aber auch typisch für Software Engineering ist. Es deckt nicht komplett den Stoff der Vorlesung Software Engineering ab, aber schneidet fast jedes Thema zu einem guten Teil an. Dabei werden wie in der Vorlesung verschiedene ähnlich klingende Begriffe definiert, aber dafür klar auseinander gehalten, verschiedene UML-Diagrammtypen vorgestellt, das Testen von Software erläutert, Modelle zur Planung eines Software-Projektes beschrieben und der Umgang mit einer Kundin erklärt.}
% \end{description}

\subsubsection{Nachschlagewerke}

Als eine kommentierte Formelsammlung können folgende Bücher dienen:

\begin{description}
\item[Bronstein/Semendjajew]{
		Eines der Bücher, die man als Physikerin von jeder Prof empfohlen bekommt -- zu Recht. Da das Buch nur die Formeln und Beispiele enthält und keine Beweise, ist es für Mathematikerinnen nicht so interessant, trotzdem aber nützlich, wenn man irgendetwas berechnen muss. Der Bronstein hat sich bei Physikerinnen zu einer Art Bibel entwickelt (jede hat es, jede benutzt es) da hierin jede Menge Integrale, Taylorreihenentwicklungen, usw. aufgelistet werden.}

\item[Stöcker: Taschenbuch der Physik]{
		Experimentell orientiertes, sehr kompaktes Nachschlagewerk mit teilweise sehr guten und einprägsamen Erklärungen für die stabile Studi- Jackentasche. Auch praktisch zum Lernen in Bus und Straßenbahn.}
\end{description}

\vfill
\eject