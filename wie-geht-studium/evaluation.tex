% !TEX ROOT = ../ersti.tex
\section{Lehrevaluation}
\label{eval}

Gegen Ende jedes Semesters hast du die Möglichkeit, deinen Professorinnen und Tutorinnen anonym Feedback zu geben. Hierfür bekommst du für jede Vorlesung, jede Übungsgruppe, jedes Praktikum, jedes Seminar und jede sonstige Veranstaltung, die du in diesem Semester belegt hast, eine E-Mail von \textit{heiquality} mit Zugangslink zu einem Onlineportal. In diesem Onlineportal kannst du dann all deine Veranstaltungen und Dozentinnen anonym bewerten.

\begin{figure}[b]
    \begin{center}
        \includegraphics{bilder/eval_1.png}\\
        \includegraphics{bilder/eval_2.png}\\
        \includegraphics{bilder/eval_3.png}\\
        \includegraphics{bilder/eval_4.png}\\
        \includegraphics{bilder/eval_5.png}\\
    \end{center}
\end{figure}

Es kann passieren, dass du einen Link zur Evaluation genau dann bekommst, wenn du dich eigentlich gerade mitten im Prüfungsstress befindest, oder aus anderen Gründen beschäftigt bist. Trotzdem ist es sehr wichtig, dass du dir die Zeit nimmst, um die Fragebögen auszufüllen, denn nur durch solches Feedback ist es möglich, dass die Veranstaltungen kontinuierlich verbessert werden können.

Die Evaluation ist aus zweierlei Gründen essentiell.
Zum einen ist es für die Professorinnen und Tutorinnen wichtig, Feedback zu bekommen, um zu wissen, was sie schon gut machen, und was sie in Zukunft an ihrem Unterricht verbessern können. Du wirst es vielleicht nicht glauben, aber der Großteil der Professorinnen und Tutorinnen wird dir sehr dankbar sein, wenn du positive wie negative Aspekte ihres Unterrichts in der Evaluation angibst; am besten nutzt du dafür das jeweils passende Freitextfeld.

Zum anderen bekommen die Studienkommissionen Einsicht in die Evaluationsergebnisse, sodass besonders schlechte Veranstaltungen mit den Dozentinnen nachbesprochen werden können, um das Lehrniveau zu steigern. In der Physik wird auf Basis der Evaluationen auch jedes Semester ein Lehrpreis für besonders gute Veranstaltungen vergeben\footnote{\url{https://mathphys.info/w/evaluation-und-lehrpreis/}}.

Darüber hinaus kannst du dich sehr gerne auch an die Fachschaft wenden, um Kritik an Lehrveranstaltungen zu äußern.
