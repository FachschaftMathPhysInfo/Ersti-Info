\section{Überblick im hochschulpolitischen Dschungel}
\label{hopo}
\marginpar{
    \centering{
        \vspace{2mm}
        \includegraphics[width=3cm]{bilder/duty_calls.png}\\\vspace{5mm}
    }
}

Hochschulpolitik findet auf vielen Ebenen statt. Gesetze, die Studierende betreffen, werden in Landes"= und Bundesparlamenten verabschiedet, in den Gremien der universitären Selbstverwaltung werden Studiengänge konzipiert, die unabhängige Studierendenvertretung verteilt Flugblätter und im Rektorat laufen nicht nur viele richtungsweisende Entscheidungen ab, sondern auch die ein oder andere studentische Besetzung gegen untragbare Zustände.
%Thematisch umfasst die Hochschulpolitik unter anderem den Bau, die Finanzierung, den rechtlichen Status der Hochschulen, den Rahmen für Forschung und Lehre, aber auch die soziale und politische Stellung der Studierenden.
Im folgenden wollen wir versuchen, ein wenig Überblick ins Chaos der Aufgaben und Kompetenzen zu bringen.

\subsection{Die Ebene der "`Großen Politik"' -- Gesetzlicher Rahmen}
Grundsätzlich ist Hochschulrecht Landesrecht. Der Bund gab lange Zeit im Hochschulrahmengesetz (HRG) Vorgaben, die im Landesrecht ausgestaltet
wurden. Doch einige Bundesländer, darunter BaWü, wollten die darin vorgesehene Einrichtung einer Verfassten Studierendenschaft verhindern (zur Verfassten Studierendenschaft mehr im entsprechenden Kapitel). Im Kompetenzstreit mit dem Bund bekamen sie 2005 vor dem Bundesverfassungsgericht Recht. Eine weitere Erörterung dieses Themas ist im UNiMUT zu finden\footnote{\url{http://unimut.fsk.uni-heidelberg.de/2unimut/aktuell/1106731377}}.
%Am 26. Januar 2005 wurde jedoch vor dem Bundesverfassungsgericht in einem Kompetenzstreit zwischen Bund und Ländern die 6. Novelle des HRG gekippt. Diese verlangte  und schrieb die Einführung von Verfassten Studierendenschaften zwingend vor, was zur Klage einiger Bundesländer, darunter BaWü, geführt hatte. (Die Verfassten Studierendenschaften werden im Laufe dieses Artikels noch besprochen.) 

Seit der Förderalismusreform im Jahr 2006 ist die Gesetzgebungskompetenz vollständig auf die Länder übergegangen. Der Bund ist nur indirekt z.B. über das BAföG (Bundesausbildungsförderungsgesetz) oder Forschungsförderungsprojekte (z.B. die Exzellenzinitiative) eingebunden. Amtierende Ministerin auf Bundesebene ist \wissenschaftsministerbund .

Wissenschaftsministerin in Baden"=Württemberg ist zur Zeit \wissenschaftsministerbawue .
Am 9. Dezember 2005 verabschiedete der Landtag von Baden"=Württemberg die Version des Landeshochschulgesetz (LHG), die bis heute die Struktur der Hochschulen bestimmt. Diese verlagert insbesondere Kompetenzen weg von Gremien hin zu Einzelpersonen. So wurden dem Senat, dem traditionellen akademischen Entscheidungsgremium der Universität (mit Mitgliedern aller Gruppen und Fakultäten) zahlreiche Kompetenzen genommen und auf das Rektorat übertragen. Genaueres zu den Gremien der Universität und ihrer Verflechtung erfahrt ihr weiter unten. Die Möglichkeiten des Wissenschaftsministeriums, auf die personelle Besetzung des Rektorats und die Zusammensetzung des Universitätsrates einzuwirken, wurden gesteigert.

Am 21. Dezember 2011 und am 27. Juni 2012 wurden mit der Abschaffung der allgemeinen Studiengebühren und der Einführung der Verfassten Studierendenschaft durch die grün-rote Landesregierung endlich zwei größere Veränderungen am LHG vorgenommen, die seit Jahren von Studierenden eingefordert wurden. Eine grundlegende Novellierung des LHG, die auch die Struktur der Hochschule angeht, ist für 2014 angekündigt.

Durch das Budgetrecht der Parlamente, die Strukturvorgaben der Kultusministerkonferenz zu den Bachelor"= und Masterstudiengängen, das LHG sowie das Bundesverfassungsgericht sind insbesondere folgende Bereiche festgelegt:
\begin{itemize}
\item Finanzielle Ausstattung der Hochschulen: Reicht das bereitgestellte Geld den Unis für gute Lehre? Gibt es Gebühren?
\item Struktur der Studiengänge: Welche Abschlüsse gibt es? Gibt es Regelstudienzeiten? Wie lang sind diese? Müssen Studiengänge modularisiert sein?
\item Struktur der Hochschulen: Welche universitären Instanzen haben welche Entscheidungsbefugnisse?
\item Unmöglichkeit studentischer Selbstbestimmung: Gemäß BVerfG-Urteil von 1973 müssen ProfessorInnen in Gremien die die Lehre betreffen eine maßgebliche (= relative), in solchen die die Forschung betreffen eine entscheidende (= absolute) Mehrheit haben.
\end{itemize}

Einfluss haben Studis hierauf nur auf der Straße, sowie über die Landesweite Studierendenvertretung.


\subsection{Studentisches Leben in der Stadt -- Studentenwerk und kommunale Arbeitskreise}

\paragraph{Studentenwerk Heidelberg -- Service statt Selbstverwaltung}

Das Studentenwerk übernimmt, neben der kulinarischen Versorgung vor Hunger verzweifelter Studis in der Mensa, viele soziale Aufgaben, die vor der Auflösung der Verfassten Studierendenschaften (siehe unten) von den Studierenden selbst erledigt wurden. Dazu zählen die Wohnheimsverwaltung, die BAföG-Beratung, die Beratung ausländischer Studierender, ein vielseitiges Kulturprogramm und viel mehr. Als Anstalt öffentlichen Rechts arbeitet es nicht gewinnorientiert, aber zunehmend unternehmerisch. Der Verwaltungsrat, in dem jeweils einE Studi von Uni, PH und Uni Heilbronn im Zweifelsfall chronisch überstimmt werden, kontrolliert die Geschäftsführung. Hier entscheiden sich insbesondere Wohnheimsmieten, Wohnheimsbauten und Mensapreise.

An das Studentenwerk zahlt ihr einen Beitrag von \EUR{\studentenwerksbeitrag}, der in den \EUR{\beitragssumme}, die ihr jedes Semester an die Uni überweisen müsst, enthalten ist. Damit wird dessen Angebot sowie das Semesterticket mitfinanziert.

\paragraph{Kommunale Arbeitskreise -- Auch Studis sind BürgerInnen}

Insbesondere in der kommunalen Verkehrspolitik von Semesterticket über Fahrradwege bis zum Straßenbahnneubau im Neuenheimer Feld betrifft Kommunalpolitik die Studis. Hier arbeitet das Kommunalreferat der \gls{FSK} mit der Stadt zusammen. Auch der kommende Abzug der US"=Truppen 2015, der riesige Flächen zurücklässt, ist aktuell ein brennendes Thema. Das Studentenwerk erwägt, ehemalige Kasernen oder Offizierswohnhäuser in Wohnheime umzufunktionieren. Außerdem gibt es diverse Initiativen für selbstverwaltete Kulturprojekte, alternatives Wohnen, ein neues autonomes Zentrum\footnote{\url{http://anarres.blogsport.de/}} und viel mehr, die sich in ganz Heidelberg vernetzen\footnote{\url{http://hdvernetzt.wordpress.com/}}.


\subsection{Die akademische Selbstverwaltung -- Gremien und Kommissionen}

Die Universität Heidelberg ist eine Lehr"= und Forschungseinrichtung mit über 400 ProfessorInnen, circa 30\,000 Studierenden und vielen sonstigen
MitarbeiterInnen. Der Etat der Universität beträgt um die 550 Millionen Euro. Die Aufgabe der Universität, die „Pflege und Entwicklung der
Wissenschaften und der Künste durch Forschung, Lehre und Studium“ (§2 % Quelle: http://www.umwelt-online.de/recht/allgemei/laender/bw/hschg01.htm
LHG), stellt hohe finanzielle und organisatorische Ansprüche.

Auf zentraler und auf dezentraler Ebene gibt werden Entscheidungen getroffen und Maßnahmen durchgeführt, um diese Aufgaben zu bewältigen. Wie viel Kompetenzen dabei Gremien unter Beteiligung  von Profs, akademischem Mittelbau, Studis und Verwaltung/Technik zukommen, und wie viel Macht bei Vorstandsstrukturen wie dem Rektorat oder den Dekanaten liegt, ist ständig Gegenstand politischer Debatten. Im folgenden stellen wir euch die Strukturen der sogenannten akademischen Selbstverwaltung, welche die Angelegenheiten der gesamten Uni regelt, vor.

\paragraph{Universitätsweite Struktur}

Entscheidungen für die gesamte Hochschule treffen die zentralen Gremien, Rektorat, Senat und Universitätsrat.

Der Senat beschließt über
universitätsweite Belange wie Verlegung der Semesterzeiten, die
Immatrikulationsordnung und generell über grundlegende Fragen, welche die
Gesamtuniversität betreffen. Der Senat bestätigt Vorschläge der einzelnen
Fakultäten über die Berufung neuer Professoren, genehmigt Lehrpläne und
beschließt über die Einrichtung oder Aufhebung von Studiengängen. Dem
Senat gehören außer dem Rektorat alle DekanInnen (das sind die
Vorsitzenden der Fakultäten), die Frauenbeauftragte, 8 weitere ProfessorInnen, 4 Studierende, 4 VertreterInnen aus dem akademischen Mittelbau und 4 Sonstige an. Das macht ein Verhältnis von 28 Profs zu 12 anderen, davon 4 Studis.

Besondere Themen wie Umweltfragen oder
Prüfungsangelegenheiten werden in beratenden Ausschüssen des Senats
vorbereitet, in denen alle Gruppen vertreten sind, die ProfessorInnen
natürlich wieder mit absoluter Mehrheit.

Den Universitätsrat gibt es erst seit dem Wintersemester 2000/2001. Er berät über die strukturelle
Ausrichtung der Universität und hat das letzte Wort in
Finanzangelegenheiten. Er besitzt damit wesentlichen Einfluss auf die
künftige Entwicklung der Universität. 6 der 11 Mitglieder sind
Universitätsexterne aus Politik, Kultur und Wirtschaft, 5 Mitglieder
kommen aus der Universität. Alle Mitglieder werden vom
Wissenschaftsministerium benannt.

Das Rektorat besteht aus dem Rektor \rektor, 4 ProrektorInnen und dem bzw. der LeiterIn der Verwaltung: dem bzw. der KanzlerIn. Neben seinen Verwaltungsaufgaben in der Uni vertritt das Rektorat sie auch nach außen, z.B. gegenüber dem Land. Das Rektorat residiert in der alten Universität, Grabengasse 1.
Die Universitätsverwaltung ist in der Seminarstr. 2 angesiedelt, wo bis zur polizeilichen Räumung 1978 das selbstverwaltete Studierendenwohnheim "Collegium Academicum" bestand.


\paragraph{Die Fakultäten als dezentrale übergeordnete Struktur}

Die speziellen Fragen eines Fachbereichs werden in der Fakultät, der
„organisatorischen Grundeinheit der Universität“, die „gleiche oder
verwandte Fachbereiche zusammenfasst“ (LHG), (vor)entschieden. Die
Universität ist in 12 Fakultäten gegliedert, zum Beispiel die Fakultät für
Mathematik und Informatik oder die Fakultät für Physik und Astronomie.

Während man an einigen Fakultäten nur in wenigen Studiengängen studieren
kann, gibt es andere Fakultäten, wie z.B. die Neuphilologische Fakultät,
an denen viele Fächer (Germanistik, Anglistik, Romanistik etc.) studiert
werden können. An den Fakultäten sind Institute und Seminare mit verschiedenen inhaltlichen Ausrichtungen angesiedelt, z.B. an der Fakultät für Physik und Astronomie das Institut für Theoretische Physik (ITP), das Institut für Umweltphysik (IUP) u.v.m. Jedes Institut wird von einem/einer InstitutsdirektorIn geleitet.

Geleitet wird die Fakultät von einem/einer DekanIn und dem Dekanat, das die laufenden Geschäfte erledigt. DekanInnen werden für vier Jahre vom Fakultätsrat gewählt. Zur Zeit ist Dekan der Mathematik Prof. \dekanmathe{} und in der Physik Prof. \dekanphysik. Zusammen mit der/dem
StudiendekanIn (s.u.) bilden DekanIn und ProdekanIn den Fakultätsvorstand. Zu finden ist das Dekanat der Mathe im \Gls{Mathematikon}, Zimmer 01.101 und das der Physik in der Albert Ueberle Str. 3-5.

Die Mitglieder der Fakultäten wählen getrennt nach Statusgruppe (Studi, Verwaltung/Technik, akad. Mittelbau, Prof) den großen Fakultätsrat, das oberste Gremium der Fakultät. Er ist zuständig für alle Fragen der Lehre und der Forschung. Ihm gehören alle Profs, 8 (bzw. 6) Studis, 5 (bzw. 4) Angehörige aus dem Mittelbau und 1 Mitglied aus der Gruppe der MitarbeiterInnen aus Administration und Technik an.

Der Fakultätsrat bildet Ausschüsse mit vergleichbarer Zusammensetzung, die sich um bestimmte Bereiche kümmern; besonders wichtig
sind zum Beispiel die Prüfungsausschüsse oder die Studienkommissionen. Die Mitglieder der Ausschüsse müssen nicht immer Mitglieder des Fakultätsrats
sein.


\paragraph{Die Studienkommissionen -- Hier geht es um die Lehre}

Seit 1995 gibt es die äußerst sinnvolle Studienkommission, die den/die StudiendenkanIn berät. StudiendekanIn und Studienkommission sollen gemeinsam zur Verbesserung der Lehre beitragen. Die Studienkommission besteht aus StudiendekanIn, drei weiteren ProfessorInnen, zwei VertreterInnen des Mittelbaus sowie vier Studierenden. Hier werden Fragen der Studiengangsgestaltung und der Lehrqualität diskutiert, wobei die Studierenden ausnahmsweise nicht maßlos unterrepräsentiert sind. Hauptaufgaben der Kommission sind:
\begin{itemize}
    \addtolength{\itemsep}{-0.7\baselineskip}
    \item Empfehlungen zur Weiterentwicklung von Gegenständen und Formen des Studiums
    \item „Verfahren zur Bewertung und Verbesserung der Qualität der Lehre unter
          Einbeziehung studentischer Veranstaltungskritik“ zu entwickeln
    \item In regelmäßigen Abständen einen Lehrbericht zu verfassen
\end{itemize}

Die Umsetzung von Empfehlungen und die Wahrnehmung laufender Aufgaben obliegt dem/der StudiendekanIn. Die Kommission ist freilich nur
beratend und die Durchsetzung ihrer Beschlüsse und Ideen ist vom guten Willen des Fakultätsrats abhängig.

\paragraph{Die Fachräte als unterstützende dezentrale Struktur}

In den Fakultäten, wo viele äußerst verschiedene Fächer zu einer Fakultät zusammengefasst sind, gibt es seit 2011 die Fachräte. Zuvor mussten die Fakultätsräte ganz alleine z.B. an der Philosophischen Fakultät über Ägyptologie, Geschichte, Philosophie, Ostasienwissenschaften und viel mehr Fächer entscheiden. Es gab auch nur eine Studienkommission pro Fakultät. Das hat infolge mangelder Kenntnis der Zustände in den Fächern, die nicht im Fakultätsrat vertreten waren, dauernd zu unsinnigen Prüfungsordnungen und dergleichen geführt. Nun ist durch die zusätzliche Ebene der Fachräte endlich die Möglichkeit zur sinnvollen Konzeption der Lehre und zu erhöhter studentischer Mitwirkung in den großen Fakultäten gegeben.

%Wenn das Ersti-Info zu fett wird, kann das raus...
\subsection{Historischer Exkurs: Es geht noch schlimmer -- Von der Ordinarienuni zur Gruppenuni}

%Das war der Higgs-Comic...
%\sidebar{
%    \centering
%    \includegraphics[width=3cm]{bilder/dear_CERN_1.png}\\\vspace{14mm}
%    \includegraphics[width=3cm]{bilder/dear_CERN_2.png}\\\vspace{14mm}
%    \includegraphics[width=3cm]{bilder/dear_CERN_3.png}\\\vspace{14mm}
%    \includegraphics[width=3cm]{bilder/dear_CERN_4.png}\\\vspace{14mm}
%    \includegraphics[width=3cm]{bilder/dear_CERN_5.png}\\\vspace{14mm}
%    \includegraphics[width=3cm]{bilder/dear_CERN_6.png}
%}
Wie euch sicher aufgefallen ist, steht es um das Stimmrecht der Studis in den Gremien meist mies. Das ist quasi historisch gewachsen\dots

Bis 1969 hatten die LehrstuhlinhaberInnen ("`Ordinarien"') das Sagen an den Universitäten. Ordinarien hatten praktisch die alleinige Entscheidungsbefugnis
für ihren Lehr"= und Forschungsbereich. Sämtliche Gremien setzten sich alleine aus Ordinarien zusammen.

1969 wurde diese Ordinarienuniversität im Zuge der Forderungen nach Demokratisierung und Mitbestimmung abgeschafft und die Gruppenuniversität eingeführt. Die Mitglieder der Universität wurden in Gruppen eingeteilt; die ProfessorInnen, die wissenschaftlichen MitarbeiterInnen (=\ akademischer Mittelbau), die
StudentInnen sowie die MitarbeiterInnen aus Administration und Technik (=\ Sonstige). Jeder dieser „Stände“ wählt bei Uniwahlen eine bestimmte Anzahl VertreterInnen in die Gremien der Uni. Die ProfessorInnen stellen zusätzlich eine gewisse Anzahl von Mitgliedern kraft Amtes. Für eine kurze Phase wählten alle Gruppen außer den Sonstigen (HausmeisterInnen, SekretärInnen, \dots) gleich viele VertreterInnen („Drittelparität“).

1973 stellte das Bundesverfassungsgericht (mit 6:2) aber fest, dass aufgrund der grundgesetzlich garantierten Freiheit von Forschung und Lehre (Art. 5 GG) die Gruppe der ProfessorInnen in allen Gremien eine maßgebende, bzw. in bestimmten Fragen eine ausschlaggebende Mehrheit haben müsse. Aufgrund dieses Urteils sind alle Gremien mit nicht nur beratender, sondern entscheidender Funktion so zusammengesetzt, dass ProfessorInnen mindestens so viele Sitze haben wie alle anderen Gruppen zusammen. In manchen Gremien zählen auch einfach die professoralen Stimmen mehrfach. Hierzu werden „gezinkte“ Stimmzettel ausgegeben: auf dem Stimmzettel wird markiert, ob mensch Prof oder nicht-Prof ist. So sieht Hochschuldemokratie aus.

Die sehr einseitige Auffassung von Lehre, die das BVerfG"=Urteil zu Tage legt, lässt sich zwar inhaltlich sehr leicht anfechten, aber juristisch kaum: Lediglich eine Landesregierung, die Bundesregierung oder eine Klage vor dem Europäischen Gerichtshof kann gegen das Urteil vorgehen.
