\section{Überblick}
\label{hopo}
\marginpar{
    \centering{
        \vspace{2mm}
        \includegraphics[width=3cm]{bilder/duty_calls.png}\\\vspace{5mm}
    }
}

Die Hochschulpolitik befasst sich mit allen politischen Vorgängen
bezüglich der Hochschulen in Deutschland. Dies beinhaltet Abläufe in den
Landes"= und Bundesparlamenten, den Gremien der universitären
Selbstverwaltung und der Öffentlichkeit. Thematisch umfasst die
Hochschulpolitik unter anderem den Bau, die Finanzierung, den rechtlichen
Status der Hochschulen, den Rahmen für Forschung und Lehre, aber auch die
soziale Stellung der Studierenden.

So wird über Studien"= und Prüfungsordnung, Organisation, Gliederung und
Ausrichtung der Hochschulen weitestgehend vor Ort entschieden, während
andere Entscheidungen die Kompetenz der Universität übersteigen. Sehr
deutlich wird dies bei den Regelungen über das BAföG
(Bundes"=Ausbildungsförderungs"=Gesetz), Wohnheimmieten oder kommunale
Verkehrspolitik (Fahrradwege, Semesterticket, \dots).

Die studentischen Belange werden bei der Entscheidungsfindung leider
häufig nur unzureichend berücksichtigt. Eine Ursache hierfür ist durch das
Hochschulrecht gegeben welches den Studierenden nur eine bescheidene
Mitwirkung in den offiziellen Gremien der Universität zugesteht. Eine
Mitarbeit in anderen Gremien der Bildungslandschaft ist überhaupt nicht
vorgesehen. Natürlich gibt es Überschneidungen zwischen den Interessen der
Studierenden und denen des Landes bzw. des Bundes als Träger und
Finanziers der Hochschulen im Großen, sowie zwischen den Studierenden und
den Professoren vor Ort. Die Entscheidungen, die in den letzten Jahren im
Bereich der Hochschule getroffen wurde, lassen allerdings deutlich
erkennen, dass diese Interessen im Vergleich zum Sparwillen der
öffentlichen Kassen nur geringe Priorität besaßen.

Ein weiterer Grund für die vergleichsweise geringe Beachtung der
Studierenden in Entscheidungsfindungsprozessen ist deren Situation und
sozialer Status. Zum einen sind die Studierenden keine homogene Gruppe:
Die eine finanziert sich ihr Studium selbst während der andere in Papis
Brieftasche ausreichend Ersparnisse vorfindet und wieder andere werden von
einer Stiftung gefördert. Zum anderen ist die Studienzeit gewöhnlich nur
ein auf politischer Skala kurzer Lebensabschnitt, der nach wenigen Jahren
wieder beendet ist. Die Wirkung von heute beschlossenen, politischen
Entscheidungen betrifft in den meisten Fällen erst die Studierenden von
morgen.

Umso wichtiger ist die studentische Mitwirkung aus eigenem Engagement. Der
folgende Artikel stellt den institutionellen Rahmen dar, in dem sich
politische Arbeit von Studierenden an der Universität und darüber hinaus
bewegt.

\subsection{Gesetzlicher Rahmen}
Grundsätzlich ist Hochschulrecht Landesrecht. Der Bund gab lange Zeit im
Hochschulrahmengesetz (HRG) Vorgaben, die im Landesrecht ausgestaltet
wurden. Am 26. Januar 2006 wurde vor dem Bundesverfassungsgericht (BVG)
ein Kompetenzstreit über die Zuständigkeit in der Hochschulpolitik
zwischen dem Bund und den Bundesländern Baden"=Württemberg, Bayern, dem
Saarland, Sachsen, Sachsen"=Anhalt und Hamburg geklärt. Der Bundestag hatte
in der 6. Novelle des Hochschulrahmengesetzes ein „in der Regel
gebührenfreies Erststudium“ verlangt und die Einführung von Verfassten
Studierendenschaften zwingend vorgeschrieben. (Die Verfassten
Studierendenschaften werden im Laufe dieses Artikels noch besprochen.) Das
Verfassungsgericht stellte fest, dass der Bund in diesen Fragen solange
nicht zuständig ist, bis die Notwendigkeit einer einheitlichen Regelung
für die Herstellung gleicher Lebensverhältnisse in den verschiedenen
Ländern nachgewiesen wurde. Die Fragestellung, ob eventuell einzuführende
Studiengebühren der Verfassung entsprechen, wurde dabei nicht verhandelt.
Eine weitere Erörterung dieses Themas ist im Unimut zu finden\footnote{\url{http://unimut.fsk.uni-heidelberg.de/2unimut/aktuell/1106731377}}.
Seit der Förderalismusreform im Jahr 2006 ist die Gesetzgebungskompetenz
vollständig auf die Länder übergegangen. Der Bund ist lediglich indirekt
z.B. über das BAföG (Bundesausbildungsförderungsgesetz) oder
Forschungsförderungsprojekte (z.B. die Exzellenzinitiative) eingebunden.
Amtierende Ministerin auf Bundesebene ist \wissenschaftsministerbund.


Der Wissenschaftsminister in Baden"=Württemberg ist zur Zeit \wissenschaftsministerbawue .
Das wesentliche Landesgesetz für die Form der
baden"=württembergischen Hochschulen wurde durch das seit 1. Januar 2000
geltende Universitätsgesetz (UG) bestimmt. Am 9. Dezember 2004
verabschiedete der Landtag von Baden"=Württemberg ein neues
Landeshochschulgesetz (LHG), das insbesondere Änderungen in der
Kompetenzverteilung zwischen den Gremien der universitären
Selbstverwaltung vorsieht. Ziel scheint dabei zu sein, Kompetenzen weg von
Gremien hin zu Einzelpersonen zu verlagern. So wurden dem Senat, dem
traditionellen akademischen Entscheidungsgremium der Universität (mit
Mitgliedern aller Gruppen und Fakultäten) zahlreiche Kompetenzen genommen
und auf das Rektorat übertragen. Genaueres zu den Gremien der Universität
und ihrer Verflechtung erfahrt ihr weiter unten. Die Möglichkeiten des
Wissenschaftsministeriums, auf die Besetzung der Position des Rektors und
die Zusammensetzung des Aufsichtsrates einzuwirken, wurden gesteigert.

\section{Aus der jüngsten Geschichte der Mitbestimmung an den Hochschulen}
\sidebar{
    \centering
    \includegraphics[width=3cm]{bilder/dear_CERN_1.png}\\\vspace{14mm}
    \includegraphics[width=3cm]{bilder/dear_CERN_2.png}\\\vspace{14mm}
    \includegraphics[width=3cm]{bilder/dear_CERN_3.png}\\\vspace{14mm}
    \includegraphics[width=3cm]{bilder/dear_CERN_4.png}\\\vspace{14mm}
    \includegraphics[width=3cm]{bilder/dear_CERN_5.png}\\\vspace{14mm}
    \includegraphics[width=3cm]{bilder/dear_CERN_6.png}
}
Bis 1969 hatten die LehrstuhlinhaberInnen (Ordinarien) das Sagen an den
Universitäten. Er/sie hatte praktisch die alleinige Entscheidungsbefugnis
für seinen/ihren Lehr"= und Forschungsbereich. Sämtliche Gremien setzten
sich alleine aus LehrstuhlinhaberInnen zusammen. 1969 wurde diese
Ordinarienuniversität im Zuge der Forderungen nach Demokratisierung und
Mitbestimmung abgeschafft und die Gruppenuniversität eingeführt. Die
Mitglieder der Universität wurden in Gruppen eingeteilt, die
ProfessorInnen, die wissenschaftlichen MitarbeiterInnen (=\ Mittelbau), die
StudentInnen sowie die sonstigen MitarbeiterInnen (=\ Sonstige). Jeder
dieser „Stände“ wählt bei Uniwahlen eine bestimmte Anzahl VertreterInnen
in die Gremien der Uni. Der „erste Stand“ , die ProfessorInnen, stellen
zusätzlich eine gewisse Anzahl von Mitgliedern kraft Amtes. Für eine kurze
Phase wählten alle Gruppen außer den Sonstigen (HausmeisterInnen,
SekretärInnen, \dots) gleich viele VertreterInnen („Drittelparität“).
1973 stellte das Bundesverfassungsgericht (mit 6:2) aber fest, dass
aufgrund der grundgesetzlich garantierten Freiheit von Forschung und Lehre
(Art. 5 GG) die Gruppe der ProfessorInnen in allen Gremien eine
maßgebende, bzw. in bestimmten Fragen eine ausschlaggebende Mehrheit haben
muss. Damit ist eine relative bzw. in einigen Gremien eine absolute
Mehrheit gemeint. Aufgrund dieses Urteils sind die Entscheidungsgremien so
zusammengesetzt, dass ProfessorInnen mindestens so viele Sitze haben wie
alle anderen Gruppen zusammen. Bei den Abstimmungen, bei denen nur die
Mehrheit der professoralen Stimmen zählt, dürfen zwar alle abstimmen,
allerdings werden nur die professoralen Stimmen gezählt. Hierzu werden
„gezinkte“ Stimmzettel ausgegeben: auf dem Stimmzettel wird markiert, ob
man Prof oder nicht-Prof ist.



\section{Studentische Vertretung}
Um die Interessen der Studierenden artikulieren und durchsetzen zu können,
muss es eine Instanz geben, die sie vertritt. In einigen Bundesländern
(allen außer Bayern und Baden"=Württemberg) nimmt diese Aufgabe die
Verfasste Studierendenschaft (VS) wahr, d.h. die Studierenden geben sich
in direkten Wahlen eine Vertretung, meistens einen StudentInnenrat (StuRa)
oder auch ein Studierendenparlament, der/das eine „Regierung“ , den AStA
(Allgemeiner Studierendenausschuß) wählt, welcher die Beschlüsse der VS
vollzieht und z.B. über die zur Verfügung stehenden Finanzen beschließt.
Die Rechte der Verfassten Studierendenschaft können dabei sogar so weit
reichen, dass diese als Gemeinschaft öffentlichen Rechtes eigene Verträge
abschließen kann -- oft basieren Semestertickets auf solchen Verträgen.

Auch in Baden"=Württemberg gab es bis 1977 eine Verfasste
Studierendenschaft. Um die Studierenden vor politischen Dummheiten zu
bewahren, wurde jedoch 1977 — als weitere Folge des Urteils von 1973 — in
den Ländern Bayern und Baden"=Württemberg die VS in der bisherigen Form
vorsichtshalber abgeschafft. Da es aber einem demokratischen Staat mit
folglich (?) demokratischen Universitäten widerspricht, wenn die
zahlenmäßig stärkste Gruppe ausgeschaltet wird, richtete man einen
besonderen Ausschuss des Senats ein: den Ausschuss für musische,
sportliche, geistige und soziale Belange der Studierenden. Diesen rein
beratenden Ausschuss bezeichnete man dreist genauso wie die bisherige
Studierendenvertretung als AStA (im folgenden nur noch als sogenannter
AStA, „AStA“ bezeichnet). Er wird gebildet aus den studentischen
Mitgliedern des Senats und einigen KandidatInnen entsprechen der
erhaltenen Stimmenanzahl. Tätig werden darf der baden"=württembergische
Pseudo-„AStA“ gemäß LHG nur unter der Rechtsaufsicht des Rektors. Zu
Fragen des Studiums, zu Problemen einzelner Fachbereiche oder gar zu
politischen Fragen, z.B. BAföG oder Semesterticket darf der „AStA“ nicht
aktiv werden. Eine Vertretung auf Fachbereichsebene war überhaupt nicht
vorgesehen. Auf diese Beschneidung ihrer Rechte reagierten die
Studierenden, indem sie ihre eigenen Vertretungen parallel zur
Pseudovertretung im „AStA“ schufen. Im Gegensatz zu den abhängigen,
offiziellen Gremien werden sie als unabhängige Gremien bezeichnet.
An vielen Fachbereichen gibt es kontinuierlich oder immer mal wieder
Institutsgruppen, Fachschaftsinitiativen oder unabhängige Fachschaften,
die sich durch öffentliche Treffen und ihre Arbeit am Fachbereich
(Gremienarbeit, Klausurensammlung, Vorlesungsumfragen, Feten,
ErstsemesterInneneinführungen) legitimieren. Sie setzen sich vor Ort für
die Belange der Studierenden eines Faches ein. Diese Vertretungs"= und
Arbeitsstrukturen ersetzen wirkungsvoll die fehlende gesetzlich verankerte
Mitbestimmung oder gar Vertretung. Wie die Zusammenarbeit mit den
jeweiligen Instituten oder Seminaren klappt, hängt jedoch immer noch vom
Wohlwollen der ProfessorInnen, des Rektors/der Rektorin und des Ministeriums ab. In den
Fachbereichen Mathematik, Physik und Informatik übernimmt diese Aufgabe
seit 1983 die Fachschaft MathPhys\footnote{\url{http://mathphys.fsk.uni-heidelberg.de}}. Bis 1988 gab es an fast allen
Fachbereichen eigene Vertretungen, die sich an den meisten Universitäten
zu eigenen Unabhängigen ASten (USten), Fachschaftsrätevollversammlungen
oder, wie in Heidelberg, zur Fachschaftskonferenz\footnote{\url{http://www.fachschaftskonferenz.de}} zusammengeschlossen
hatten. Die Landesregierung entschloss sich daher, auf das seit dem „AStA“
bewährte Mittel des Etikettenschwindels zurückzugreifen: Seit Anfang 1990
bilden die studentischen Mitglieder im Fakultätsrat einen Ausschuss des
Fakultätsrats, der „Fachschaft“ genannt wird. Sie vertreten nicht — gemäß
der üblichen Verwendung der Bezeichnung — die Studierenden eines
Fachbereichs, sondern sind nur ein beratendes Anhängsel des Fakultätsrats.
Hauptziel der Etikettieraktion war nicht die Anerkennung jahrelanger
Fachbereichsarbeit, sondern die „Zurückdrängung der Substrukturen, die
sich als Ersatz für die Verfaßte Studierendenschaft gebildet haben“ (Zitat
Klaus von Trotha, damals CDU"=Fraktionssprecher im Landtag,
zwischenzeitlich auch Wissenschaftsminister).


\subsection{Zentrale Entscheidungsgremien}
Die Universität Heidelberg ist eine Lehr"= und Forschungseinrichtung mit
über 400 ProfessorInnen, circa 25\,000 Studierenden und vielen sonstigen
MitarbeiterInnen. Der Etat der Universität beträgt um die 550 Millionen
Euro. Die Aufgabe der Universität, die „Pflege und Entwicklung der
Wissenschaften und der Künste durch Forschung, Lehre und Studium“ (§2
% Quelle: http://www.umwelt-online.de/recht/allgemei/laender/bw/hschg01.htm
LHG), stellt hohe finanzielle und organisatorische Ansprüche.
Entscheidungen für die gesamte Hochschule treffen die zentralen Gremien,
Rektorat, Senat und Universitätsrat. Der Senat beschließt über
universitätsweite Belange wie Verlegung der Semesterzeiten, die
Immatrikulationsordnung und generell über grundlegende Fragen, welche die
Gesamtuniversität betreffen. Der Senat bestätigt Vorschläge der einzelnen
Fachbereiche über die Berufung neuer Professoren, genehmigt Lehrpläne und
beschließt über die Einrichtung oder Aufhebung von Studiengängen. Dem
Senat gehören außer dem Rektorat alle DekanInnen (das sind die
Vorsitzenden der Fakultäten), die Frauenbeauftragte und 8 gewählte
ProfessorInnen, 4 Studierende, 4 VertreterInnen aus dem Mittelbau und 4
Sonstige an. Das macht also ein Verhältnis von 27, bzw. 28 Profs gegen 12
andere, davon 4 Studis. Besondere Themen wie Umweltfragen oder
Prüfungsangelegenheiten werden in beratenden Ausschüssen des Senats
vorbereitet, in denen alle Gruppen vertreten sind, die ProfessorInnen
natürlich mit absoluter Mehrheit.

\sidebar{
    \centering
    \includegraphics[width=3cm]{bilder/inequivalence_principle_1.png}\\\vspace{13mm}
    \includegraphics[width=3cm]{bilder/inequivalence_principle_2.png}\\\vspace{13mm}
    \includegraphics[width=3cm]{bilder/inequivalence_principle_3.png}\\\vspace{13mm}
    \includegraphics[width=3cm]{bilder/inequivalence_principle_4.png}\\\vspace{13mm}
    \includegraphics[width=3cm]{bilder/inequivalence_principle_5.png}\\\vspace{13mm}
    \includegraphics[width=3cm]{bilder/inequivalence_principle_6.png}
}


Den Universitätsrat gibt es erst seit dem Wintersemester 2000/2001. Er ist
eine Art Aufsichtsrat der Universität. Er berät über die strukturelle
Ausrichtung der Universität und hat das letzte Wort in
Finanzangelegenheiten. Er besitzt damit wesentlichen Einfluss auf die
künftige Entwicklung der Universität. 6 der 11 Mitglieder sind
Universitätsexterne aus Politik, Kultur und Wirtschaft, 5 Mitglieder
kommen aus der Universität. Alle Mitglieder werden vom
Wissenschaftsministerium benannt.

Sieht man den Universitätsrat als Aufsichtsrat der Universität, wird das
Rektorat zum zugehörigen Vorstand. Es besteht aus dem/der RektorIn, 4
ProrektorInnen und dem/der LeiterIn der Verwaltung: dem/der KanzlerIn.
Rektor ist zur Zeit \rektor . Neben seinen Verwaltungsaufgaben in
der Uni, vertritt er sie auch nach außen, z.B. gegenüber dem Land. Das
Rektorat residiert in der alten Universität, Grabengasse 1. Die
Universitätsverwaltung ist in der Seminarstr. 2 angesiedelt.

\subsection{Die Fakultät}

Die speziellen Fragen eines Fachbereichs werden in der Fakultät, der
„organisatorischen Grundeinheit der Universität“, die „gleiche oder
verwandte Fachbereiche zusammenfasst“ (LHG), (vor)entschieden. Die
Universität ist in zwölf Fakultäten gegliedert, zum Beispiel die Fakultät für
Mathematik und Informatik und die Fakultät für Physik und Astronomie.
Während man an einigen Fakultäten nur ein bzw. wenige Fächer studieren
kann, gibt es andere Fakultäten, wie z.B. die Neuphilologische Fakultät,
an denen viele Fächer (Germanistik, Anglistik, Romanistik etc.) studiert
werden können. Geleitet wird die Fakultät von einem/einer DekanIn und dem
ihm unterstehenden Dekanat, das die laufenden Geschäfte erledigt. Der/Die
DekanIn wird für vier Jahre vom Fakultätsrat gewählt. Zur Zeit ist Dekan der
Mathematik Prof. \dekanmathe{} und in der Physik Prof. \dekanphysik. Zusammen mit dem
Studiendekan (s.u.) bilden Dekan und Prodekan den Fakultätsvorstand. Zu
finden ist das Dekanat der Mathe \gls{INF} 288, Zimmer 277 und das der Physik
in der Albert Ueberle Str. 3-5.

Die Mitglieder der Fakultäten wählen in dem beschriebenen
Vierklassenwahlrecht den Fakultätsrat, das oberste Gremium der Fakultät.
Er ist zuständig für alle Fragen der Lehre und der Forschung. Ihm gehören
6 gewählte Profs, 5 Studis, 4 Angehörige aus dem Mittelbau und 1 Mitglied
aus der Gruppe der MitarbeiterInnen aus Administration und Technik
an, außerdem alle InstitutsdirektorInnen und der Fakultätsvorstand, also Dekan,
Prodekan und Studiendekan. Der Fakultätsrat bildet Ausschüsse mit vergleichbarer
Zusammensetzung, die sich um bestimmte Bereiche kümmern, besonders wichtig
sind zum Beispiel die Prüfungsausschüsse oder die Studienkommissionen. Die
Mitglieder der Ausschüsse müssen nicht immer Mitglieder des Fakultätsrats
sein. Die Fakultäten gliedern sich weiter in Institute. Jedes Institut
wird von einem Institutsdirektor geleitet.

\subsection{Studiendekan und Studienkommission}

Auch im Januar 1995 gab es schon einmal ein neues UG — und auch damals gab es ein
neues Gremium: Die Studienkommission. In diesem Fall allerdings noch zum Vorteil
aller. Die Studienkommission ist ein Gremium, das den Studiendekan/die
StudiendekanIn, der/die Kraft Amtes ihr Vorsitzender/ihre Vorsitzende ist, berät.
Studiendekan und Studienkommission sollen gemeinsam zur Verbesserung der Lehre
beitragen. Die Studienkommission besteht aus StudiendekanIn, drei weiteren
ProfessorInnen, zwei VertreterInnen des Mittelbaus sowie vier Studierenden.
Hauptaufgaben der Kommission sind:
\begin{itemize}
    \addtolength{\itemsep}{-0.7\baselineskip}
    \item Empfehlungen zur Weiterentwicklung von Gegenständen und Formen des Studiums
    \item „Verfahren zur Bewertung und Verbesserung  der Qualität der Lehre unter
          Einbeziehung studentischer Veranstaltungskritik“ zu entwickeln
    \item In regelmäßigen Abständen einen Lehrbericht zu verfassen.
\end{itemize}


Die Umsetzung von Empfehlungen und die Wahrnehmung laufender Aufgaben
obliegt dem Studiendekan/der Studiendekanin. Die Kommission ist nur
beratend und die Durchsetzung ihrer Beschlüsse und Ideen ist vom guten
Willen des Fakultätsrats abhängig. Der Studiendekan / die Studiendekanin
nimmt die mit Forschung und Lehre zusammenhängenden laufenden Aufgaben
wahr; seine/ihre Aufgabe ist es insbesondere, auf ein ordnungsgemäßes und
vollständiges Lehrangebot hinzuwirken und die Beschlussfassung über
Studienpläne, Studien"= und Prüfungsordnungen und Lehrberichte
vorzubereiten.


\section{Die Fachschaft MathPhys}
\hyperref[diefsmathphys]{Informationen über die Fachschaft MathPhys findet Ihr im gleichnamigen Kapitel.}

\section{Die FSK}
\sidebar{
    \centering{
        \vspace{13mm}
        \includegraphics[width=3cm]{bilder/ringtone_1.png}\\\vspace{35mm}
        \includegraphics[width=3cm]{bilder/ringtone_2.png}\\\vspace{35mm}
        \includegraphics[width=3cm]{bilder/ringtone_3.png}\\\vspace{35mm}
    }
}
Die unabhängigen Fachschaften koordinieren sich in Heidelberg
universitätsweit in der \gls{FSK}\footnote{\url{www.fachschaftskonferenz.de}}. Ihr gehören als
stimmberechtigte Mitglieder alle aktiven Fachschaften an. Die
fachbereichsübergreifende Arbeit der \gls{FSK} läuft hauptsächlich über
Referate, Arbeitskreise und in Gremien. Manchmal übernehmen auch
Fachschaften uniweite Aufgaben. Vier Hauptfunktionen nimmt die \gls{FSK} wahr:
die Fachschaften koordinieren sich untereinander, die hochschul"= und
allgemeinpolitischen Aktivitäten von Referaten und Arbeitskreisen werden
diskutiert, geplant und ggf. beschlossen, die Arbeit der
\gls{FSK}-VertreterInnen in offiziellen Gremien wird koordiniert und es wird
über die Verwendung von Geldern zur Förderung studentischer Aktivität
(z.B. Unterstützung für Theatergruppen, etc.) entschieden. Die \gls{FSK} tagt im
Semester jede zweite Woche öffentlich am Dienstag um 19.00 Uhr \gls{s.t.} in den
Räumen des Zentralen Fachschaftenbüros (ZFB). Jeder kann hinkommen und
mitreden. Darüber hinaus unterstützt die \gls{FSK}, auf Antrag, auch viele
unterschiedliche studentische Gruppen organisatorisch und finanziell. Von
der \gls{FSK} wird auch die Zeitung UNiMUT herausgegeben, auch online\footnote{\url{http://unimut.fsk.uni-heidelberg.de/aktuell/}}.

Die \gls{FSK} ist eure Studierendenvertretung an der Hochschule und darüber hinaus. Sie agiert unabhängig von parteipolitischen Interessen. Referate und Arbeitskreise unterstützen die Arbeit der Fachschaften inhaltlich; sie befassen sich mit übergreifenden Themen wie Studienreform, Semesterticket, Studiengebühren, BAföG, Öffentlichkeitsarbeit oder EDV. Ergebnisse und Anregungen aus dieser Arbeit bringen sie in die \gls{FSK} ein. Während die Arbeitskreise inhaltlich unabhängiger sind, sind ReferentInnen wie \gls{FSK}-VertreterInnen in Gremien der \gls{FSK} rechenschaftspflichtig und setzen in ihrer Arbeit die Beschlüsse der \gls{FSK} um.

In Kooperation mit Studierendenvertretungen anderer Hochschulen, durch Öffentlichkeitsarbeit und Aktionen bemüht sich die \gls{FSK} um Durchsetzung studentischer Interessen.

Tagesordnungen und Protokolle der \gls{FSK}-Sitzung sind für alle Studierenden auf der Homepage der Fachschaftskonferenz einsehbar.

\paragraph{Arbeitsbereiche und Referate}
Die \gls{FSK} und die Fachschaften arbeiten in vielen Bereichen, zum Beispiel:

\vspace{-0.6\baselineskip}
\begin{itemize}
 \addtolength{\itemsep}{-0.6\baselineskip}
 \item Fachschaftsvernetzung und -koordination
 \item Unterstützung studentischer Gruppen
 \item landes- und bundesweite studentische Politik
 \item Erstsemestereinführungen
 \item Studentische Studienberatung
 \item Lehramtsbildung
 \item Kommunales, z.B. Semesterticket
 \item Gremienarbeit, z.B. in den Fakultätsräten, im Senat, beim Studentenwerk
 \item Studienreform
\end{itemize}

Koordiniert wird die Arbeit in den zuständigen Referaten:

\vspace{-0.6\baselineskip}
\begin{itemize}
 \addtolength{\itemsep}{-0.6\baselineskip}
 \item EDV-Referat
 \item Sozialreferat
 \item Referat für Frauen- und Geschlechterpolitik
 \item Referat für Kommunales und Verkehr
 \item Kultur- und Sportreferat
 \item Referat für Finanzen und Internes
 \item Referat für Studienreform und hochschulpolitische Entwicklungen
 \item Referat für Öffentlichkeitsarbeit \& Agitation
 \item Referat für Politische Bildung und Vernetzung
\end{itemize}

Sämtliche Referate und Arbeitskreise der FSK freuen sich jederzeit über Studierende, die Interesse haben, sich in den entsprechenden Themen einzubringen und mitzuarbeiten. Die Kontaktdaten der Referate und Arbeitskreise findet ihr auf der Homepage der FSK.


