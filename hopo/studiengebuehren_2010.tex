\section{Studiengebühren}

Seit dem Sommersemester 2007 gibt es an der Universität Heidelberg und in ganz BaWü
Studiengebühren in Höhe von \EUR{\studiengebuehren}. Zudem muss jeder Studierende einen
Verwaltungsbetrag von \EUR{\verwaltungsbetrag} und einen Beitrag an das Studentenwerk von \EUR{\studentenwerksbeitrag}
bezahlen. Das macht \EUR{\beitragssumme} die ihr jedes Semester an die Uni überweisen müsst.
Ganz schön viel Geld. Wie kam es dazu?

\subsection*{Geschichte der Studiengebühren}

Im Wintersemester 1970/71 wurden in der Bundesrepublik Deutschland die allgemeinen
Studiengebühren, damals Hörergeld, mit Hilfe von Protesten und Boykotten abgeschafft.
Aber immer wieder gab es Bewegungen in der Politik, die Studiengebühren forderten.
Es wurden so genannte Rückmeldegebühren eingeführt um die Haushaltslöcher der Uni zu
stopfen und darüber nachgedacht Studiengebühren für Langzeitstudierende einzuführen.
Diese wurden in verschiedenen Ländern auch durchgesetzt (In BaWü zum WS 1998/99).

Im August 2002 wurde von der Kultusministerkonferenz ein allgemeines
Studiengebührenverbot festgeschrieben. Langzeitstudiengebühren oder aber so genannte
Studienkonten demgegenüber, waren in bestimmten Ausnahmefällen erlaubt.

Gegen eben dieses Verbot hatten sich die Bundesländer Baden"=Württemberg, Bayern,
Hessen, Sachsen, Hamburg und das Saarland gewandt. Die klagenden Bundesländer
führten als Grund ihres Ganges nach Karlsruhe an, dass der Bund seine
Gesetzgebungskompetenz überschritten und in die Länderkompetenz eingegriffen habe.

Am 26.01.2005 fällte das Bundesverfassungsgericht in Karlsruhe das Urteil, dass der
Bund nur dann die Rahmenregelungen für die Bildungspolitik der Länder festlegen
kann, und ein Verbot allgemeiner Studiengebühren nur dann gerechtfertigt sei, um
„gleichwertige Lebensbedingungen“ zu wahren. In diesem Fall sei aber
kein Anlass zu solcher Sorge gegeben.  Somit wurde das Gesetz von 2002 gekippt und
der Weg für die allgemeinen Studiengebühren geebnet.

Kurz darauf wurde in Baden"=Württemberg ein Gesetzentwurf zur Einführung allgemeiner
Studiengebühren von Minister Frankenberg vorgelegt und im Dezember 2005 von der
Landesregierung beschlossen.  Studiengebühren in Höhe von \EUR{500} pro Semester
wurden daraufhin zum Sommersemester 2007 eingeführt. Es gab weiter Proteste gegen
die Einführung, Demos und Boykotts wurden organisiert.

Der AK Studiengebühren der \gls{FSK} organisierte auch in Heidelberg einen Boykott und es
wurden Klagen beim Verwaltungsgericht eingereicht. Wie an vielen anderen
Universitäten in Baden"=Württemberg wurden die Studierenden dazu aufgefordert, die
\EUR{500} nicht an die Universität, sondern auf ein Treuhandkonto zu überweisen. Auf
einer studentischen Vollversammlung wurde beschlossen, den Boykott ab einer
Beteiligung von 4500 Studierenden durchzuführen. Man hätte dann unter Bezug auf das
zurückgehaltene Geld als Druckmittel mit der Landesregierung Gespräche begonnen.
Leider zahlten nur ca. 1200 Studierende auf das Treuhandkonto ein. Das Quorum wurde
nicht erreicht, der Boykott daher nicht durchgeführt und das Geld an die Universität
überwiesen. Die Studiengebühren konnten damit leider nicht verhindert werden.

Seit dem 1.\ Januar 2009 gilt die sog. Zweigeschwisterregelung: „Jene Studierende
werden von der Studiengebühr in Höhe von \EUR{500} befreit, die zwei oder mehr
Geschwister haben, von denen zwei keine Befreiung nach dieser Vorschrift in Anspruch
nehmen oder genommen haben; wurde ein Studierender für weniger als sechs Semester
nach dieser Vorschrift befreit, kann die verbleibende Semesterzahl von einem anderen
Geschwister in Anspruch genommen werden.“ Im Gegensatz zu früher können sich jetzt
auch Studierende befreien lassen, deren Geschwister gar nicht studieren, deren
Geschwister ihr Studium bereits abgeschlossen haben, deren Geschwister wegen einer
Behinderung von den Studiengebühren befreit sind oder deren Geschwister außerhalb
von Baden"=Württemberg studieren.

Von Gebühren befreit bist du, wenn du ein Kind unter 8 Jahren hast, ein Praxissemester absolvierst, zwei ältere Geschwister hast oder du unter einer studienerschwerenden Behinderung leidest. Eine Befreiung aufgrund außerordentlicher Studienleistungen ist gesetzlich möglich, wird aber von den Fakultäten für Mathematik und Physik abgelehnt und darum nicht gewährt.



\subsection*{Verteilung der Studiengebühren an der Universität Heidelberg}

Die Studierenden in Heidelberg setzten durch, dass die Studiengebühren ausschließlich
zur Verbesserung der Lehre verwendet werden dürfen und dass in den Kommissionen zur
Verteilung der Gebühren Studierende die absolute Mehrheit haben. Die studentischen
Vertreter der Kommission werden auf Vorschlag der Fachschaft vom Fakultätsrat
gewählt. Diese Gremien haben allerdings nur beratende Funktion. Der von der
Kommission beschlossene Verwendungsplan wird dem Fakultätsrat zur Entscheidung
vorgelegt. Die Budgetverantwortung liegt beim Fakultätsvorstand.


Der größte Teil der Gebühren, etwa \EUR{330} pro Studi, geht an dein Fach. Ein
kleiner Teil -- etwa \EUR{25} -- geht an die zentralen Einrichtungen der Universität,
also an die Bibliothek, das Rechenzentrum oder den Hochschulsport. Der Rest wird vom
Land Baden"=Württemberg zur Sicherung der Gebührenkredite einbehalten.

Die einzelnen Fächer haben in den letzten Semestern die ihnen zur Verfügung
stehenden Gelder nur teilweise ausgegeben, sodass insgesamt noch eine Summe von
\EUR{1,3\,Mio} auf den einzelnen Konten der Fakultäten liegen. Daraufhin
versucht das Rektorat durchzusetzen, dass mehr Gelder in zentrale Einrichtungen
fließen. Allerdings regt sich hier seitens der Fakultäten und der studentischen
Vertreter Widerstand. Unklar ist auch, wie viel Prozent das Rektorat einbehalten
wird.


\subsection*{Verwendung der Studiengebühren}

Die größten Defizite im Bereich Lehre waren an unseren Fakultäten das schlechte
Betreuungsverhältnis, vor allem in den Übungsgruppen und Seminaren, sowie ein
ungenügendes Serviceangebot. Um diese Defizite auszugleichen wurde mit den
Sondermitteln aus Studiengebühren insbesondere in folgende Bereiche investiert:

\vspace{5mm}
\textbf{Mathematik und Informatik}
\begin{itemize}
 \item {Mitarbeiterstellen}\\Lehr-/Servicepersonal, Lehraufträge, AssistentInnen
\item {Hilfskraftmittel}\\ TutorInnen und zusätzliche Übungsgruppen
\item {Ausstattung}\\ Computerpools, Hörsäle, Seminarräume
\item {Materialien}\\ Softwarelizenzen, Skripte, Bücher, eBooks\footnote{\url{http://www.ub.uni-heidelberg.de/helios/epubl/eb/Welcome.html}}, Zeitschriften
\end{itemize}

Weiter wurden Mittel zur Modernisierung der mangelhaften Ausstattung der Hörsäle,
Computerpools, sowie der Bibliothek investiert. Außerdem fließt ein Teil des Geldes
in Zuschüsse an Studierende zu Exkursionen, Sprachkursen, Teilnahme an Tagungen und
ähnlichem.

\vspace{5mm}
\textbf{Physik}
\begin{itemize}
\item {Mittelbaustellen}\\Medizinerausbildung, Studienberatung, Lehramtsausbildung, Aufbau des IUP (Glossar) Praktikums
\item {AP + FP Versuche (Glossar)}\\Modernisierung bestehender Versuche, neue
Praktika im FP und dem IUP
\item {Öffnungszeiten}\\Erweiterung der Öffnungszeiten in der \gls{KIP}-Bibliothek sowie dem Studentensekretariat
\item {Hilfskraftmittel}\\ TutorInnen und zusätzliche Übungsgruppen
\item {Materialien}\\Skripte, Bücher, eBooks\footnote{\url{http://www.ub.uni-heidelberg.de/helios/epubl/eb/Welcome.html}}, Zeitschriften
\end{itemize}

Da experimentelle Erfahrung enorm wichtig für einen Physiker ist, wurde und wird
viel Geld in aktuellste Versuche investiert. Eine Bereicherung stellen sicher auch
die Exkursionen nach vielen Kursvorlesungen dar (CERN, GSI\dots).

Eine genauere aktuelle Auflistung findet ihr im aktuellen MathPhys-Info
(siehe FS-Homepage\footnote{\url{http://mathphys.fsk.uni-heidelberg.de/mpi.html}}).

\vfill
\begin{figure}[h]
\centering{
    \includegraphics[width=0.7\textwidth]{bilder/studiengebuehren.png}
}
\end{figure}
\vfill
