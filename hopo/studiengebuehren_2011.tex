\section{Studiengebühren}

Seit dem Sommersemester 2007 gibt es in ganz BaWü
Studiengebühren in Höhe von \EUR{\studiengebuehren}. Zudem muss jedeR StudentIn einen
Verwaltungsbetrag von \EUR{\verwaltungsbetrag} und einen Beitrag an das Studentenwerk von \EUR{\studentenwerksbeitrag}
bezahlen. Das macht \EUR{\beitragssumme} die ihr jedes Semester an die Uni überweisen müsst.
Ganz schön viel Geld. Wie kam es dazu?


\subsection*{Geschichte der Studiengebühren}

Im Wintersemester 1970/71 wurden in der Bundesrepublik Deutschland die allgemeinen
Studiengebühren, damals Hörergeld, mit Hilfe von Protesten und Boykotten abgeschafft.

Aber immer wieder gab es Bewegungen in der Politik, die Studiengebühren forderten. Das ging meistens mit einem Bild von Universität einher, das diese nicht als öffentliche Einrichtung zur Verbreitung des gesellschaftlichen Guts Bildung, sondern als Anbieterin von Leistungen auf dem Markt der Ware Bildung auffasst. Entsprechend wenig ist solche Politik bereit, Bildungseinrichtungen eine angemessene finanzielle Ausstattung zukommen zu lassen.

Es wurden Rückmeldegebühren eingeführt um die Haushaltslöcher der Unis zu
stopfen. Diese mussten in BaWü Anfang der 1990er arge Kürzungen im Haushalt hinnehmen, der 1996 für 10 Jahre eingefroren wurde (bis auf Baumaßnahmen).
Gleichzeitig wurde laut darüber nachgedacht, Studiengebühren für Langzeitstudierende einzuführen.
Diese wurden in verschiedenen Ländern auch durchgesetzt (in BaWü zum WS 1998/99).

2002 wurde von der Kultusministerkonferenz ein allgemeines
Studiengebührenverbot festgeschrieben. Langzeitstudiengebühren waren in bestimmten Ausnahmefällen erlaubt.

Gegen dieses Verbot klagten einige Bundesländer vorm Bundesverfassungsgericht, darunter BaWü. Sie
führten an, dass der Bund seine Gesetzgebungskompetenz überschritten und in die Länderkompetenz eingegriffen habe.

Am 26.01.2005 fällte das Bundesverfassungsgericht das Urteil, dass ein Verbot allgemeiner Studiengebühren nur gerechtfertigt sei, um
„gleichwertige Lebensbedingungen“ in den Ländern zu wahren. In diesem Fall sei aber
kein Anlass zu solcher Sorge gegeben. Somit wurde das Gesetz von 2002 gekippt und
der Weg für die allgemeinen Studiengebühren geebnet.

Schon im Dezember 2005 wurde in Baden"=Württemberg Minister Frankenbergs Gesetzentwurf zur Einführung allgemeiner
Studiengebühren von der
Landesregierung beschlossen. Studiengebühren in Höhe von \EUR{500} pro Semester
wurden zum Sommersemester 2007 eingeführt. Es gab weiter Proteste gegen
die Einführung, Demos und Boykotts wurden organisiert.

Der AK Studiengebühren der \gls{FSK} organisierte einen Boykott und es
wurden Klagen beim Verwaltungsgericht eingereicht. Wie an vielen anderen
Universitäten in Baden"=Württemberg wurden die Studierenden in Heidelberg von der studentischen Vollversammlung dazu aufgefordert, die
\EUR{500} nicht an die Universität, sondern auf ein Treuhandkonto zu überweisen. Man hätte dann unter Bezug auf das
zurückgehaltene Geld als Druckmittel mit der Landesregierung Gespräche begonnen.
Leider wurde das beschlossene Quorum mit nur ca. 1200 statt 4500 eingegangenen Zahlungen
nicht erreicht und der Boykott daher nicht durchgeführt. Die Studiengebühren konnten damit nicht abgeschafft werden.

Auch im Bildungsstreik, der seit 2009 die Bildungspolitik aufrüttelt, war die Forderung nach Abschaffung der Studiengebühren (als einer Bildungsgebühr von vielen) stets zentral. Darum organisierten viele baden"=württembergische Bildungsstreik-Gruppen, auch in Heidelberg, nach den Erfahrungen aus Hessen und NRW im Hinblick auf die Landtagswahl nochmals Demonstrationen im Januar 2011. Von SPD über Grüne bis Linkspartei fand sich im Wahlprogramm denn auch die Absichtserklärung wieder, die Studiengebühren abzuschaffen.

Einige Monate nach der Wahl der grün-roten Landesregierung im März 2011 zeichnet sich nun ab, dass Studis in BaWü ab dem Sommersemester 2012 keine Studiengebühren mehr zahlen müssen!

Ein vollständiger finanzieller Ausgleich soll aus Landesmitteln erfolgen. An den Verteilungsmodalitäten soll sich nichts ändern, sodass die Mittel weiterhin zweckgebunden für die Verbesserung der Studienbedingungen sind und ihre Verwendung weiterhin von Studis mitberaten wird.

Von Gebühren befreit bist du auch schon in diesem Semester, wenn du ein Kind unter 14 Jahren hast, ein Praxissemester absolvierst, ein Stipendium erhältst, du mindestens zwei (auch Stief"=/Halb"=) Geschwister hast die nicht diese baden"=württembergische Geschwisterregelung in Anspruch nehmen, oder du unter einer studienerschwerenden Behinderung leidest. Eine Befreiung aufgrund außerordentlicher Studienleistungen ist gesetzlich möglich, wird aber von den Fakultäten für Mathematik und Physik abgelehnt und darum nicht gewährt.


\subsection*{Ein paar Argumente gegen Studiengebühren}

Wie ihr sich bemerkt habt, sind wir oben a priori davon ausgegangen, dass es gut sei, wenn es keine Studiengebühren gibt. Was veranlasst uns dazu, mal abgesehen von persönlicher Betroffenheit?
\begin{itemize}
\item {Barrieren zwischen Abi und Studium einreißen!}\\Ein Studium ist verglichen mit einer Ausbildung sehr teuer. Durch die Studiengebühren ist es nochmals knapp \EUR{100}/Monat teurer geworden. Das schreckt nachweislich insbesondere AbiturientInnen ab, deren Eltern selbst nicht studiert haben. Außerdem schreckt es prozentual mehr Mädchen als Jungs ab. Kurzum: Menschen, deren Umgebung von ihnen traditionell weniger erwartet dass sie studieren, studieren weniger; gesellschaftliche Ungleichheit, Stratifizierung und überkommene Rollenbilder werden perpetuiert.
\item {Aufhebung von finanziell bedingten Nachteilen im Studium!}\\Studis, die knapp über die BAföG-Grenze fallen, sind am härtesten von den Studiengebühren betroffen. Denn bei ihnen reicht das Geld der Eltern oft nicht, um Lebensunterhalt und Gebühren zu finanzieren. Sie müssen also jobben gehen und haben somit deutlich weniger Zeit, sich um ihr eigentliches Studium zu kümmern. Gerade in Zeiten der Master"=Zulassungsbeschränkung führt das im Endeffekt zu finanzieller Auslese.
\item {Bildung ist öffentliches Gut!}\\Die Einführung der Studiengebühren wurde von eingefrorenen Haushälten und massiven Kürzungen im Bildungsbereich seitens der Landesregierungen flankiert. Studiengebühren sind keine Notwendigkeit der Lage, sondern Politik. Unserer Auffassung nach ist Bildung jedoch eins der wichtigsten Güter einer Gesellschaft, und das meinen wir nicht im rein ökonomischen Sinn. Gerade die universitäre Bildung bietet (noch, vergleichsweise) viele Freiräume zum eigenständigen Denken. Dessen Bedeutung ist für eine Gesellschaft, die sich demokratisch nennen möchte, nicht zu unterschätzen. Deshalb sollte Bildung in den Landeshaushalten höchste Priorität haben.
\end{itemize}


\subsection*{Verteilung der Studiengebühren an der Universität Heidelberg}

Die Studierenden in Heidelberg setzten durch, dass die Studiengebühren ausschließlich
zur Verbesserung der Lehre verwendet werden dürfen und dass in den Kommissionen zur
Verteilung der Gebühren Studierende die absolute Mehrheit haben. Die studentischen
VertreterInnen der Kommission werden auf Vorschlag der Fachschaft vom Fakultätsrat
gewählt. Diese Gremien haben allerdings nur beratende Funktion. Der von der
Kommission beschlossene Verwendungsplan wird dem Fakultätsrat zur Entscheidung
vorgelegt. Die Budgetverantwortung liegt beim Fakultätsvorstand.

Der größte Teil der Gebühren, etwa \EUR{330} pro Studi, geht an dein Fach. Ein
kleiner Teil -- etwa \EUR{25} -- geht an die zentralen Einrichtungen der Universität,
also an die Bibliothek, das Rechenzentrum oder den Hochschulsport. Der Rest wird vom
Land Baden"=Württemberg zur Sicherung der Gebührenkredite einbehalten.

Die einzelnen Fächer haben in den letzten Semestern die ihnen zur Verfügung
stehenden Gelder nur teilweise ausgegeben, sodass insgesamt noch eine Summe von
%  STIMMT DIE SUMME??
\EUR{1,3\,Mio} auf den einzelnen Konten der Fakultäten liegen. Daraufhin
versucht das Rektorat durchzusetzen, dass mehr Gelder in zentrale Einrichtungen
fließen. Allerdings regt sich hier seitens der Fakultäten und der studentischen
Vertretung Widerstand. Unklar ist auch, wie viel Prozent das Rektorat einbehalten
wird.


\subsection*{Verwendung der Studiengebühren}

Die größten Defizite im Bereich Lehre waren an unseren Fakultäten das schlechte
Betreuungsverhältnis, vor allem in den Übungsgruppen und Seminaren, sowie ein
ungenügendes Serviceangebot. Um diese Defizite auszugleichen wurde mit den
Sondermitteln aus Studiengebühren insbesondere in folgende Bereiche investiert:

\vspace{5mm}
\textbf{Mathematik und Informatik}
\begin{itemize}
 \item {Mitarbeiterstellen}\\Lehr-/Servicepersonal, Lehraufträge, AssistentInnen
\item {Hilfskraftmittel}\\ TutorInnen und zusätzliche Übungsgruppen
\item {Ausstattung}\\ Computerpools, Hörsäle, Seminarräume
\item {Materialien}\\ Softwarelizenzen, Skripte, Bücher, eBooks\footnote{\url{http://www.ub.uni-heidelberg.de/helios/epubl/eb/Welcome.html}}, Zeitschriften
\end{itemize}

Weiter wurden Mittel zur Modernisierung der mangelhaften Ausstattung der Hörsäle,
Computerpools, sowie der Bibliothek investiert. Außerdem fließt ein Teil des Geldes
in Zuschüsse an Studierende zu Exkursionen, Sprachkursen, Teilnahme an Tagungen und
ähnlichem.

\vspace{5mm}
\textbf{Physik}
\begin{itemize}
\item {Mittelbaustellen}\\Medizinerausbildung, Studienberatung, Lehramtsausbildung, Aufbau des IUP (Glossar) Praktikums
\item {AP + FP Versuche (Glossar)}\\Modernisierung bestehender Versuche, neue
Praktika im FP und dem IUP
\item {Öffnungszeiten}\\Erweiterung der Öffnungszeiten in der \gls{KIP}-Bibliothek sowie dem Studentensekretariat
\item {Hilfskraftmittel}\\ TutorInnen und zusätzliche Übungsgruppen
\item {Materialien}\\Skripte, Bücher, eBooks\footnote{\url{http://www.ub.uni-heidelberg.de/helios/epubl/eb/Welcome.html}}, Zeitschriften
\end{itemize}

Da experimentelle Erfahrung enorm wichtig für einen Physiker ist, wurde und wird
viel Geld in aktuellste Versuche investiert. Eine Bereicherung stellen sicher auch
die Exkursionen nach vielen Kursvorlesungen dar (CERN, GSI\dots).

%IST DAS SO??
%Eine genauere aktuelle Auflistung findet ihr im aktuellen MathPhys-Info
%(siehe FS-Homepage\footnote{\url{http://mathphys.fsk.uni-heidelberg.de/mpi.html}}).

\vfill
\begin{figure}[h]
\centering{
    \includegraphics[width=0.7\textwidth]{bilder/studiengebuehren.png}
}
\end{figure}
\vfill
