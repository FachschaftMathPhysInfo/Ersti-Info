% Kurzanleitung:
% https://mirror.ctan.org/macros/latex/contrib/glossaries/glossariesbegin.pdf

% Syntax:
% \newglossaryentry{ label }{ settings }
% \newacronym{ label }{ abbrv }{ full }


% Beispiele: Einträge erstellen
% \newglossaryentry{electrolyte}{name=electrolyte, description={solution able to conduct electric current}}

% \newglossaryentry{oesophagus}{name=œsophagus, description={canal from mouth to stomach}, plural=œsophagi}

% \newacronym{label}{svm}{support vector machine}


% Beispiel: sich auf einen Glossareintrag beziehen
%       \gls{ label }
%       \glspl{ label }   % gibt die Pluralform aus


\newglossaryentry{c.t.}{name={c.t.}, description={lat.: cum tempore, mit Zeit, also eine viertel Stunde später}}
\newglossaryentry{s.t.}{name={s.t.}, description={lat.: sine tempore, ohne Zeit, also genau so, wie es da steht}}
\newglossaryentry{HEIDI}{name={HEIDI}, description={So heißt das Computersystem der \gls{UB}, mit dem ihr nach Büchern suchen und sie auch gleich vorbestellen könnt. Eure aktuellen Ausleihen werden auch dort gelistet, ebenso besteht die Möglichkeit dort insgesamt zweimal eure Ausleihfristen um je einen Monat zu verlängern. Die Adresse ist \url{heidi.ub.uni-heidelberg.de}, aber eine Suche nach \emph{heidi heidelberg} führt auch zum Ziel}}


\newglossaryentry{URZ}{
    name={URZ},
    first={Universitätsrechenzentrum (URZ)},
    description={Das Universitätsrechenzentrum stellt alles bereit, was sich grob mit dem Begriff \emph{Computer} in Verbindung bringen lässt. Siehe Artikel auf \autopageref{urz}}
}

\newglossaryentry{AM}{
    name={AM},
    first={Angewandte Mathematik},
    description={Die Angewandte Mathematik ist eines der Institute der Fakultät für Mathematik und Informatik und sitzt im Mathematikon}
}

\newglossaryentry{IAM}{
    name={IAM},
    first={Institut für Angewandte Mathematik},
    description={siehe \Gls{AM}}
}

\newglossaryentry{MI}{
    name={MI},
    first={Mathematisches Institut},
    description={Das Mathematisches Institut -- auch Reine Mathematik genannt -- ist eines der Institute der Fakultät für Mathematik und Informatik und sitzt im Mathematikon im 3. OG}
}

\newglossaryentry{RM}{
    name={RM},
    first={Reine Mathematik},
    description={siehe \Gls{MI}}
}

\newglossaryentry{IfI}{
    name={IfI},
    first={Institut für Informatik},
    description={Das Institut für Informatik ist eines der Institute der Fakultät für Mathematik und Informatik und sitzt im Mathematikon. Die Mitglieder decken unter anderem einen Großteil der Lehre im Bereich Informatik ab}
}

\newglossaryentry{Mathematikon}{
    name={Mathematikon},
    first={Mathematikon},
    description={Das Mathematikon (INF 205) ist das neu gebaute Ge\-bäu\-de an der Berliner Straße, in dem die Fakultät für Mathematik und Informatik und das IWR untergebracht sind. Dort finden auch die meisten Seminare, Übungen und kleinere Vorlesungen statt. Außerdem befindet sich dort die Bereichsbibliothek, einige Arbeitsplätze und natürlich die Fachschaft}
}

\newglossaryentry{FSK}{
    name={FSK},
    first={Fachschaftskonferenz (FSK)},
    description={Die Fachschaftskonferenz war die Vorgängerin des StuRa als Vertretung aller Studis vor der Wiedereinführung der Verfassten Studierendenschaft}
}

\newglossaryentry{StuRa}{
    name={StuRa},
    first={Studierendenrat (StuRa)},
    description={Der StuRa ist das Zentralorgan der Verfassten Studierendenschaft an der Uni Heidelberg und tagt zweiwöchentlich im neuen Hörsaal am Philosophenweg. Details in den Artikeln auf \autopageref{hopo} ff}
}

\newglossaryentry{ZFB}{
    name={ZFB},
    first={Zentrales Fachschaftenbüro (ZFB)},
    description={siehe \Gls{StuRa-Buero}}
}

\newglossaryentry{StuRa-Buero}{
    name={StuRa-Büro},
    first={StuRa-Büro},
    description={früher auch: \Gls{ZFB}. Büro- und Tagungsräume für den StuRa und andere studentische Gruppen. Auch die Sozialberatung u.ä. finden hier statt. Die genaue Adresse ist Albert-Ueberle-Straße 3-5, 69120 Heidelberg}
}


\newacronym{UB}{UB}{Universitätsbibliothek}
\newacronym{OPNV}{ÖPNV}{Öffentlicher Personen Nahverkehr}
\newacronym{INF}{INF}{im Neuenheimer Feld}
\newacronym{HS}{HS}{Hörsaal}
\newglossaryentry{FSWE}{
    name={FSWE},
    first={Fachschaftswochenende (FSWE)},
    description={Das Fachschaftswochenende veranstalten wir einmal pro Semester, um Themen diskutieren zu können, die sich nicht sinnvoll in einer Fachschaftssitzung unterbringen lassen. Und natürlich um jede Menge Spaß zu haben \smiley. Die Mitfahrt ist für euch kostenlos}
}

\newglossaryentry{SWS}{
    name={SWS},
    first={Semesterwochenstunden (SWS)},
    description={Die Semesterwochenstunden geben an, wie viel Aufwand eine Veranstaltung ungefähr ist. Genau genommen wird nur die Anwesenheitszeit pro Woche angegeben: Ein Seminar hat dann meist 2 SWS, eine Vorlesung mit Übung dagegen 4+2 SWS}
}

\newglossaryentry{LP}{
    name={LP},
    first={Leistungspunkte (LP)},
    description={Leistungspunkte -- auch Credit Points (CP) oder ECTS (European Transfer Credit System) genannt -- sind ein Maß für den Aufwand eines Moduls. Dabei sollte ein LP etwa 30 Stunden Arbeit entsprechen -- das kommt aber öfter nicht als hin}
    % TODO adjust last sentence
}

\newglossaryentry{CP}{
    name={CP},
    first={Credit Points (CP)},
    description={siehe \Gls{LP}}
}
\newglossaryentry{ECTS}{
    name={ECTS},
    first={European Credit Transfer System},
    description={Die Erfindung, die uns Leistungspunkte beschert hat; Wird häufig auch dafür verwendet, siehe \Gls{LP}}
}


\newacronym{ZUV}{ZUV}{Zentrale Universitätsverwaltung}
\newacronym{StuWe}{StuWe}{\href{www.studierendenwerk.uni-heidelberg.de}{Studierendenwerk}}

% Für bestimmte Akronyme, die keine Mehrzahl haben, missbrauchen wir das
% Mehrzahlfeld um Genitiv (oder so… "des KIPs") zu speichern
\newacronym[\glslongpluralkey={Kirchhoff-Instituts für Physik},\glsshortpluralkey={KIP}]{KIP}{KIP}{Kirchhoff-Institut für Physik}
\newglossaryentry{PI}{
    name={PI},
    first={Phy\-si\-ka\-li\-sches Institut (PI)},
    description={Das Physikalische Institut wird des Öfteren auch Klaus-Tschira-Gebäude genannt. Es handelt sich dabei um eines der neueren Gebäude im Feld. Dort finden eure ersten verpflichtenden Physikpraktika statt, welche aus diversen Versuchen bestehen}
}
\newglossaryentry{ITP}{
    name={ITP},
    first={Institut für Theoretische Physik (ITP)},
    description={Das Institut für Theoretische Physik bevölkert die meisten Uni-Ge\-bäu\-de am Philosophenweg. Ihr findet die Dozentinnen und Mitarbeiterinnen in den Gebäuden Philosophenweg 12, 16 und 19. Außerdem gibt es insbesondere in der 12 noch einige Seminarräume}
}
\newacronym[\glslongpluralkey={Verkehrsverbundes Rhein-Neckar},\glsshortpluralkey={VRN}]{VRN}{VRN}{Verkehrsverbund Rhein-Neckar}

%Vorlesungstitel und deren Abkürzungen:
%Informatik
\newglossaryentry{IPI}{name=IPI, description={Die Vorlesung Einführung in die Praktische Informatik, wird manchmal auch Info1 genannt}}
\newglossaryentry{ITE}{name=ITE, description={Die Vorlesung Einführung in die Technische Informatik, kurz auch einfach Technische Info}}
\newglossaryentry{IDB1}{name=IDB, description={Die Vorlesung Datenbanken}}
\newglossaryentry{BeNe}{name=BeNe, description={Die Vorlesung Betriebssysteme und Netzwerke, im  Modulhandbuch mit IBN abgekürzt}}
\newglossaryentry{ISW}{name=ISW, description={Die Vorlesung Einführung in Software Engineering}}
\newglossaryentry{IPK}{name=IPK, description={Programmierkurs}}
\newglossaryentry{MafIn}{name=MafIn, description={Die Vorlesungen Mathematik für Informatiker 1 und 2, im  Modulhandbuch mit IMI abgekürzt}}
\newglossaryentry{AlDa}{name=AlDa, description={Die Vorlesung Algorithmen und Datenstrukturen, im Modulhandbuch mit IAD abgekürzt}}
%Mathe
\newglossaryentry{LA}{name=LA, description={Die Vorlesungen Lineare Algebra 1 und 2, selten auch LinA genannt}}
\newglossaryentry{Ana}{name=Ana, description={Die Vorlesungen Analysis 1 und 2 sowie Höhere Analysis -- letztere wird manchmal auch Analysis 3 genannt}}
\newglossaryentry{Num0}{name=Num0, description={Die Vorlesung Einführung in die Numerik}}
\newglossaryentry{WTheo0}{name=WTheo0, description={Die Vorlesung Einführung in die Wahrscheinlichkeitstheorie und Statistik}}
\newglossaryentry{FunkTheo}{name=Funktheo, description={Die Vorlesungen Funktionentheorie 1 und 2}}
%Physik
\newglossaryentry{Ex}{name=Ex, description={Die Vorlesungen Experimentalphysik 1 bis~5}}
\newglossaryentry{HoMa}{name=HöMa, description={Die Vorlesungen Höhere Mathematik für Physiker 2 und 3 (HöMa 1 gibt es nicht)}}
%Physik&Info
\newglossaryentry{Theo}{name=Theo, description={Physik: Die Vorlesungen Theoretische Physik 1 bis 4. Informatik: Die Vorlesung Einführung in die Theoretische Informatik, im Modulhandbuch mit ITH abgekürzt}}
\newglossaryentry{AP}{name=AP, description={Physik: Das Physikalische Praktikum für Anfänger 1 und 2, meist Anfängerpraktikum genannt oder wie im Modulhandbuch mit PAP1 bzw. PAP2 abgekürzt. Informatik: Das (Software)-Anfängerpraktikum}}
\newglossaryentry{FP}{name=FP, description={Physik: Das Physikalische Fortgeschrittenenpraktikum 1 und 2, meist Fortgeschrittenenpraktikum genannt. Informatik: Das (Software)-Fortgeschrittenenpraktikum}}

\newglossaryentry{PO}{
    name={PO},
    description={Die Prüfungsordnung regelt, wie euer Studium abläuft. Sie ist verbindlich. Die aktuelle Prüfungsordnung findet ihr auf der Website der jeweiligen Fakultät}
}

% Zeige alle Einträge an, auch die ohne Referenz
\glsaddall
