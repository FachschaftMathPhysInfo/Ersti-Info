% !TEX ROOT = ../ersti.tex
\section{Vorkurs Informatik}
Da im Informatikstudium in den ersten drei Semestern verhältnismäßig viele Mathe-Module belegt werden, ist der Mathe-Vorkurs auch für alle zukünftigen Infostudentinnen zu empfehlen. Es wird darüber hinaus eine Vorlesung zum Thema Algorithmen und Datenstrukturen geben, die auch für Mathestudentinnen interessant ist. Den Veranstaltungsplan (identisch mit dem Mathe-Vorkurs) findet ihr online\footnote{\label{info-vorkursplan}\url{https://mathphys.info/vorkurs/plan\#info}}.

\subsection{Programmiervorkurs}
Der Programmiervorkurs richtet sich an alle, die noch keinerlei Programmiererfahrung haben und vermittelt Grundlagen wie die Arbeit mit der Shell, Variablen, Schleifen, Funktionen, Objektorientierung in C++ und vieles mehr. Der Kurs findet vom 25. bis 29. September im \gls{Mathematikon} statt und ist auch für Mathematikerinnen zu empfehlen.

Die Kenntnisse werden in der \vl{Einführung in die praktische Informatik} (\gls{IPI}, erstes Semester) bzw. spätestens in der \vl{Einführung in die Numerik} (\gls{Num0}) hilfreich sein. Davon abgesehen ist es nützlich, erste Erfahrungen mit UNIX-artigen Betriebssystemen („Linux“) zu machen, da sie im naturwissenschaftlichen Bereich weit verbreitet sind.

Für den Programmiervorkurs ist aufgrund der begrenzten Plätze im Computer-Pool eine vorherige Anmeldung erforderlich! Den Link findet ihr ebenso im Vorkursplan.\footref{info-vorkursplan}
