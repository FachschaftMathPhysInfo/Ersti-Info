% !TEX ROOT = ../ersti.tex
\section{Rahmenprogramm}
\label{vorkurs-rahmenprogramm}

\subsection{Wanderungen}
An den Wochenenden -- Sonntag 09.10. und Sonntag 16.10. -- haben wir vor jeweils eine kleine Wanderung unternehmen. Die erste Tour führt auf den Heiligenberg, die zweite auf den Königstuhl -- die beiden Hausberge (Hügel) Heidelbergs. Im Anschluss an die erste Tour werden wir an der Thingstätte\footnote{alle Infos hierzu während der Wanderung erfragen} grillen, im Anschluss an die zweite Tour im Schlossgarten picknicken. Bringt euch für das Picknick alles Wichtige selbst mit.\\

Zusätzlich bieten wir am 29.09 eine spezielle Nachtwanderung an, die um 20 Uhr beginnt.

\noindent\emph{In der Regel treffen wir uns jeweils um 11 Uhr am Bismarckplatz. Coronabedingt kann es zu Änderungen kommen, die wir euch dann mitteilen.}

\vfill


\eject

\subsection{Spieleabende}
Das Abendprogramm wird spontan entschieden, meist handelt es sich um Gesellschaftsspiele -- bringt auf jeden Fall eigene Spiele mit, was auch immer ihr unter Spielen versteht.

Bisher sind Gesellschaftsspiele super angekommen -- da lernt man sich kennen -- Alternativvorschläge sind natürlich trotzdem immer gern gesehen. Falls euch etwas einfällt, meldet euch doch einfach, schließlich wird der Abend ja für euch veranstaltet.

Auch hier kann es zu Änderungen kommen, wir versuchen euch, eine digitale Möglichkeit anzubieten. Nähere Infos kommen noch.

\subsection{Brunch}
Am Samstag den 08. Oktober bereiten wir euch einen wunderschönen, leckeren Brunch im \gls{Mathematikon}, der das beste Frühstück wird, das ihr in euren ersten zwei Wochen bekommen werdet. Bringt bitte euer eigenes Geschirr/Besteck mit, damit wir Plastikquatsch sparen.


\vspace{4cm}

\begin{figure}[h]
\centering
\includegraphics[width=.7\linewidth]{bilder/su_doku.png}
\end{figure}
