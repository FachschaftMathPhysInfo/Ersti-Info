% !TEX ROOT = ../ersti.tex
\newpage
\section{Rahmenprogramm}

Das Rahmenprogramm wird von der Fachschaft MathPhysInfo organisiert und durchgeführt und ist sowohl für Mathematikerinnen, Physikerinnen und Informatikerinnen gedacht. Alle Details wie Uhrzeit und Startpunkt findet ihr im jeweiligen Online-Vorkursplan.

\subsection{Wanderungen}
An den Wochenenden -- Sonntag 08.10. und Sonntag 15.10. -- werden wir jeweils eine kleine Wanderung unternehmen. Die erste Tour führt auf den Heiligenberg, die zweite auf den Königstuhl -- die beiden Hausberge (Hügel) Heidelbergs. Im Anschluss an die erste Tour werden wir an der Thingstätte grillen, im Anschluss an die zweite im Schlossgarten picknicken. Bringt euch für das Picknick alles Wichtige selbst mit.

Zusätzlich bieten wir am 28.09 eine spezielle Nachtwanderung an, die um 20 Uhr beginnt.

\subsection{Spieleabende}
Die Spielabende werden vom AK Sven (Arbeitskreis Spiele, Veranstaltungen und Events mit Niveau) organisiert und finden am 6. und 12. Oktober um 18 Uhr im \gls{Mathematikon} statt. Das Abendprogramm wird spontan entschieden, meist handelt es sich um Gesellschaftsspiele. Bringt, auf jeden Fall eigene Spiele mit. Bisher sind Gesellschaftsspiele super ankommen. Falls euch Alternativvorschläge einfallen, meldet euch doch einfach, schließlich wird der Abend ja für euch veranstaltet.

\subsection{Brunch}
Am Samstag, den 7. Oktober, bereiten wir für euch einen wunderbaren, leckeren Brunch im \gls{Mathematikon} vor, der das beste Frühstück wird, das ihr in euren ersten zwei Wochen bekommen werdet. Bringt bitte euer eigenes Geschirr mit, damit wir Plastik sparen können.

\subsection{Campusführungen}
Am Mittwoch, den 4. Oktober, habt ihr ab 17 Uhr die Möglichkeit, an einer Campusführung durch das Neuenheimer Feld teilzunehmen. Wir laufen entlang von wunderschönen Bauten aus dem letztem Jahrhundert, die ihr tagtäglich in den nächsten Jahren bewundern dürft. Ein Faible für Beton sollte vorhanden sein.

\subsection{Fahrradtouren}
Am Dienstag, den 3. Oktober, finden mehrere Fahrradtouren im Umland von Heidelberg statt. Mehr Infos hierzu findet ihr im Online-Vorkursplan.

\subsection{Workshops}
Mittels zahlreichen Workshops könnt ihr euch die Vorkurs-Abende versüßen: sei es beim Weintasting, bei Stadtführungen, Linux-Partys oder anderen Aktivitäten. Schaut einfach mal in den Online-Vorkursplan rein.\footref{mathe-vorkursplan}
