% !TEX ROOT = ../ersti.tex
\section{Vorkurs Mathematik}
\label{vkmathe}

Der Vorkurs für die Studienanfänger der Mathematik und Informatik wird von der
Fachschaft MathPhys organisiert. Er soll helfen, euch den Einstieg in das
Studium zu erleichtern. Dazu gibt es vormittags fachliche Vorträge und Übungen
und nachmittags alles, was nicht direkt mit Mathe zu tun hat: wichtige Infos zu
wechselnden Themen und ein umfangreiches Kennenlern- und Spaßprogramm. Die
Veranstaltungen finden an unterscheidlichen Orten statt, die auf dem Plan im
Internet unter \url{http://mathphys.info/vorkurs/plan/} 
\marginQR{http://mathphys.info/vorkurs/plan/\#mathe}
abzulesen sind. Die Begrüßungsveranstaltung findet am Dienstag den 04.10. um 9
Uhr im Gebäude \gls{INF} 231 (Zoologie) in großen Hörsaal statt. Die
Informationsvorträge finden täglich 14 bis 15 Uhr im Mathematikon
(\gls{INF} 205) im Hörsaal statt.


\subsection{Upstream Mathematik-Mentorinnenprogramm}
Das Interdisziplinäre Zentrum für Wissenschaftliches Rechnen (IWR) an der Uni
Heidelberg bietet seit 2013 mit Upstream ein Mentorinnenprogramm für
Mathematik-interessierte Frauen.  Upstream ist ein Netzwerk für junge
Mathematikerinnen in allen Stufen der Ausbildung -- von Schülerinnen ab der 10.
Klasse, Studentinnen, Doktorandinnen, Nachwuchswissenschaftlerinnen bis hin zu
Professorinnen. In regelmäßigen Abständen organisieren wir Meet \&
Greet-Treffen, Public Lectures, Diskussionsrunden und Workshops zu Themen, in
denen Schlüsselqualifikationen und studiengangspezifische Inhalte vermittelt
werden. Dabei nehmen erfahrene Mathematikerinnen die Mentorinnenrolle ein. Sie
stehen den Teilnehmerinnen zur Seite, beraten und beantworten Fragen rund um
das Studium und dem Berufsleben.

Wenn Du Mitglied werden willst, schicke einfach etwa eine halbe Seite
Motivationsschreiben per E-Mail an: upstream@iwr.uni-heidelberg.de. Die Aufnahme
in das Programm ist nicht an Noten oder Wettbewerbserfolge geknüpft; Wir suchen
vor allem begeisterte Mathematikerinnen auf allen Ebenen.  Weitere Infos unter:
\url{www.mathcomp.uni-heidelberg.de}
