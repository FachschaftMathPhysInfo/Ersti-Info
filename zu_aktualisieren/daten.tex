% !TEX ROOT = ../ersti.tex
% hier werden die Posten definiert, damit bei Neuwahlen nicht der
% Text durchsucht werden muss

% BITTE BEACHTEN:
%  - Titel sind Böse. Keine Titel. Keine Titel.
%  - Keine abschließenden Leerzeichen. Das ist einfach falsch und flahsc.

%Physik
\newcommand{\dekanphysik}{Berges}
\newcommand{\dekanphysiklang}{Jürgen Berges}
\newcommand{\dekanphysikfoto}{bilder/berges_mon.jpg}
\newcommand{\prodekanphysik}{Andre Butz, Björn Malte Schäfer, Fred Hamprecht}
\newcommand{\prodekanphysikA}{A. Butz}
\newcommand{\prodekanphysikfotoA}{}
\newcommand{\prodekanphysikB}{B. Schäfer, F. Hamprecht}
\newcommand{\prodekanphysikfotoB}{}
\newcommand{\studiendekanphysik}{Björn Malte Schäfer}
\newcommand{\studiendekanphysikfoto}{bilder/schaefer_mon.jpg}
\newcommand{\pruefausschussvorsitzphysik}{Cornelis Dullemond}
\newcommand{\studienberatungphysik}{Tilman Enss, Stephanie Hansmann-Menzemer, Michael Hausmann, Selim Jochim und Andreas Just}
\newcommand{\dozentvorkurs}{Eduard Thommes und Joerg Jaeckel}
\newcommand{\bafogphysik}{Michael Hausmann}
\newcommand{\dekanatphysik}{Dewald-Klussmann}
\newcommand{\dekanatphysiktelefon}{+49\,62\,21 / 54\,-\,19\,648}
\newcommand{\gleichstellungsbeauftragtephysik}{Loredana Gastaldo}
\newcommand{\pruefsekphysik}{Frau Hiemenz und Frau Nerger}
\newcommand{\pruefsekphysikfotoA}{bilder/hiemenz_mon.jpg}
\newcommand{\pruefsekphysikA}{Frau Hiemenz}
\newcommand{\pruefsekphysikfotoB}{bilder/nerger_mon.jpg}
\newcommand{\pruefsekphysikB}{Frau Nerger}


%Mathe
\newcommand{\dekanmathe}{Schnörr}
\newcommand{\dekanmathelang}{Christoph Schnörr}
\newcommand{\dekanmathefoto}{bilder/schnoerr_mon.png}
\newcommand{\prodekanmathe}{Artur Andrzejak}
\newcommand{\prodekanmathefoto}{bilder/andrzejak_mon.jpg}
\newcommand{\studiendekanmathe}{Markus Banagl}
\newcommand{\studiendekanmathefoto}{bilder/banagel_mon.jpg}
\newcommand{\pruefausschussvorsitzmathe}{Rainer Dahlhaus, Robert Scheichl (Scientific Computing)}
\newcommand{\studienberatungmathe}{Hendrik Kasten, Michael Winckler (Scientific Computing)}
\newcommand{\gleichstellungsbeauftragtemathe}{Ekaterina Kostina}
\newcommand{\bafogmathe}{Markus Banagl}
\newcommand{\dekanatmathe}{Herr Schmidt}
\newcommand{\dekanatmathetelefon}{+49\,62\,21 / 54\,-\,14\,014}
\newcommand{\pruefsekmathe}{Frau Kiesel}
\newcommand{\pruefsekmathefoto}{bilder/kiesel_mon.jpg}

%Info
\newcommand{\studiendekaninformatik}{Filip Sadlo}
\newcommand{\studiendekaninformatikfoto}{bilder/sadlo_mon.jpg}
\newcommand{\studienberatunginformatik}{Wolfgang Merkle}
\newcommand{\pruefausschussvorsitzinformatik}{Michael Gertz}
\newcommand{\bafoginformatik}{Artur Andrzejak}
\newcommand{\pruefsekinfo}{Frau Sopka}
\newcommand{\pruefsekinfofoto}{bilder/sopka_mon.jpg}

%Uni
\newcommand{\gleichstellungsbeauftragteuni}{Christiane Schwieren}
\newcommand{\rektor}{Bernhard Eitel}
\newcommand{\kanzler}{Holger Schroeter}
%\newcommand{\vsstudiimsenat}{Philipp Strehlow}

%Rest
\newcommand{\redaktionsschluss}{16.09.2022}
%\newcommand{\anfi}{15.\,Oktober 2021}             % Termin für die AnfiFete
\newcommand{\fsraum}{\Gls{INF} 205, Raum 01.301}
\newcommand{\auflage}{600}                        % wie viel Erstiinfos sollen
% gedruckt werden

\newcommand{\semester}{Wintersemester 2022/23}    % bislang nur fuer titel.tex
%\newcommand{\drucktag}{2021-druck}
%\newcommand{\webtag}{2021-web}
\newcommand{\vorsitzVS}{Michèle Pfister und Peter Abelmann} % Mantelbogen -> Impressum

\newcommand{\mathphystheotermin}{21.\,Oktober 2022}

% diverse Minister

\newcommand{\wissenschaftsministerbawue}{Theresia Bauer (Grüne)}
\newcommand{\wissenschaftsministerbund}{Bettina Stark-Watzinger (FDP)}


% Geldbeträge
%\newcommand{\studiengebuehren}{500}
\newcommand{\verwaltungsbetrag}{70}
\newcommand{\studentenwerksbeitrag}{54}
\newcommand{\vsbeitrag}{10}
\newcommand{\vrnextbikebeitrag}{2,50}
\newcommand{\beitragssumme}{171,80}
%\newcommand{\quasimi}{280}

\newcommand{\lebenshaltungskosten}{\EUR{750 -- 860} }
\newcommand{\studentenwohnheim}{\EUR{170 -- 350} }
\newcommand{\bafoeghoechstsatz}{\EUR{934}}

% http://www.vrn.de/vrn/tickets/zeitkarten/studenten/vrn-semesterticket/index.html
\newcommand{\semesterticket}{180}
\newcommand{\sockelbeitrag}{35,30}

\newcommand{\vaterrheinspaghetti}{2,20} %Preis für einen Teller Bolo oder Tomatensauce, wenn man ein Getränk dazu bestellt


% Öffnungszeiten für diverses im Format \newcommand{\ortundzeit}{start & ende}

\newcommand{\ubAltAusMoFr}{9 & 20}
\newcommand{\ubAltAusSa}{13 & 17}
\newcommand{\ubFeldAusMoFr}{9 & 18}
\newcommand{\ubFeldAusSa}{13 & 17}

\newcommand{\ubAltLesMoFr}{8:30 & 1}
\newcommand{\ubAltLesSa}{9 & 1}
\newcommand{\ubFeldLesMoFr}{8:30 & 22}
\newcommand{\ubFeldLesSa}{9 & 22}

\newcommand{\mathekonMoFr}{9 & 21}
\newcommand{\mathekonSa}{9 & 15}

\newcommand{\physikMoFr}{9 & 12:30}
\newcommand{\physikSa}{13:30 & 16:30}

\newcommand{\auslandsinfooeff}{Mo: 10 -- 15, Di: 10 -- 16, Mi und Do: 10 -- 15, Fr: 10 -- 13} %https://www.uni-heidelberg.de/studium/imstudium/ausland/allgemein.html
\newcommand{\urrmelOeff}{Öffnungszeiten: Di von 16 - 20 Uhr \& Do von 18 - 20 Uhr}
%%% Local Variables:

%%% mode: latex
%%% TeX-master: "ersti"
%%% End:

