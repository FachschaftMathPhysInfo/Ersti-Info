\documentclass{mathphys-letter}
\usepackage[ngerman]{babel}
\usepackage[T1]{fontenc}
\usepackage[utf8]{inputenc}
\usepackage{datetime,microtype}
\setkomavar{fromname}{Tom Rix}
\setkomavar{fromemail}{trix@mathphys.info}
\setkomavar{subject}{Ablieferung Ersti-Info \the\year}
\setkomavar{myref}[ISSN]{2199-8310}
\setkomavar{invoice}[]{\null}
\begin{document}
\setkomavar{yourref}{Signatur Z 2013 B 3013}
\begin{letter}{Deutsche Nationalbibliothek\\Adickesallee 1\\60322 Frankfurt am Main}
        \opening{Sehr geehrte Damen und Herren,}
        anbei finden Sie die Pflichtexemplare unseres diesjährigen Erstsemester-Infohefts „Ersti-Info“.
        \closing{Mit freundlichen Grüßen,}
        \encl{zwei Pflichtexemplare „Ersti-Info \the\year“}
\end{letter}
\setkomavar{yourref}{Fachschaft-math-phy}
\begin{letter}{Badische Landesbibliothek\\Postfach 1429\\76003 Karlsruhe}
        \opening{Sehr geehrte Damen und Herren,}
        anbei finden Sie das Pflichtexemplar unseres diesjährigen Erstsemester-Infohefts „Ersti-Info“.
        \closing{Mit freundlichen Grüßen,}
        % lt. Mail vom 08.10.2014 besteht bei der Zeitschriftenstelle der württembergischen
        % Landesbibliothek kein Sammelinteresse, deshalb nur ein Exemplar!
        \encl{ein Pflichtexemplar „Ersti-Info \the\year“}
\end{letter}
\setkomavar{yourref}{}
\setkomavar{myref}{}
\begin{letter}{Universitätsarchiv Heidelberg\\Akademiestraße 4-8\\69117 Heidelberg\\\quad\\HAUSPOST}
        \opening{Sehr geehrte Damen und Herren,}
        anbei finden Sie ein Exemplar unseres diesjährigen Erstsemester-Infohefts „Ersti-Info“ für Ihr Archiv.
        \closing{Mit freundlichen Grüßen,}
        % sein WS 2014 schicken wir dem Uniarchiv je ein Exemplar
        % Ansprechpartnerin: Sabrina Zinke
        \encl{ein Archivexemplar „Ersti-Info \the\year“}
\end{letter}
\end{document}
