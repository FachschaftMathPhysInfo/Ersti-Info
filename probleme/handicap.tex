\section{Studieren mit Handicap}
Die Beauftragten für behinderte und chronisch kranke Studierende beraten Studieninteressierte und Studentinnen der Uni Heidelberg in zentralen Fragen zum Studium mit gesundheitlicher Beeinträchtigung.
Das Angebot richtet sich an alle Studentinnen, die chronisch gesundheitlich eingeschränkt sind. Darunter fallen auch „unsichtbare“ Erkrankungen wie AD(H)S, Autismus, Legasthenie oder Depressionen. Generell gilt: Nicht die Diagnose, sondern der Hilfebedarf ist relevant.

Viele hilfreiche Informationen zum Angebot des Handicap-Teams, zu Ansprechpartnerinnen und Abläufen findet ihr bereits im Internet\footnote{\url{https://www.uni-heidelberg.de/studiummithandicap}}. Mit allen weiteren Fragen rund um das Thema Studium mit gesundheitlicher Beeinträchtigung, beispielsweise Fragen zur Gebäudezugänglichkeit,  zur Verfügbarkeit von speziellen Hilfsmitteln an der Universität oder zu diversen behinderungs- bzw. krankheitsbezogenen Anträgen, könnt ihr euch gerne per E-Mail an das Handicap-Team wenden. Ihr bekommt dann zeitnah Rückmeldung und könnt gegebenenfalls einen telefonischen oder persönlichen Beratungstermin vereinbaren.
Eure persönlichen Daten werden selbstverständlich vertraulich behandelt.

Studentinnen mit nachgewiesenen Beeinträchtigungen bei Prüfungen, zum Beispiel Legasthenie, Prüfungsangst oder Lese-Rechtschreib-Schwäche können unter Vorlage einer ärztlichen Bestätigung bei ihrem jeweiligen Prüfungsausschuss einen Nachteilsausgleich beantragen. Für Klausuren kann das zum Beispiel eine Schreibverlängerung sein.
