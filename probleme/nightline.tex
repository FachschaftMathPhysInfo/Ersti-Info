% !TEX ROOT = ../ersti.tex
%\section{nightline}
\section{Nightline}%FIXME
Die Nightline ist eine telefonische Anlaufstelle von Studierenden für Studierende. Jede kann bei ihnen anrufen oder eine Mail schreiben, um über alles, was sie gerade beschäftigt, anonym und vertraulich zu reden. Egal ob Ersti oder Doktorandin, ob 18 oder 48 Jahre alt, egal, ob jemand einfach nur kurz etwas loswerden will oder gerade alles über einem zusammenbricht. Die Nightline bietet die Möglichkeit zum offenen Gespräch am späten Abend und nachts, wenn belastende Gefühle und Ängste bestehen, die tagsüber noch gekonnt verdrängt wurden und andere Gesprächspartnerinnen nicht oder nicht mehr erreichbar sind. Sie sind selbst Studierende, befinden sich also in einer ähnlichen Lebenslage wie ihr.

\begin{figure}[b]
    \centering
    \includegraphics[width=\linewidth]{bilder/nightline_logo.png}
\end{figure}


\begin{figure*}[!b]
\centering{
    \includegraphics[width=\textwidth]{bilder/cirith_ungol.png}
}
\end{figure*}

Für viele Anruferinnen ist es sicherlich einfacher, sich erstmal ebenfalls an Studierende zu wenden, da sich diese in einer ähnlichen Lebenssituation befinden. Die Universität und das Studierendenwerk bieten ihren Studentinnen zwar eine professionelle Betreuung in der psychotherapeutischen Beratungsstelle, doch ist dazu zunächst eine Terminvereinbarung nötig, weshalb es Studierenden erstmal leichter fällt, bei der Nightline anzurufen. Hier kann man sofort, wenn der Schuh drückt, der Kummer überhand nimmt oder man während einer nächtlichen Lerneinheit eine kleine Sinnkrise erleidet anrufen und seinen Problemen Platz machen.

Generell werden sie aus ganz unterschiedlichen Gründen angerufen. Häufige Themen sind zum Beispiel Probleme mit Freunden oder Familie, Stress in der Uni oder Liebeskummer. Außerdem helfen sie auch gerne bei allgemeinen Fragen zur Uni oder zum Unileben weiter.

Einer der Grundpfeiler ihrer Arbeit am Telefon ist die Vertraulichkeit. Die Themen, die am Telefon angesprochen werden, bleiben innerhalb der Nightline und kursieren nicht darüber hinaus. Die Anrufende muss ihren Namen nicht nennen und auch die Nightlinerin stellt sich nicht persönlich vor. Dadurch wollen sie einen vorurteilsfreien und neutralen Raum schaffen. Die Anrufende bleibt vollkommen anonym – am Telefon der Nightline ist noch nicht einmal dessen Rufnummer zu sehen. Die bestehende Anonymität kann der Anrufenden helfen, ihr Problem offen auszusprechen ohne das Gefühl zu haben, sich auf irgendeine Art und Weise der Nightlinerin offenbaren zu müssen.

Im Grunde verstehen sie ihre Aufgabe im Zuhören. Ihnen ist es wichtig, vorurteilsfrei, anonym und vertraulich zu sein. Die Briten haben den Leitspruch: „We listen, not lecture!“ Daran halten sie sich auch. Ihre Aufgabe ist es nicht, Ratschläge zu erteilen, sondern zuzuhören. Die Nightline wird von zwei Diplompsychologen betreut, die eine Schulung für Nightliner durchführen und mehrmals im Semester Supervisionen abhalten. Sie sind ausgebildete Therapeuten und vermitteln ihre Kenntnisse über Gesprächsführung an sie weiter.

Prinzipiell versuchen sie immer zu helfen und zuzuhören, doch wenn sie merken, dass eine Anrufende mehr braucht, als sie bieten können, oder sie ihre eigenen Grenzen überschritten sehen, leiten sie, wenn dies gewünscht wird, an professionelle Dienste weiter. Dazu haben sie eine Sammlung von verschiedenen Beratungsstellen und~Nummern angelegt, um beispielsweise an Suchtberatungsstellen, Trauerbegleitungen und fremdsprachige Telefonseelsorgen weiter verweisen zu können.

Die Nightline besteht aus etwa 30 ehrenamtlichen Mitarbeiterinnen aus verschiedensten Fachbereichen. Das Geschlechterverhältnis ist nahezu ausgeglichen, so dass sich in der Nightline das breite Spektrum der Studierendenschaft widerspiegelt. Mitmachen kann jede, die bereit ist, sich ehrenamtlich zu engagieren und einen Teil ihrer Freizeit in den Dienst anderer zu stellen. Ein vorgegebenes Studienfach für Nightliner gibt es nicht.

Die Nightline ist während der Vorlesungszeit täglich zwischen 21 Uhr abends und 2 Uhr nachts unter der Nummer 0\,62\,21 / 18\,47\,08, via Skype unter \emph{nightline.heidelberg} oder über das Mailsystem unter \url{www.nightline-heidelberg.de} erreichbar.
