%!TEX ROOT = ersti.tex

\usepackage[ngerman]{babel}
\usepackage{color}

\usepackage[utf8]{inputenc} % wir wollen das erstiinfo /richtig/
% kodiert haben

% Boolean, mit dem wir abfragen können, ob Druck- oder Webversion benötigt wird,
% um nicht alles in config_druck.tex bzw. config_web.tex angeben zu müssen.
\usepackage{ifthen}
\newboolean{druckversion}

% Diese Datei wird automatisch vom Makefile erstellt aus config_web.tex
% und config_druck.tex, je nachdem was man will
\input{config.tex}

\usepackage{eurosym} % eurozeichen kann man schon gut gebrauchen
\usepackage{amsmath}

\usepackage{memhfixc} % fix für die kollision memoir/hyperref

\usepackage{euler} % euler ist die schrift für die kapitelüberschriften
\usepackage{wasysym}  % für die Simleys
\usepackage[squaren]{SIunits}  % für "m^2" und andere Maßeinheiten
\usepackage{newcent}
\usepackage[pdftex]{graphicx} % hübsche bilder
\usepackage{epstopdf} % direkt eps einbinden können
\usepackage{booktabs} % hübsche tabellen
\usepackage[protrusion=true,expansion]{microtype} % hübscherer Blocksatz
\usepackage{multirow} % mehrzeilige Spalten in Tabellen
\usepackage[T1]{fontenc}
\usepackage{subcaption} %für subfigures
\usepackage{enumitem} % für descriptions mit style=unboxed
\setlist[itemize]{itemsep=1pt} % reduce spacing between bullet points
\usepackage{dblfloatfix}    % To enable figures at the bottom of page
\usepackage{wrapfig} % für bildumfließende texte

\usepackage[section=chapter,numberedsection=false,numberline,nonumberlist]{glossaries}
\renewcommand*{\glsclearpage}{\clearpage}
\makeglossaries
% Kurzanleitung:
% http://mirror.ctan.org/macros/latex/contrib/glossaries/glossariesbegin.pdf

% Syntax:
% \newglossaryentry{ label }{ settings }
% \newacronym{ label }{ abbrv }{ full }


% Beispiele: Einträge erstellen
% \newglossaryentry{electrolyte}{name=electrolyte, description={solution able to conduct electric current}}

% \newglossaryentry{oesophagus}{name=œsophagus, description={canal from mouth to stomach}, plural=œsophagi}

% \newacronym{label}{svm}{support vector machine}


% Beispiel: sich auf einen Glossareintrag beziehen
%       \gls{ label }
%       \glspl{ label }   % gibt die Pluralform aus


\newglossaryentry{c.t.}{name={c.t.}, description={lat.: cum tempore, mit Zeit, also eine viertel Stunde später}}
\newglossaryentry{s.t.}{name={s.t.}, description={lat.: sine tempore, ohne Zeit, also genau so, wie es da steht}}
\newglossaryentry{HEIDI}{name={HEIDI}, description={So heißt das Computersystem der \gls{UB}, mit dem ihr nach Büchern suchen und sie auch gleich vorbestellen könnt. Eure aktuellen Ausleihen werden auch dort gelistet, ebenso besteht die Möglichkeit dort insgesamt zweimal eure Ausleihfristen um je einen Monat zu verlängern. Die Adresse ist \url{heidi.ub.uni-heidelberg.de}, aber eine Suche nach \emph{heidi heidelberg} führt auch zum Ziel}}


\newglossaryentry{URZ}{
    name={URZ},
    first={Universitätsrechenzentrum (URZ)},
    description={Das Universitätsrechenzentrum stellt alles bereit, was sich grob mit dem Begriff \emph{Computer} in Verbindung bringen lässt. Siehe Artikel auf \autopageref{urz}}
}

\newglossaryentry{AM}{
    name={AM},
    first={Angewandte Mathematik},
    description={Die Angewandte Mathematik ist eines der Institute der Fakultät für Mathematik und Informatik und sitzt im Mathematikon}
}

\newglossaryentry{IAM}{
    name={IAM},
    first={Institut für Angewandte Mathematik},
    description={siehe \Gls{AM}}
}

\newglossaryentry{MI}{
    name={MI},
    first={Mathematisches Institut},
    description={Das Mathematisches Institut -- auch Reine Mathematik genannt -- ist eines der Institute der Fakultät für Mathematik und Informatik und sitzt im Mathematikon im 3. OG}
}

\newglossaryentry{RM}{
    name={RM},
    first={Reine Mathematik},
    description={siehe \Gls{MI}}
}

\newglossaryentry{IfI}{
    name={IfI},
    first={Institut für Informatik},
    description={Das Institut für Informatik ist eines der Institute der Fakultät für Mathematik und Informatik und sitzt im Mathematikon. Die Mitglieder decken unter anderem einen Großteil der Lehre im Bereich Informatik ab}
}

\newglossaryentry{Mathematikon}{
    name={Mathematikon},
    first={Mathematikon},
    description={Das Mathematikon (INF 205) ist das neu gebaute Ge\-bäu\-de an der Berliner Straße, in dem die Fakultät für Mathematik und Informatik und das IWR untergebracht sind. Dort finden auch die meisten Seminare, Übungen und kleinere Vorlesungen statt. Außerdem befindet sich dort die Bereichsbibliothek, einige Arbeitsplätze und natürlich die Fachschaft}
}

\newglossaryentry{FSK}{
    name={FSK},
    first={Fachschaftskonferenz (FSK)},
    description={Die Fachschaftskonferenz war die Vorgängerin des StuRa als Vertretung aller Studis vor der Wiedereinführung der Verfassten Studierendenschaft}
}

\newglossaryentry{StuRa}{
    name={StuRa},
    first={Studierendenrat (StuRa)},
    description={Der StuRa ist das Zentralorgan der Verfassten Studierendenschaft an der Uni Heidelberg und tagt zweiwöchentlich im neuen Hörsaal am Philosophenweg. Details in den Artikeln auf \autopageref{hopo} ff}
}

\newglossaryentry{ZFB}{
    name={ZFB},
    first={Zentrales Fachschaftenbüro (ZFB)},
    description={siehe \Gls{StuRa-Buero}}
}

\newglossaryentry{StuRa-Buero}{
    name={StuRa-Büro},
    first={StuRa-Büro},
    description={früher auch: \Gls{ZFB}. Büro- und Tagungsräume für den StuRa und andere studentische Gruppen. Auch die Sozialberatung u.ä. finden hier statt. Die genaue Adresse ist Albert-Ueberle-Straße 3-5, 69120 Heidelberg}
}


\newacronym{UB}{UB}{Universitätsbibliothek}
\newacronym{OPNV}{ÖPNV}{Öffentlicher Personen Nahverkehr}
\newacronym{INF}{INF}{im Neuenheimer Feld}
\newacronym{HS}{HS}{Hörsaal}
\newglossaryentry{FSWE}{
    name={FSWE},
    first={Fachschaftswochenende (FSWE)},
    description={Das Fachschaftswochenende veranstalten wir einmal pro Semester, um Themen diskutieren zu können, die sich nicht sinnvoll in einer Fachschaftssitzung unterbringen lassen. Und natürlich um jede Menge Spaß zu haben \smiley. Die Mitfahrt ist für euch kostenlos}
}

\newglossaryentry{SWS}{
    name={SWS},
    first={Semesterwochenstunden (SWS)},
    description={Die Semesterwochenstunden geben an, wie viel Aufwand eine Veranstaltung ungefähr ist. Genau genommen wird nur die Anwesenheitszeit pro Woche angegeben: Ein Seminar hat dann meist 2 SWS, eine Vorlesung mit Übung dagegen 4+2 SWS}
}

\newglossaryentry{LP}{
    name={LP},
    first={Leistungspunkte (LP)},
    description={Leistungspunkte -- auch Credit Points (CP) oder ECTS (European Transfer Credit System) genannt -- sind ein Maß für den Aufwand eines Moduls. Dabei sollte ein LP etwa 30 Stunden Arbeit entsprechen -- das kommt aber öfter nicht als hin}
    % TODO adjust last sentence
}

\newglossaryentry{CP}{
    name={CP},
    first={Credit Points (CP)},
    description={siehe \Gls{LP}}
}
\newglossaryentry{ECTS}{
    name={ECTS},
    first={European Credit Transfer System},
    description={Die Erfindung, die uns Leistungspunkte beschert hat; Wird häufig auch dafür verwendet, siehe \Gls{LP}}
}


\newacronym{ZUV}{ZUV}{Zentrale Universitätsverwaltung}
\newacronym{StuWe}{StuWe}{\href{www.studierendenwerk.uni-heidelberg.de}{Studierendenwerk}}

% Für bestimmte Akronyme, die keine Mehrzahl haben, missbrauchen wir das
% Mehrzahlfeld um Genitiv (oder so… "des KIPs") zu speichern
\newacronym[\glslongpluralkey={Kirchhoff-Instituts für Physik},\glsshortpluralkey={KIP}]{KIP}{KIP}{Kirchhoff-Institut für Physik}
\newglossaryentry{PI}{
    name={PI},
    first={Phy\-si\-ka\-li\-sches Institut (PI)},
    description={Das Physikalische Institut wird des Öfteren auch Klaus-Tschira-Gebäude genannt. Es handelt sich dabei um eines der neueren Gebäude im Feld. Dort finden eure ersten verpflichtenden Physikpraktika statt, welche aus diversen Versuchen bestehen}
}
\newglossaryentry{ITP}{
    name={ITP},
    first={Institut für Theoretische Physik (ITP)},
    description={Das Institut für Theoretische Physik bevölkert die meisten Uni-Ge\-bäu\-de am Philosophenweg. Ihr findet die Dozentinnen und Mitarbeiterinnen in den Gebäuden Philosophenweg 12, 16 und 19. Außerdem gibt es insbesondere in der 12 noch einige Seminarräume}
}
\newacronym[\glslongpluralkey={Verkehrsverbundes Rhein-Neckar},\glsshortpluralkey={VRN}]{VRN}{VRN}{Verkehrsverbund Rhein-Neckar}

%Vorlesungstitel und deren Abkürzungen:
%Informatik
\newglossaryentry{IPI}{name=IPI, description={Die Vorlesung Einführung in die Praktische Informatik, wird manchmal auch Info1 genannt}}
\newglossaryentry{ITE}{name=ITE, description={Die Vorlesung Einführung in die Technische Informatik, kurz auch einfach Technische Info}}
\newglossaryentry{IDB1}{name=IDB, description={Die Vorlesung Datenbanken}}
\newglossaryentry{BeNe}{name=BeNe, description={Die Vorlesung Betriebssysteme und Netzwerke, im  Modulhandbuch mit IBN abgekürzt}}
\newglossaryentry{ISW}{name=ISW, description={Die Vorlesung Einführung in Software Engineering}}
\newglossaryentry{IPK}{name=IPK, description={Programmierkurs}}
\newglossaryentry{MafIn}{name=MafIn, description={Die Vorlesungen Mathematik für Informatiker 1 und 2, im  Modulhandbuch mit IMI abgekürzt}}
\newglossaryentry{AlDa}{name=AlDa, description={Die Vorlesung Algorithmen und Datenstrukturen, im Modulhandbuch mit IAD abgekürzt}}
%Mathe
\newglossaryentry{LA}{name=LA, description={Die Vorlesungen Lineare Algebra 1 und 2, selten auch LinA genannt}}
\newglossaryentry{Ana}{name=Ana, description={Die Vorlesungen Analysis 1 und 2 sowie Höhere Analysis -- letztere wird manchmal auch Analysis 3 genannt}}
\newglossaryentry{Num0}{name=Num0, description={Die Vorlesung Einführung in die Numerik}}
\newglossaryentry{WTheo0}{name=WTheo0, description={Die Vorlesung Einführung in die Wahrscheinlichkeitstheorie und Statistik}}
\newglossaryentry{FunkTheo}{name=Funktheo, description={Die Vorlesungen Funktionentheorie 1 und 2}}
%Physik
\newglossaryentry{Ex}{name=Ex, description={Die Vorlesungen Experimentalphysik 1 bis~5}}
\newglossaryentry{HoMa}{name=HöMa, description={Die Vorlesungen Höhere Mathematik für Physiker 2 und 3 (HöMa 1 gibt es nicht)}}
%Physik&Info
\newglossaryentry{Theo}{name=Theo, description={Physik: Die Vorlesungen Theoretische Physik 1 bis 4. Informatik: Die Vorlesung Einführung in die Theoretische Informatik, im Modulhandbuch mit ITH abgekürzt}}
\newglossaryentry{AP}{name=AP, description={Physik: Das Physikalische Praktikum für Anfänger 1 und 2, meist Anfängerpraktikum genannt oder wie im Modulhandbuch mit PAP1 bzw. PAP2 abgekürzt. Informatik: Das (Software)-Anfängerpraktikum}}
\newglossaryentry{FP}{name=FP, description={Physik: Das Physikalische Fortgeschrittenenpraktikum 1 und 2, meist Fortgeschrittenenpraktikum genannt. Informatik: Das (Software)-Fortgeschrittenenpraktikum}}

% Zeige alle Einträge an, auch die ohne Referenz
\glsaddall


% alternative Fußnoten-Symbole
\makeatletter
\newcommand*{\@greek}[1]{\ensuremath{\ifcase#1 \or \alpha \or \beta \or \gamma \or \delta \or \varepsilon \or \zeta \or \eta \or \vartheta \or \iota \or \kappa \or \lambda \or \mu \or \nu \or o \or \pi \or \varrho \or \sigma \or \tau \or \upsilon \or \varphi \or \chi \or \psi \or \omega \or \varsigma \or \alpha \or \beta \or \gamma \or \delta \or \varepsilon \else \@cterr \fi}}
\newcommand*{\greek}[1]{\expandafter\@greek\csname c@#1\endcsname}
\makeatother
\renewcommand{\thefootnote}{\greek{footnote}}

% !TEX ROOT = ../ersti.tex
% hier werden die Posten definiert, damit bei Neuwahlen nicht der
% Text durchsucht werden muss

% BITTE BEACHTEN:
%  - Titel (z.B. Dr. Prof. etc.) sind Böse. Keine Titel. Keine Titel. Führt nur zu Problemen.
%  - Keine abschließenden Leerzeichen. Das ist einfach falsch und flahsc.

%Physik
\newcommand{\dekanphysik}{Butz}
\newcommand{\dekanphysiklang}{André Butz}
\newcommand{\dekanphysikfoto}{bilder/butz_mon.jpg}

\newcommand{\prodekanphysik}{Stephanie Hansmann-Menzemer und Tilman Plehn}
\newcommand{\prodekanphysikA}{S. Hansmann-Menzemer}
\newcommand{\prodekanphysikfotoA}{}
\newcommand{\prodekanphysikB}{T. Plehn}
\newcommand{\prodekanphysikfotoB}{}

\newcommand{\studiendekanphysik}{Norbert Christlieb}
\newcommand{\studiendekanphysikfoto}{bilder/schaefer_mon.jpg}
\newcommand{\studiendekanphysikemail}{n.christlieb@lsw.uni-heidelberg.de}

\newcommand{\pruefausschussvorsitzphysik}{Cornelis Dullemond}
\newcommand{\pruefausschussvorsitzphysikemail}{pavorsitz@zah.uni-heidelberg.de}

\newcommand{\studienberatungphysik}{Tilman Enss (theoretische Physik), Stephanie Hansmann-Menzemer (Experimentalphysik), Selim Jochim (Doppelfach und Lehramt) und Andreas Just (Astronomie)} %https://physik.uni-heidelberg.de/studienberatung
\newcommand{\studienberatungphysikersti}{Björn Malte Schäfer}
\newcommand{\studienberatungphysikemail}{bjoern.malte.schaefer@uni-heidelberg.de}

\newcommand{\bafogphysik}{Michael Hausmann}
\newcommand{\bafogphysikemail}{hausmann@kip.uni-heidelberg.de}

\newcommand{\dozentvorkurs}{Eduard Thommes und Joerg Jaeckel}

\newcommand{\dekanatphysik}{Dewald-Klussmann} %https://physik.uni-heidelberg.de/dekanat
\newcommand{\dekanatphysikemail}{dekanat@physik.uni-heidelberg.de}
\newcommand{\dekanatphysiktelefon}{+49\,62\,21 / 54\,-\,19\,648}

\newcommand{\gleichstellungsbeauftragtephysik}{Loredana Gastaldo}
\newcommand{\gleichstellungsbeauftragtephysikemail}{Loredana.Gastaldo@kip.uni-heidelberg.de}
%https://lsf.uni-heidelberg.de/qisserver/rds?state=wtree&search=2&trex=step&root220232=1|921857|130000&P.vx=mittel

\newcommand{\pruefsekphysik}{May-Britt Hiemenz und Birgit Jacob} %https://physik.uni-heidelberg.de/dekanat
\newcommand{\pruefsekphysikfotoA}{bilder/hiemenz_mon.jpg}
\newcommand{\pruefsekphysikA}{Frau Hiemenz}
\newcommand{\pruefsekphysikfotoB}{bilder/nerger_mon.jpg}
\newcommand{\pruefsekphysikB}{Frau Nerger}
\newcommand{\pruefsekphysikemail}{p_a_secr@physik.uni-heidelberg.de}
\newcommand{\pruefsekphysiktel}{+49\,62\,21 / 54\, -\,41\,24}
%\newcommand{\pruefsekphysikfotoC}{bilder/jacob_mon.jpg}
%\newcommand{\pruefsekphysikC}{Frau Jacob}


%Mathe
\newcommand{\dekanmathe}{Venjakob}
\newcommand{\dekanmathelang}{Ottmar Venjakob}
\newcommand{\dekanmathefoto}{bilder/venjakob2_mon.jpg}

\newcommand{\prodekanmathe}{Felix Joos}
\newcommand{\prodekanmathefoto}{bilder/joos_mon.jpg}

\newcommand{\studiendekanmathe}{Ekaterina Kostina}
\newcommand{\studiendekanmathefoto}{bilder/ekaterina_mon.jpg}
\newcommand{\studiendekanmatheemail}{ekaterina.kostina@iwr.uni-heidelberg.de}

\newcommand{\pruefausschussvorsitzmathe}{Alexander Schmidt}
\newcommand{\pruefausschussvorsitzmatheemail}{schmidt@mathi.uni-heidelberg.de}

\newcommand{\studienberatungmathe}{Hendrik Kasten}
\newcommand{\studienberatungmatheemail}{kasten@mathi.uni-heidelberg.de}

\newcommand{\gleichstellungsbeauftragtemathe}{Peter Bastian}
\newcommand{\gleichstellungsbeauftragtemathe}{peter.bastian@iwr.uni-heidelberg.de}
%https://lsf.uni-heidelberg.de/qisserver/rds?state=wtree&search=2&trex=step&root220232=1|921857|110000&P.vx=mittel
\newcommand{\bafogmathe}{Markus Banagl}
\newcommand{\bafogmatheemail}{banagl@mathi.uni-heidelberg.de}

\newcommand{\dekanatmathe}{Herr Schmidt}
\newcommand{\dekanatmathetelefon}{+49\,62\,21 / 54\,-\,14\,014}
%https://lsf.uni-heidelberg.de/qisserver/rds?state=wtree&search=2&trex=step&root220232=1|921857|110000&P.vx=mittel
% https://mathinf-old.uni-heidelberg.de/de/examcreditsmath
\newcommand{\pruefsekmathe}{Petra Kiesel}
\newcommand{\pruefsekmathefoto}{bilder/kiesel_mon.jpg}
\newcommand{\pruefsekmatheemail}{pruefungen.mathematik@mathinf.uni-heidelberg.de}
\newcommand{\pruefsekmathetel}{06221 54-14018}

%Info
\newcommand{\studiendekaninformatik}{Filip Sadlo}
\newcommand{\studiendekaninformatikfoto}{bilder/sadlo_mon.jpg}
\newcommand{\studiendekaninformatikemail}{studiendekan.informatik@mathinf.uni-heidelberg.de}

\newcommand{\studienberatunginformatik}{Wolfgang Merkle}
\newcommand{\studienberatunginformatikemail}{merkle@math.uni-heidelberg.de}

\newcommand{\pruefausschussvorsitzinformatik}{Michael Gertz}
\newcommand{\pruefausschussvorsitzinformatikemail}{gertz@informatik.uni-heidelberg.de}

\newcommand{\bafoginformatik}{Artur Andrzejak}
\newcommand{\bafoginformatikemail}{artur.andrzejak[at]uni-heidelberg.de}

% https://www.mathinf.uni-heidelberg.de/de/studium/informatik-studieren-in-heidelberg/pruefungsamt-fach-informatik
\newcommand{\pruefsekinfo}{Anke Sopka}
\newcommand{\pruefsekinfofoto}{bilder/sopka_mon.jpg}
\newcommand{\pruefsekinfoemail}{sekretariat@informatik.uni-heidelberg.de}
\newcommand{\pruefsekinfotel}{06221 54-14300}

%Uni
\newcommand{\gleichstellungsbeauftragteuni}{Christiane Schwieren} %https://www.uni-heidelberg.de/gleichstellungsbeauftragte/ueberuns/gleichstellungsbeauftragte.html
% https://www.uni-heidelberg.de/de/einrichtungen/rektorat
\newcommand{\rektor}{Frauke Melchior} 
\newcommand{\kanzler}{Holger Schroeter}
%\newcommand{\vsstudiimsenat}{Philipp Strehlow}

%Rest
\newcommand{\jahr}{2024}
\newcommand{\redaktionsschluss}{04.09.2024}
%\newcommand{\anfi}{15.\,Oktober 2021}             % Termin für die AnfiFete
% TODO: AnfiFete Termin
\newcommand{\fsraum}{\Gls{INF} 205, Raum 01.301}
\newcommand{\auflage}{500}                        % wie viel Erstiinfos sollen
% gedruckt werden

\newcommand{\semester}{Wintersemester 2024/25}    % bislang nur fuer titel.tex
%\newcommand{\drucktag}{2021-druck}
%\newcommand{\webtag}{2021-web}
\newcommand{\vorsitzVS}{Diana Zhunussova und Peter Abelmann} % Mantelbogen -> Impressum

% Zur Zeit wird der Termin für die AnfiFete nicht verwendet, da
% akfest.tex nicht eingebunden ist. 2023 war der Termin der MathPhysTheo zum Zeitpunkt der Finalisierung der Erst-Info noch nicht bekannt.
\newcommand{\mathphystheotermin}{21.\,Oktober 2022}

% diverse Minister
\newcommand{\wissenschaftsministerbawue}{Petra Olschowski (Grüne)}
\newcommand{\wissenschaftsministerbund}{Bettina Stark-Watzinger (FDP)}


% Geldbeträge
%https://www.uni-heidelberg.de/de/studium/studienorganisation/beitraege-gebuehren/studienbeitraege
%\newcommand{\studiengebuehren}{500}
\newcommand{\verwaltungsbetrag}{70}
\newcommand{\studentenwerksbeitrag}{66}
\newcommand{\vsbeitrag}{10}
\newcommand{\vrnextbikebeitrag}{2,55}
\newcommand{\theaterflatratebeitrag}{2,50}
\newcommand{\beitragssumme}{151,05}
%\newcommand{\quasimi}{280}

\newcommand{\lebenshaltungskosten}{\EUR{750 -- 860} }
\newcommand{\studentenwohnheim}{\EUR{214 -- 385} }
\newcommand{\bafoeghoechstsatz}{\EUR{934}}

% Semesterticket
\newcommand{\semesterticket}{180}
\newcommand{\sockelbeitrag}{35,30}

\newcommand{\vaterrheinspaghetti}{3} %Preis für einen Teller Bolo oder Tomatensauce, wenn man ein Getränk dazu bestellt


% Öffnungszeiten für diverses im Format \newcommand{\ortundzeit}{start & ende}
%https://www.ub.uni-heidelberg.de/allg/profil/adoeftel.html
\newcommand{\ubAltAusMoFr}{9 & 20}
\newcommand{\ubAltAusSa}{13 & 17}
\newcommand{\ubFeldAusMoFr}{9 & 18}
\newcommand{\ubFeldAusSa}{13 & 17}

\newcommand{\ubAltLesMoFr}{8:30 & 1}
\newcommand{\ubAltLesSa}{9 & 1}
\newcommand{\ubFeldLesMoFr}{8:30 & 22}
\newcommand{\ubFeldLesSa}{9 & 22}
%	https://www.mathinf.uni-heidelberg.de/de/facultylibrary -- gerade geht der Link nicht
\newcommand{\mathekonMoFr}{9 & 21}
\newcommand{\mathekonSa}{9 & 15}

\newcommand{\physikMoFr}{9 & 12:00}
\newcommand{\physikSa}{14:00 & 16:30}

\newcommand{\auslandsinfooeff}{Mo: 10 -- 15, Di,Mi und Do: 10 -- 14, Fr: 10 -- 12} %https://www.uni-heidelberg.de/studium/imstudium/ausland/allgemein.html -- mgl veraltet
%https://www.uni-heidelberg.de/de/studium/studium-international/studium-im-ausland/infocenter-studium-im-ausland
\newcommand{\urrmelOeff}{Öffnungszeiten: Di von 16 - 20 Uhr \& Do von 16 - 20 Uhr}
%%% Local Variables:

%%% mode: latex
%%% TeX-master: "ersti"
%%% End:

 %% hier sind die ganzen wichtigen
%% aktualisierungswürdigen daten
%% drin. das ist wichtig!!!

%%%%%% eigens definierter krimskrams
\newenvironment{spalten}{}{}
\newcommand{\nurimdruck}[1]{#1}
\newcommand{\email}[1]{\href{mailto:#1}{#1}}
\urlstyle{same}
\newcommand{\vl}[1]{\textbf{\emph{#1}}}

%%%% chapterstyle
\makechapterstyle{mathphys}{
    \renewcommand{\chaptitlefont}{
        \checkoddpage
        \ifoddpage
            \fontsize{28}{28} \selectfont \flushright\sffamily
        \else
            \fontsize{28}{28} \selectfont \flushleft\sffamily
        \fi
    }
    \renewcommand{\printchaptertitle}[1]{
        \checkoddpage
        \ifoddpage
            % ungerade seiten
            \colorbox{kapitelhintergrund}{
                \parbox{19.2cm}{
                    \vspace{0.5cm}
                    \hspace{0.3cm}
                    \parbox{0.985\textwidth}{%
                        \textcolor{white}{\chaptitlefont \textsc{##1}}%
                    }
                    \raisebox{-9mm}{
                        \makebox[0pt][l]{
                            \resizebox{!}{42pt}{\chapnumfont
                                % das ist nicht nett. stimmt. tut mir leid. ehrlich.
                                \hspace*{-1.8mm}
                                \textcolor{white}{$\thechapter$}}
                        }
                    }

                    \vspace{0.5cm}
                }
            }
        \else
            % gerade seiten
            \hspace*{-\foremargin}
            \hspace*{-5.6mm} % keine Ahnung warum die fehlen
            \colorbox{kapitelhintergrund}{
                \parbox{19.2cm}{
                    \vspace{0.5cm}
                    \hspace{0.5cm}
                    \raisebox{-7mm}{
                        \makebox[0pt][l]{
                            \resizebox{!}{42pt}{
                                \textcolor{white}{$\thechapter$}
                            }
                        }
                    }
                    \ifthenelse{\value{chapter}>9}{\hspace*{3.2cm}}{\hspace*{2.3cm}}
                    \parbox{0.985\textwidth}{
                        \textcolor{white}{\chaptitlefont \textsc{##1}}
                    }
                    % keine Ahnung warum die \colorbox sonst höher wird
                    \vspace{0.5cm}
                }
            }
        \fi
    }

    \renewcommand{\chapnumfont}{\sffamily\bfseries}
    \renewcommand{\printchaptername}{}
    \renewcommand{\afterchapternum}{}
    \renewcommand{\printchapternum}{}
    \renewcommand{\printchapternonum}{}
}

\chapterstyle{mathphys}

\newcommand{\mathphyssubsubsec}[1]{\noindent\sffamily\bfseries\textcolor{sectiontextfarbe}{#1}}%
\newcommand{\mathphyssubsec}[1]{\large\mathphyssubsubsec{#1}}
\newcommand{\mathphyssec}[1]{\LARGE{\mathphyssubsubsec{#1}}}
\setsecheadstyle{\mathphyssec}
\setsubsecheadstyle{\mathphyssubsec}
\setsubsubsecheadstyle{\mathphyssubsubsec}
\setparaheadstyle{\sffamily\textbf}
\setsecnumformat{}

\pagestyle{plain}
\makeoddfoot{plain}{}{}{\thepage}
\makeevenfoot{plain}{\thepage}{}{}

%%%% Impressum
\newcommand{\impressum}[2]{
    \vspace*{\fill}
    \begin{tabular*}{0.77\textwidth}{ll}
        \multicolumn{2}{l}{
            \parbox{0.77\textwidth}{
                Der Redaktionsschluss für diesen Text war am \redaktionsschluss. Wir freuen uns
                sehr über Kommentare, Anregungen, Verbesserungsvorschläge,
                Mitarbeit und Kuchen -- melde dich bei
                \email{fachschaft@mathphys.info}.\\

                Um die Lesbarkeit zu gewährleisten, nutzt dieses Schriftstück das generische Femininum. Soweit personenbezogene Bezeichnungen in weiblicher Form aufgeführt sind, beziehen sie sich auf alle Geschlechter in gleicher Weise.\\

                \textbf{Danksagung}: Die Ersti-Info \jahr \, wäre ohne die Unterstützung von zahlreichen Helferinnen nicht möglich gewesen. Wir möchten uns insbesondere bedanken bei: \textit{Max Wipplinger} für umfangreiche textuelle Überarbeitungen und technische Umsetzung, \textit{Arvin Nikou} für Mitarbeit in der Organisation und die Verfassung eines Artikels und \textit{Lorena Ergenzinger} für die Leitung der neuen Ersti-Info. Außerdem danken wir \textit{Dominik Plein} für die Einarbeitung und Wissensübergabe. Danke an die Fachschaft und die Dekanate MathInf und Physik, für die Versorgung mit mehr als aktuellen Informationen, insbesondere an \textit{Frau Dewald} für die Organisation des Druckereiprozesses. Ein großes Dankeschön gebührt auch den vielen Studis der letzten Jahre, die die vorherigen Ersti-Infos geschrieben haben und deren Texte in großen Teilen übernommen wurden! Wenn ihr die Ersti-Info gerade in den Händen haltet, dann habt ihr wohl auch eine Erstitüte bekommen, darum mussten sich natürlich auch Leute kümmern. Ganz herzlichen Dank also an \textit{Adam Fuge} für die Organisation und an \textit{Bjela Böttcher}, \textit{Caroline Niewa}, \textit{Aaron Fath} und \textit{Holger Heck} für die Umsetzung. In diesem Zug danke an \textit{Michael} vom Fantasy Kolosseum \url{https://www.kolosseum.de/}, einem Laden für Brettspiele, Trading Card Games, Warhammer und Pen\&Paper Rollenspiele in Bergheim, der dafür gesorgt hat, dass Demospiele in den Tüten landen. Und zu guter Letzt: vielen Dank an euch als Leserinnen, dass ihr euch die Zeit nehmt, diese Ersti-Info zu lesen. Wir hoffen, dass sie euch hilft, euch an der Uni zurechtzufinden und wünschen euch einen guten Start ins Studium!
                }
            \vspace{5cm}
        }\\
        Herausgeberin: & Fachschaft MathPhysInfo\\
        & Im Neuenheimer Feld 205, Raum 1.301\\
        & 69120 Heidelberg\\
        & www.mathphys.info\\
        & vertr. durch den Vorsitz der\\
        & Verfassten Studierendenschaft\\
        & \vorsitzVS\\
        & Albert-Überle-Straße 3-5\\
        & 69120 Heidelberg
    \end{tabular*}

    \vfill

    \begin{textblock*}{202mm}[0,1](8mm,290mm)
        \begin{flushleft}
            \footnotesize\noindent
            ISSN: \ifthenelse{\boolean{druckversion}}{2199-8310}{2199-8329}\\
            Auflage: \auflage\ Stück\\
            Letzter Commit: \input{GITHASH}\\
            Letzte Änderung: \input{GITDATE}\\
            Source Code: \ifthenelse{\boolean{druckversion}}{https://github.com/FachschaftMathPhysInfo/Ersti-Info}{\url{https://github.com/FachschaftMathPhysInfo/Ersti-Info}}\\

        \end{flushleft}
    \end{textblock*}

    \begin{textblock*}{203mm}[0,1](-7mm,290mm)
        \begin{flushright}
            \footnotesize
            \href{https://mathphys.info/}{Cover und Rückseite: Lorena Ergenzinger}\\ %2012
            \href{https://xkcd.com/}{xkcd Comics: Randall Munroe} \href{https://creativecommons.org/licenses/by-nc/2.5/}{(CC-BY-NC)}\\
            \href{https://abstrusegoose.com/}{Abstruse Goose Comics: lcfr} \href{https://creativecommons.org/licenses/by-nc/3.0/us/}{(CC-BY-NC)}\\
            \href{https://nfccomic.com}{Not From Concentrate Comics: \copyright{} Thomas Dobrosielski}\\
            \href{https://www.phdcomics.com/}{Piled Higher and Deeper (PhD) Comics: \copyright{} Jorge Cham}\\
            \href{https://www.openstreetmap.org/}{Landkarten: \copyright{} OpenStreetMap contributors}
        \end{flushright}
    \end{textblock*}
}

%%%%%%%%% Suche nach Grafiken in ./bilder und  . :
\graphicspath{{./bilder/}{./}}

%%%%%%%%% Silbentrennung
\input{silbentrennung}

%%% Local Variables:
%%% mode: latex
%%% TeX-master: ersti
%%% End:
