% !TEX ROOT = ../ersti.tex
\section{Verkehrsmittel in Heidelberg}
\label{verkehrsmittel}

Besorg' dir ein Fahrrad! Es ist das schnellste, zuverlässigste und auch günstigste\footnote{Wenn du dich an die StVo hälst -- oder dich nicht erwischen lässt} Verkehrsmittel in Heidelberg. Für fast alle wichtigen Routen gibt es Radstreifen oder ausgeschilderte Wege über Seitenstraßen, sodass man vom Berufsverkehr weitgehend verschont bleibt. Selbst für die Distanzen auf dem Campus kann sich eine Anschaffung lohnen: Vom Hörsaal über die \gls{UB} zur Mensa läuft man gerne 15 Minuten. Auch abends, wenn die Bahnen nur noch halbstündlich oder gar nicht mehr fahren, ist das Fahrrad oft die bessere Alternative. Und falls es mal kaputt geht, schaust du einfach im \emph{URRmEL}\footnote{\urrmelOeff} vorbei, der Uni-eigenen Fahrradwerkstatt. Da musst du dein Fahrrad zwar selber reparieren, dafür sind aber immer Leute da, die dir sagen, wie das geht und dir auch mal helfen. An Werkzeugen und Ersatzteilen herrscht auch kein Mangel.

\label{nextbike}
Wenn du am Anfang kein Fahrrad hast oder dein Fahrrad verloren gegangen ist, gibt es seit dem Sommersemester 2018 die Möglichkeit, für 30 Minuten ein Fahrrad kostenlos von VRNnextbike\footnote{Finanziert wird dies durch einen Solidarbeitrag ähnlich zum Semesterticket in Höhe von \EUR{\vrnextbikebeitrag}.} zu leihen. Nach den kostenlosen 30 Minuten kostet jede halbe Stunde 50 Cent. Wird das Fahrrad vorher abgegeben, so wird für die Benutzerin für 15 Minuten die kostenlose Mietoption \emph{gesperrt} (danach kann man aber wieder kostenlos weiterfahren). Die Rückgabe und Ausleihe ist an diversen Standorten in Heidelberg und Umgebung über Hotline, App oder an der Station selbst möglich. Auch im Urlaub in einigen Städten Deutschlands kann der kostenlose Grundtarif verwendet werden\footnote{Mehr Informationen über Nextbike findest du beim StuRa unter \url{https://stura.uni-heidelberg.de/angebote/vrnnextbike/}}.

Seit Anfang des Jahres gibt es für alle unter 27 Jahren, die an einer Hochschule in Baden-Württemberg studieren, das sogenannte \emph{Jugendticket BW}. Über die Uni kann man damit für \EUR{147,20} 6 Monate lang den kompletten Nah- und Regionalverkehr in BW inklusive des VRN- und VRS-Gebiets nutzen. Ab Dezember 2023 wird das Ticket außerdem (ohne Preisaufschlag!) auf ein Deutschland-Ticket geupgraded. Damit ist es wohl die günstige Möglichkeit mit dem Regionalverkehr zu reisen.

% Abendregelung gilt seit WS 2023 nicht mehr (ab 19 Uhr kostenlos fahren nur mit Studienausweis).

Zu guter Letzt bleibt noch das Auto... Auto? Wie 20. Jahrhundert bist du denn? Sofern du nicht unglaublich viel Geld, Zeit und Nerven hast, um die Parkplatzsuche und den Berufsverkehr zu bewältigen, kann man vom Auto nur abraten. Zum Pendeln in die Heimat am Wochenende ist es vielleicht noch zu gebrauchen, sofern du hier leicht Zugang zu einem Parkplatz hast. Wer täglich pendeln will oder muss, parkt jedoch besser weit außerhalb und legt den Rest mit OPNV oder Fahrrad zurück und ist damit deutlich umweltfreundlicher unterwegs.
