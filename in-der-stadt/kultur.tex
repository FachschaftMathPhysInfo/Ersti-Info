% !TEX ROOT = ../ersti.tex

\section{Kultur für Studis}

Auch wenn der regelmäßige Besuch der oben genannten Trinkkulturstätten essentieller Bestandteil eines gesunden Studierendenlebens ist, so sind diese nicht in der Lage, unseren intellektuellen Durst zu stillen. Daher sind auch Kinos, Theater und Musikfestivals ein wichtiger Teil des studentischen kulturellen Lebens in Heidelberg.

\subsection{Kinos}
\begin{itemize}
\item Das \emph{Gloria \& Gloriette}-Filmkunsttheater in der Hauptstraße zwischen Uniplatz und Marktplatz ist vor allem für Arthouse-Fans interessant, die Filme abseits des Mainstreams sehen möchten. Regelmäßig sind für Sondervorstellungen auch Regisseurinnen anwesend, welche im Nachklang zum Film mit den Zuschauerinnen über das Gesehene diskutieren. Kinotag ist montags, der Eintritt kostet dann \EUR{6,50} statt regulär \EUR{8}.
\item \emph{Die Kamera} in der Brückenstraße wird von der gleichen Betreiberin wie das \emph{Gloria} geführt, das aktuelle Programm findet man dementsprechend auf der gleichen Website.
\item Im bereits genannten Karlstorbahnhof befindet sich auch das \emph{Karlstorkino}, hier bekommt man Arthouse-Kino aus der ganzen Welt im Original mit Untertiteln vorgeführt, es werden aber auch regelmäßig große Klassiker der Filmgeschichte auf die Leinwand gebracht. Daher insbesondere für geübte Kinozuschauerinnen geeignet.
\item Enthusiastinnen von Blockbuster-Produktionen müssen seit 2018 nicht mehr nach Mannheim oder Walldorf fahren, sondern können nun im \emph{Luxor-Filmpalast} bei der Czerny-Brücke in den Genuss besagter Produkte der Kulturindustrie kommen. Das Kino bietet eine große Anzahl Kinosäale und hat auch 3D-Filme im Programm, für den Besuch muss man jedoch tief ins Portmonnaie greifen (Abendvorstellungen ab \EUR{11}, Kinotag montags ab \EUR{7}).
\end{itemize} 

\subsection{Theater}
Sehr zu empfehlen ist das \emph{Theater und Orchester Heidelberg}. Wer denkt, dass ein Konzert, Theater oder Opernbesuch für Studis unbezahlbar ist, irrt gewaltig. Beim Heidelberger Theater zahlen Studierende für alle Preiskategorien nur den halben Preis, sodass man oft für nur einen einstelligen Betrag einen denkwürdigen Abend genießen kann\footnote{Von der letzten Reihe aus lässt es sich für uns junge Menschen auch noch sehr gut sehen.}. Von Mozart bis Houellebecq über Brecht und Schiller ist für jeden etwas dabei, man sollte jedoch frühzeitig Karten reservieren. Ein Highlight sind im Sommer die Schlossfestspiele.

\subsection{Festivals}

\begin{itemize}
\item Beim klassischen Musikfestival \emph{Heidelberger Frühling}, welches sich über März und April erstreckt, gibt es Studitickets an der Abendkasse für \EUR{8}.
\item Im Rahmen des \emph{Queer Festival Heidelberg} im Mai finden nicht nur zahlreiche interessante Lesungen, Filmvorstellungen, Partys und Kunstvorstellungen statt, sondern auch Konzerte aller möglichen Musikrichtungen.
\item Die Konzerte des internationalen \emph{Enjoy Jazz Festival} finden im Oktober und November in der gesamten Rhein-Neckar-Region statt. Neben Szenegrößen werden auch noch weniger bekannte Künstlerinnen eingeladen, sodass ein möglichst breites Spektrum des Jazz abgedeckt wird, wobei auch Wert darauf gelegt wird, Künstlerinnen aus angrenzender oder gar nicht zuordenbarer Musik zu buchen.
\end{itemize}
